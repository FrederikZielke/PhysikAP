\section{Durchführung}
\label{sec:Durchführung}

\subsection{Aufbau}
\label{sec:Aufbau}

Zu Beginn des Experiments muss die Filterkurve des Selektivverstärkers aufgenommen werden.
Dazu wird an den Selektivverstärker ein Sinusgenerator angeschlossen.
Der Verstärker wird auf eine Verstärkung von $0.1$ eingestellt und auf eine Güte von $20$.
Der Sinusgenerator wird auf eine Frequenz von $\SI{22}{\kilo\hertz}$ und eine Stromstärke von $\SI{1}{volt}$ gestellt.
Der Ausgang des Selektivverstärkers wird an ein Spannungsmessgerät angeschlossen.
Die Frequenz des Sinusgenerators wird in sinnvollen Schrittgrößen erhöht und die Spannung am Spannungsmessgerät wird notiert.
Da ein exponentieller An- und Abstieg erwartet wird, wird die Frequenz um den Peak herum in $\SI{10}{\hertz}$ Schritten gemessen,
während sie weiter weg in größeren Schritten gemessen wird.
\\
\\
\subsection{Messung}
\label{sec:Messung}

Die Signalfrequenz des Sinusgenerators wird genau auf die vorher ermittelte Durchlassfrequenz des Selektivversterkärkers eingestellt.
Die Verstärkung des Selektivverstärkers wird auf $1$ eingestellt.
Der Sinusgenerator wird an die Brückenschaltung angeschlossen. Der Ausgang der Brückenschaltung wird an den Selektivversterkärker geschlossen.
Die Widerstände der Brückenschaltung werden so eingestellt, dass die Brückenspannung minimal wird.
Die zu untersuchende Probe wird in die Spule der Brückenschaltung eingesetzt. 
Der Widerstand der Brückenschaltung wird so eingestellt, dass die Brückenspannung wieder minimal wird.
Dieser Vorgang wird drei mal Wiederholt. Die Widerstände und Ausgangsspannungen werden notiert.
Der Vorgang wird für zwei weitere Proben wiederholt.
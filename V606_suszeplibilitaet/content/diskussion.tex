\section{Diskussion}
\label{sec:Diskussion}
Die Abweichungen der berechneten Größen von der Theorie werden nach 
\begin{equation}\label{1}
    \Delta = |\frac{exp - theo}{theo} \cdot 100|
\end{equation}
berechnet.
\subsection{Güte des Selektivverstärkers}
Am Selektivverstärker wurde eine Güte von $Q = 20$ eingestellt. Die durch die Durchlasskurve bestimmte Güte $Q_{exp} = \SI{63.15(0.26)}{}$ ist somit deutlich höher.
Das führt dazu, dass die Störspannungen besser rausgefiltert werden können und somit die weiteren Messergebnisse genauer sind.

\subsection{Suszeptibilität}
In \autoref{sec:sus} wurde eine theoretische Vorhersage für die jeweiligen Suszeptibilitäten berechnet. Diese Werte werden nun genutzt, um die Genauigkeit des 
Experiments zu untersuchen. Mit \autoref{1} ergeben sich die prozentualen Abweichungen.
\begin{table}[H]
    \centering
    \caption{$\chi_{theorie}$, $\chi_{exp}$ und prozentuale Abweichung.}
    \begin{tabular}{c c c c}
        \toprule
        {Ion-Verbindung} & {$\chi_{theorie}$} & {$\chi_{exp}$} & {$\Delta/\%$} \\
        \midrule
        $\symup{Dy_2O_3}$ & $\SI{0.02540(0.00009)}{}$ & $\SI{0.0240(0.0006)}{}$ & 5.5\\
        $\symup{Nd_2O_3}$ & $\SI{0.00302(0.00001)}{}$ & $\SI{0.0016(0.0004)}{}$ & 47.0\\
        $\symup{Gd_2O_3}$ & $\SI{0.01379(0.00004)}{}$ & $\SI{0.0119(0.0006)}{}$ & 15.9\\
        \bottomrule
    \end{tabular}
    \label{tab:dis}
\end{table}
Die Abweichungen von Dysprosiumoxid und Gadoliniumoxid fallen mit $\SI{5.5}{\%}$ und $\SI{15.9}{\%}$ moderat aus. Bei Neodymoxid fällt die Abweichung mit $\SI{47.0}{\%}$
größer aus. 
Für die ersten beiden genannten lassen sich die Abweichungen noch mit Störspannungen und kleinen Ungenauigkeiten beim Abgleichen der Brückenschaltung zurückführen.
Bei Neodymoxid sind dies zwar auch Gründe, aber diese erklären nicht die größere Abweichung. Auffällig ist, dass bei Neodymoxid die Widerstandsdifferenz deutlich kleiner als
bei den anderen beiden Oxiden war. Das bedeutet, dass selbst wenn die absoluten Abweichungen bei allen Oxiden gleich groß sind, die Abweichungen bei Neodymoxid relativ 
mehr Einfluss auf das Ergebnis haben. Generell lässt sich dieser Fakt vermuten, da die Oxide mit größerem $\Delta_R$ auch genauer werden.

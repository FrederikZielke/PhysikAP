\section{Auswertung}
\label{sec:Auswertung}
Die in \autoref{sec:Auswertung} gezeigten Grafiken und Rechnungen sind mithilfe der Python-Bibliotheken Matplotlib \cite{matplotlib}, Scipy \cite{scipy} und Numpy \cite{numpy}
erstellt worden.

\subsection{Bestimmung der Filterkurve des Selektivverstärkers}
Die Messwerte des ersten Versuchteils sind in \autoref{tab:guete} aufgelistet.

\begin{table}[H]
  \centering
  \caption{Messwerte Frequenz $f$ und Ausgangspannung $U_A$.}
  \begin{tabular}{c c}
      \toprule
      {$f\,$[Hz]} & {$U_A\,$[V]}\\
      \midrule
      21780 & 4.3 $\pm 0.05$ \\
      21800 & 4.5 $\pm 0.05$\\
      21820 & 4.6 $\pm 0.05$\\
      21850 & 4.6 $\pm 0.05$\\
      21880 & 4.55$\pm 0.05$ \\
      21900 & 4.35$\pm 0.05$ \\
      21950 & 3.95$\pm 0.05$ \\
      22000 & 3.5 $\pm 0.05$\\
      22050 & 3.1 $\pm 0.05$\\
      22100 & 2.7 $\pm 0.05$\\
      22150 & 2.35$\pm 0.05$ \\
      22200 & 2.1 $\pm 0.05$\\
      22250 & 1.9 $\pm 0.05$\\
      22300 & 1.7 $\pm 0.05$\\
      22350 & 1.55$\pm 0.05$ \\
      22400 & 1.4 $\pm 0.05$\\
      22450 & 1.3 $\pm 0.05$\\
      22500 & 1.2 $\pm 0.05$\\
      22600 & 1.05$\pm 0.05$ \\
      22700 & 0.9 $\pm 0.05$\\
      22800 & 0.8 $\pm 0.05$\\
      22900 & 0.75$\pm 0.05$ \\
      23000 & 0.7 $\pm 0.05$\\
      24000 & 0.35$\pm 0.05$ \\
      25000 & 0.25$\pm 0.05$ \\
      27000 & 0.15$\pm 0.05$ \\
      21750 & 4.0 $\pm 0.05$\\
      21720 & 3.65$\pm 0.05$ \\
      21700 & 3.4 $\pm 0.05$\\
      21650 & 2.9 $\pm 0.05$\\
      21600 & 2.5 $\pm 0.05$\\
      21550 & 2.2 $\pm 0.05$\\
      21500 & 1.95$\pm 0.05$ \\
      21450 & 1.7 $\pm 0.05$\\
      21400 & 1.55$\pm 0.05$ \\
      21300 & 1.2 $\pm 0.05$\\
      21200 & 1.1 $\pm 0.05$\\
      21100 & 1.0 $\pm 0.05$\\
      21000 & 0.85$\pm 0.05$ \\
      20800 & 0.75$\pm 0.05$ \\
      20000 & 0.4 $\pm 0.05$\\
      19000 & 0.2 $\pm 0.05$\\
      17000 & 0.1 $\pm 0.05$\\
      \bottomrule
  \end{tabular}
  \label{tab:guete}
\end{table}

Um die Güte des Selektivverstärker zu bestimmen werden die Messwerte $f$ gegen das Spannungsverhältnis $\frac{U_A}{U_E}$ aufgetragen. $U_A$ ist die 
Ausgangspannung und $U_E = \SI{1}{V}$ die Eingangsspannung. Da beim Ablesen der Spannung $U_A$ eine analoge Anzeige verwendet wurde, wird ein Ablesefehler von $\pm 0.05\,\unit{V}$ festgelegt.
Die Güte wird nach \autoref{eq:gute} berechnet. 
Die benötigten Werte $f_0$, $f_{-}$ und $f_{+}$ werden dabei aus \autoref{fig:guetekurve} entnommen. Da im Bereich um $\frac{1}{\sqrt(2)}U_{max}$ nicht genug Messwerte
aufgenommen wurden, wird der Bereich durch Hilfsgeraden linear dargestellt.

\begin{figure}[H]
  \centering
  \includegraphics[width=0.8\textwidth]{build/guetekurve.pdf}
  \caption{Messwerte zur Bestimmung der Filterkurve und Güte des Selektivverstärkers.}
  \label{fig:guetekurve}
\end{figure}

Ablesen ergibt:
\begin{align*}
  f_0 &= \SI{21850(1)}{}\\
  f_{-} &= \SI{21685(1)}{},\\
  \text{und }f_{+} &= 21850.
\end{align*}
Der Fehler beim Ablesen für $f_{-}$ und $f_{+}$ wird auf $\pm 1$ festgelegt.
Damit ergibt sich nach \autoref{eq:gute} eine Güte von
\begin{equation*}
  Q_{exp} = \SI{63.15(0.26)}{}.
\end{equation*}

\subsection{Bestimmung der Suszeptibilität}
\subsubsection{Effektiver Querschnitt der Proben}
Die untersuchten Proben bestehen in diesem Experiment aus staubförmigem Material. Da sich das Pulver nicht beliebig verdichten lässt, muss berücksichtigt werden, dass die Dichte geringer als bei einem Einkristall.
Berücksichtigt wird dies, indem der reale Querschnitt $Q_{real}$, den die Probe hätte, wenn sie aus einem Einkristall bestünde, in der Berechnung der Suszeptibilität verwendet wird.
Dieser berechnet sich aus der Masse $m$, der Länge $l$ und der Dichte $\rho$ der Probe mit
\begin{equation*}
  Q_{real} = \frac{m}{L\rho}.
\end{equation*}
Die Daten der Proben und der daraus resultierende Querschnitt $Q_{real}$ sind in \autoref{tab:proben} aufgelistet. Aufgrund der Waage wird ein Fehler von $\pm 0.01\,\unit{g}$ bei den Massen angenommen. 
Da die Länge $l$ der Proben mit einem Maßband, auf dem nur Messstriche in $\SI{0.1}{cm}$ Intervallen angegeben waren, abgemessen wurde, wird ein Fehler von $\pm \SI{0.1}{cm}$ abgeschätzt.
\begin{table}[H]
  \centering
  \caption{Daten der Proben und Querschnitt $Q_{real}$.}
  \begin{tabular}{c c c c c}
      \toprule
      {Material} & {$m /\symup{g}$} & {$l/\symup{cm}$} & {$\rho/\symup{\frac{g}{cm^3}}$} & {$Q/\symup{cm^2}$} \\
      \midrule
      $\symup{Dy_2O_3}$ & $\SI{14.38(0.01)}{}$ & $\SI{16.3(0.1)}{}$ & $7.80$ & $\SI{0.1131(0.0007)}{}$\\
      $\symup{Nd_2O_3}$ & $\SI{18.48(0.01)}{}$ & $\SI{14.5(0.1)}{}$ & $7.24$ & $\SI{0.1760(0.0012)}{}$\\
      $\symup{Gd_2O_3}$ & $\SI{14.08(0.01)}{}$ & $\SI{17.3(0.1)}{}$ & $7.40$ & $\SI{0.1100(0.0006)}{}$\\
      \bottomrule
  \end{tabular}
  \label{tab:proben}
\end{table}
\subsubsection{Grundlagen zur theoretischen Bestimmung der Suszeptibilität}
Die Suszeptibilität lässt sich aus dem Bahndrehimpuls $L$, der Spinquantenzahl $S$ und der Gesamtdrehimpulsquantenzahl $J$ ableiten. Für die Berechnung ist es wichtig, die 
Anzahl der 4f-Elektronen in der jeweiligen Atomhülle zu kennen. Das Madelung-Schema \cite{madelung} beschreibt wie die Orbitale mit Elektronen aufgefüllt werden. Im Normalfall gilt folgende
Besetzungsreihenfolge:
\begin{equation*}
  1s ⇒ 2s ⇒ 2p ⇒ 3s ⇒ 3p ⇒ 4s ⇒ 3d ⇒ 4p ⇒ 5s ⇒ 4d ⇒ 5p ⇒ 6s ⇒ 4f ⇒ 5d ⇒ 6p ⇒ ...
\end{equation*}
Da es sich bei den zu untersuchenden Proben um Lanthanoide handelt, besetzt zuerst ein Elektron ein Orbital der 5d-Unterschale, bevor 4f aufgefüllt wird. Die daraus
folgenden Grundelektronenkonfigurationen sind in \autoref{tab:grund} dargestellt.
\begin{table}[H]
  \centering
  \caption{Grundelektronenkonfigurationen der Stoffe Nd \cite{nd}, Gd \cite{gd} und Dy \cite{dy}.}
  \begin{tabular}{c c}
      \toprule
      {Material} & {Elektronenkonfiguration} \\
      \midrule
      $\symup{Dy}$ & $\symup{4f^7\,5d^1\,6s^2}$\\
      $\symup{Nd}$ & $\symup{4f^4\,6s^2}$\\
      $\symup{Gd}$ & $\symup{4f^{10}\,6s^2}$\\
      \bottomrule
  \end{tabular}
  \label{tab:grund}
\end{table}
Die Darstellung der Elektronenkonfiguration gibt immer das letzte voll besetzte Orbital an und in der Potenz, wieviele Elektronen sich im jeweiligen Orbital befinden.
Da in diesem Versuch aber Ion-Verbindungen untersucht werden, d.h. $\symup{Dy^{3+}}$, $\symup{Nd^{3+}}$ und $\symup{Gd^{3+}}$, müssen 3 Elektronen von der ursprünglichen
Konfiguration abgezogen werden. Dies geschieht nach der Befüllungsreihenfolge, außer wenn sich durch das abziehen der Elektronen von einem anderen Orbital ein günstiger
stabiler Zustand, d.h. ein halbvolles oder volles Orbital, einstellen kann.
Damit ergeben sich die Konfiguration in \autoref{tab:ion}.
\begin{table}[H]
  \centering
  \caption{Elektronenkonfigurationen der Ion-Verbindungen $\symup{Dy^{3+}}$, $\symup{Nd^{3+}}$ und $\symup{Gd^{3+}}$.}
  \begin{tabular}{c c}
      \toprule
      {Material} & {Elektronenkonfiguration} \\
      \midrule
      $\symup{Dy^{3+}}$ & $\symup{4f^7}$\\
      $\symup{Nd^{3+}}$ & $\symup{4f^3}$\\
      $\symup{Gd^{3+}}$ & $\symup{4f^9}$\\
      \bottomrule
  \end{tabular}
  \label{tab:ion}
\end{table}
\subsubsection{Berechnung des Gesamtdrehimpulses von Dysprosiumoxid}
Dysprosiumoxid hat neun Elektronen im 4f Orbital. Nach den Hundschen Regeln und dem Pauli-Prinzip folgt, dass sieben der neun Elektronen einen Spin von $\frac{1}{2}$ und zwei 
einen Spin von $-\frac{1}{2}$ annehmen. Der Gesamtspin ist somit
\begin{equation*}
  \textbf{S} = -\frac{1}{2} \cdot 2 + \frac{1}{2} \cdot 7 = 2.5.
\end{equation*}
Da Dysprosiumoxid ein f-Orbital hat sind die Zustände $l \in {-3, ..., 3}$ möglich. Die sieben Elektronen mit positiv orientiertem Spin schöpfen alle Zustände für $l$ bereits aus
und haben somit keinen Beitrag am Bahndrehimpuls. Nach der zweiten Hundschen Regel nehmen die Elektronen mit negativ orientiertem Spin dann die maximalen Zustände an.
Somit gilt für den Bahndrehimpuls
\begin{equation*}
  \textbf{L} = 3 + 2 = 5.
\end{equation*}
Weil das 4f-Orbital mehr als zur Hälfte gefüllt ist, gilt nach der dritten Hundschen Regel
\begin{equation*}
  \textbf{J} = \textbf{L} + \textbf{S} = 7.5.
\end{equation*}
\subsubsection{Berechnung des Gesamtdrehimpulses von Neodymoxid}
Neodymoxid hat drei Elektronen im 4f Orbital. Der maximale Gesamtspin beträgt somit
\begin{equation*}
  \textbf{S} = \frac{1}{2} \cdot 3 = 1.5.
\end{equation*}
Für den Bahndrehimpuls folgt aus der zweiten Hundschen Regel
\begin{equation*}
  \textbf{L} = 3 + 2 + 1 + 0 = 6.
\end{equation*}
Da das 4f-Orbital weniger als zur Hälfte gefüllt ist, folgt aus der dritten Hundschen Regel
\begin{equation*}
  \textbf{J} = \textbf{L} - \textbf{S} = 6 - 1.5 = 4.5.
\end{equation*}
\subsubsection{Berechnung des Gesamtdrehimpulses von Gadoliniumoxid}
Gadoliniumoxid hat sieben Elektronen im 4f Orbital. Der maximale Gesamtspin beträgt somit
\begin{equation*}
  \textbf{S} = \frac{1}{2} \cdot 7 = 3.5.
\end{equation*}
Für den Bahndrehimpuls folgt dann aus der zweiten Hundschen Regel
\begin{equation*}
  \textbf{L} = 0.
\end{equation*}
Für den Gesamtdrehimpuls folgt
\begin{equation*}
  \textbf{J} = \textbf{S} = 3.5.
\end{equation*}
\subsubsection{Berechnung der Suszeptibilität}\label{sec:sus}
Aus den berechneten Spin-, Bahndrehimpuls- und Gesamtdrehimpulszahlen lässt sich nun der Landé-Faktor nach \autoref{eq:lande} bestimmen. Die Zahl der Momente pro
Volumeneinheit $N$ berechnet sich mit
\begin{equation*}
  N = 2\cdot\frac{N_A\rho}{m_{mol}}.
\end{equation*}
Der Vorfaktor kommt durch das doppelte Vorkommen der jeweiligen Lanthanoid-Atome in den Lanthanoidoxiden zustande. Die Raumtemperatur $T$ wird auf $\SI{293.15(1)}{K}$ geschätzt.
Mit $\mu_0 = \SI{1.38e-23}{}$, $k_B = \SI{1.26e-6}{}$ und $\mu_B = \SI{9.27e-24}{}$ ergeben sich die Suszeptibilitäten in \autoref{tab:theo}.
\begin{table}[H]
  \centering
  \caption{Landé-Faktor $g_j$, molare Masse $m_{mol}$, $N$ und Suszeptibilität $\chi$.}
  \begin{tabular}{c c c c c}
      \toprule
      {Ion-Verbindung} & {$m_{mol}/\symup{\frac{g}{mol}}$} & {$N/\symup{\frac{10^28}{m^3}}$} & {$g_j$} & {$\chi$} \\
      \midrule
      $\symup{Dy_2O_3}$ & $\SI{373.00}{}$ & $\SI{2.5186}{}$ & $1.333$ & $\SI{0.02541(0.00008)}{}$\\
      $\symup{Nd_2O_3}$ & $\SI{336.48}{}$ & $\SI{2.5916}{}$ & $0.727$ & $\SI{0.00301(0.00001)}{}$\\
      $\symup{Gd_2O_3}$ & $\SI{362.50}{}$ & $\SI{2.4586}{}$ & $2.000$ & $\SI{0.01378(0.00005)}{}$\\
      \bottomrule
  \end{tabular}
  \label{tab:theo}
\end{table}
Die molaren Massen von Dysprosiumoxid \cite{mol_dy}, Neodymoxid \cite{mol_nd} und Gadoliniumoxid \cite{mol_gd} wurden dabei aus entsprechender Quelle entnommen.
\subsubsection{Bestimmung der Suszeptibilität mit der Brückenschaltung}
\begin{table}[H]
  \centering
  \caption{Messwerte von $\symup{Dy_20_3}$ und die Differenz $\Delta R$.}
  \begin{tabular}{c c c}
      \toprule
      {$R_0/\symup{\Omega}$} & {$R_{Probe}/\symup{\Omega}$} & {$\Delta_{R}/\symup{\Omega}$} \\
      \midrule
      $\SI{2.60(0.05)}{}$ & $\SI{0.98(0.05)}{}$ & $\SI{1.63(0.07)}{}$ \\
      $\SI{2.42(0.05)}{}$ & $\SI{0.91(0.05)}{}$ & $\SI{1.51(0.07)}{}$ \\
      $\SI{2.53(0.05)}{}$ & $\SI{0.98(0.05)}{}$ & $\SI{1.56(0.07)}{}$ \\
      \bottomrule
  \end{tabular}
  \label{tab:R_dy}
\end{table}
\begin{table}[H]
  \centering
  \caption{Messwerte von $\symup{Nd_20_3}$ und die Differenz $\Delta R$.}
  \begin{tabular}{c c c}
      \toprule
      {$R_0/\symup{\Omega}$} & {$R_{Probe}/\symup{\Omega}$} & {$\Delta_{R}/\symup{\Omega}$} \\
      \midrule
      $\SI{2.60(0.05)}{}$ & $\SI{2.32(0.05)}{}$ & $\SI{0.28(0.07)}{}$ \\
      $\SI{2.52(0.05)}{}$ & $\SI{2.35(0.05)}{}$ & $\SI{0.16(0.07)}{}$ \\
      $\SI{2.43(0.05)}{}$ & $\SI{2.38(0.05)}{}$ & $\SI{0.05(0.07)}{}$ \\
      \bottomrule
  \end{tabular}
  \label{tab:R_nd}
\end{table}
\begin{table}[H]
  \centering
  \caption{Messwerte von $\symup{Gd_20_3}$ und die Differenz $\Delta R$.}
  \begin{tabular}{c c c}
      \toprule
      {$R_0/\symup{\Omega}$} & {$R_{Probe}/\symup{\Omega}$} & {$\Delta_{R}/\symup{\Omega}$} \\
      \midrule
      $\SI{2.51(0.05)}{}$ & $\SI{1.74(0.05)}{}$ & $\SI{0.77(0.07)}{}$ \\
      $\SI{2.42(0.05)}{}$ & $\SI{1.75(0.05)}{}$ & $\SI{0.67(0.07)}{}$ \\
      $\SI{2.54(0.05)}{}$ & $\SI{1.73(0.05)}{}$ & $\SI{0.81(0.07)}{}$ \\
      \bottomrule
  \end{tabular}
  \label{tab:R_gd}
\end{table}
Aus den Messwerten wird jeweils die Differenz $\Delta_R$ gebildet und anschließend gemittelt. Da die Spannung nicht immer genau über den Widerstand minimiert werden konnte,
wird ein Fehler von $\pm \SI{0.05}{\Omega}$ angesetzt.
Mit dem Widerstand $R_3 = \SI{998}{\Omega}$, den Widerstandsdifferenzen $\Delta_R$, der Spulenquerschnittsfläche $F = \SI{0.886}{cm^2}$ und $Q_{real}$ folgen mit 
\autoref{eqn:chi2} die Suszeptibilitäten in \autoref{tab:sus}.
\begin{table}[H]
  \centering
  \caption{Gemittelte Widerstandsdifferenz $\overline{\Delta_R}$, Querschnitt $Q_{real}$ und Suszeptibilität $\chi$.}
  \begin{tabular}{c c c c}
      \toprule
      {Ion-Verbindung} & {$\overline{\Delta_R}$} & {$Q_{real}/\symup{cm^2}$} & {$\chi$} \\
      \midrule
      $\symup{Dy_2O_3}$ & $\SI{1.56(0.04)}{}$ & $\SI{0.1131(0.0007)}{}$ & $\SI{0.0240(0.0006)}{}$\\
      $\symup{Nd_2O_3}$ & $\SI{0.17(0.04)}{}$ & $\SI{0.1760(0.0012)}{}$ & $\SI{0.0016(0.0004)}{}$\\
      $\symup{Gd_2O_3}$ & $\SI{0.75(0.04)}{}$ & $\SI{0.1100(0.0006)}{}$ & $\SI{0.0119(0.0006)}{}$\\
      \bottomrule
  \end{tabular}
  \label{tab:sus}
\end{table}
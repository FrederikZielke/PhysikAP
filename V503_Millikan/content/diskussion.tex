\section{Diskussion}
\label{sec:Diskussion}
Im Folgenden werden die prozentualen Abweichungen mit 
\begin{equation}\label{eq:1}
    \Delta = |\frac{exp - theo}{theo}|\cdot 100\%
\end{equation}
berechnet.

\subsection{Elementarladung}
In \autoref{tab:Diskussion} ist die berechnete, der Literaturwert der Elementarladung \cite{Elementarladung} und die Abweichung zu sehen.
\begin{table}[H]
    \centering
    \caption{Experimentell bestimmte Elementarladung $e_{exp}$, Theoriewert $e_{theo}$ und Abweichung in \%.}
    \begin{tabular}{c c c}
        \toprule
        {$e_{exp}\,/\symup{10^{-19}\,C}$} & {$e_{theo}\,/\symup{10^{-19}\,C}$} & {$\Delta\,/\symup{\%}$}\\
        \midrule
        $\SI{1.5592(0.6624)}{}$ & 1.6022 & 2.68 \\
        \bottomrule
    \end{tabular}
    \label{tab:Diskussion}
\end{table}
Die Abweichung ist mit 2.68\% relativ gering. Jedoch unterliegt der bestimmte Wert für die Elementarladung einem großen Fehler, sodass die Abweichung nicht aussagekräftig ist.
Der große Fehler ist größtenteils auf den großen Fehler bei der Bestimmung des Mittelwertes zurückzuführen. Die Ungenauigkeit der Strecke $s$ ist ebenfalls ein großer Fehlerfaktor,
da die Skala nur in 0.1\,mm Schritten abgelesen werden kann. Der Fehler beim Mittelwert entsteht durch die breite Streuung der gemessenen Steige- und Sinkzeiten. Um diesen
zu minimieren, müssten mehr Messwerte aufgenommen werden, da so große Abweichungen rausgemittelt werden. Die Streuung der Steige- und Sinkzeiten hat zum Beispiel den Grund,
dass das beobachtete Öltröpfchen mit anderen Tröpfchen kollidiert und so seine Geschwindigkeit verändert. Dieser Effekt ist nicht zu vermeiden, da die Tröpfchen nicht einzeln 
erzeugt werden können.
Die Ungenauigkeit der Strecke $s$ könnte durch eine feinere Skala verringert werden.

\subsection{Avogadro-Konstante}
In \autoref{tab:Diskussion2} ist die berechnete, der Literaturwert der Avogadro-Konstante \cite{Avogadro} und die Abweichung zu sehen.
\begin{table}[H]
    \centering
    \caption{Experimentell bestimmte Elementarladung $e_{exp}$, Theoriewert $e_{theo}$ und Abweichung in \%.}
    \begin{tabular}{c c c}
        \toprule
        {$N_{exp}\,/\symup{10^{23}\,C}$} & {$N_{theo}\,/\symup{10^{23}\,C}$} & {$\Delta\,/\symup{\%}$}\\
        \midrule
        $\SI{6.1883(2.6023)}{}$ & 6.0221 & 2.76 \\
        \bottomrule
    \end{tabular}
    \label{tab:Diskussion2}
\end{table}
Die Abweichung ist mit 2.76\% relativ gering. Jedoch unterliegt der bestimmte Wert für die Avogadro-Konstante auch einem großen Fehler, sodass die Abweichung nicht aussagekräftig ist.
Das ist auf die gleichen Fehlerfaktoren wie bei der Elementarladung zurückzuführen, da die Avogadro-Konstante aus der Elementarladung berechnet wird.

\newpage
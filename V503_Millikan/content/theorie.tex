\section{Zielsetzung}
\label{sec:Zielsetzung}

Mit dem Millikan-Versuch soll die Elementarladung $e_0$ bestimmt werden.
Dies wird durch die Messung der Gleichgewichts-, Sink- und Steiggeschwindigkeit von Öltröpfchen in einem elektrischen Feld erreicht.
Außerdem soll die Avogadro-Konstante $N_A$ bestimmt werden.

\section{Theorie}
\label{sec:Theorie}

Zur Versuchsdurchführung werden Tröpfchen durch Zerstäuben von Öl in eine Kammer gebracht.
Durch die Reibung bei der Zerstäubung werden die Tröpfchen elektrisch geladen.
Die Ladung der Tröpfchen beträgt immer ein ganzzahliges Vielfaches der Elementarladung $e_0$.

\subsection{Wirkende Kräfte}%Vielleicht lösche ich noch alle Überschriften
\label{sec:Wirkende Kräfte}

Auf die Tröpfchen wirken bevor eine Spannung an den Kondensator angelegt wird die Gravitationskraft $\vec{F_g} = m \vec{g}$ und die Stokesche Reibungskraft $\vec{F_R} = -6\pi r \eta_{\text{Lu}} \vec{v}$.
Dabei ist $m$ die Masse des Tröpfchens, $g$ die Erdbeschleunigung, $r$ der Radius des Tröpfchens, $\eta_{\text{Lu}}$ die Viskosität der Luft und $\vec{v}$ die Geschwindigkeit des Tröpfchens.
Die Stokesche Reibungskraft wirkt der Bewegungsrichtung entgegen.
Nach kurzer zeit stellt sich eine Gleichgewichtsgeschwindigkeit $v_0$ ein, bei der die beiden Kräfte gleich groß sind.
Das Kräftegleichgewicht lautet
\begin{equation*}
    \frac{4 \pi}{3}r^3 \left(\rho_{\text{Öl}} - \rho_{\text{Lu}}\right) g = 6 \pi r \eta_{\text{Lu}} v_0 \, .
\end{equation*}
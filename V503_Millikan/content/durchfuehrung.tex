\section{Durchführung}
\label{sec:Durchführung}

\subsection{Versuchsaufbau}
\label{sec:Versuchsaufbau}


Der Millikan-Versuch wird mit einem Millikan-Öltröpfchen-Apparat durchgeführt.
Dieser besteht aus einem Plattenkondensator, einer Zerstäubungskammer, einer Beleuchtungseinrichtung und einem Mikroskop.
An den Plattenkondensator wird eine Spannungsquelle angeschlossen.
Die Polung der angelegten Spannung kann mit einem Schalter eingestellt werden.
Die tatsächlich anliegende Spannung wird mit einem Spannungsmessgerät gemessen.
Die Temperatur in der Zerstäubungskammer wird mit einem Thermowiderstand gemessen,
welcher mit einem Widerstandsmeßgerät verbunden ist.
Die Zerstäubungskammer ist mit einem Mikroskop verbunden, mit dem die Tröpfchen beobachtet werden können.
Am Mikroskop kann seperat auf die Messskala und auf die Tröpfchen scharf gestellt werden.
Mithilfe einer Libelle kann der gesamte Aufbau wagerecht ausgerichtet werden.
Mit einem Zerstäuber werden die Öltröpfchen erzeugt.\\
\\
Vor Beginn der Messung wird der Fokus des Mikroskops auf die Skala und die Tröpfchenebene scharf gestellt.
Die Spannungsquelle wird eingeschaltet und die Spannung auf $\SI{250}{\volt}$ eingestellt.



\subsection{Messung}
\label{sec:Messung}


Es werden 20 Öltröpfchen vermessen.\\
Zunächst werden mithilfe des Zerstäubers Öltröpfchen in die Zerstäubungskammer gegeben.
Es wird ein Tröpfchen ausgewählt, welches sich möglichst langsam bewegt.
Dann wird die Zeit aufgenommen, die das Tröpfchen für eine Strecke von einer großen Teilung der Skala $s = \SI{0.5}{mm}$ benötigt.
Diese Zeit wird zur Berechnung von $v_0$ verwendet. 
Danach wird der Kondensator so eingestellt, dass sich das Tröpfchen nach oben bewegt.
Es wird wieder die Zeit aufgenommen die es benötigt, um $s = \SI{0.5}{mm}$ zurückzulegen.
Diese Zeit wird zur Berechnung von $v_\text{auf}$ verwendet.
Als letztes wird die Polung der Spannung umgeschaltet, sodass sich das Tröpfchen nach unten bewegt.
Es wird wieder die Zeit aufgenommen die das Tröpfchen benötigt um die gleiche Strecke wie zuvor zurückzulegen.
Diese Zeit wird zur Berechnung von $v_\text{ab}$ verwendet.
Die beiden letzten Messungen werden für jedes Tröpfchen noch 2 mal wiederholt.
\\
Nachdem die Messung für ein Tröpfchen abgeschlossen ist, wird die Spannung auf $\SI{300}{\volt}$ erhöht und die Messung für zehn weitere Tröpfchen wiederholt.
\\
Zu jedem Tröpfchen wird der Thermowiderstand abgelesen und notiert.\\

\newpage
\section{Diskussion}
\label{sec:Diskussion}

Im folgenden werden die prozentualen Abweichungen mit 
\begin{equation}\label{eq:1}
    \Delta = |\frac{exp - theo}{theo} \cdot 100|
\end{equation}
berechnet.
\\
Die in \autoref{sec:auswertung_grenzspannung} ermittelten Grenzspannungen liegen in einer unerwarteten Reihenfolge.
Es wurde ein proportionaler Zusammenhang zur Wellenlänge des verwendeten Lichts erwartet.
Aus den Ausgleichsgeraden geht jedoch kein erkennbarer Zusammenhang hervor. 
Mögliche Ursache ist eine fehlerhafte Messung des Photostroms. 
Die Spannungsquelle für die Gegen- und Beschleunigungsspannung ist hoch empfindlich und reagiert auf kleine Erschütterungen.
Dadurch kann es zu Schwankungen der Spannung kommen, die sich auf die Messung auswirken.
Der Photostrom wurde von einem Analogen Messgerät abgelesen, wodurch geringe Unterschiede in der Ablesung schwer zu erkennen sind.
Der Photostrom bei rotem Licht war so gering, dass selbst auf kleinster Messeinstellung des Amperemeters kaum ein Unterschied zwischen 
verschiedenen Spannungen zu erkennen war.\\
Durch den größeren Spannunungsbereich bei gelbem Licht, wurden auch größere Abstände zwischen den einzelnen Messpunkten gewählt.
Gerade im kritischen Bereich um die Grenzspannung, wurde dadurch die Messung ungenauer. 
Bei einer erneuten Messung sollten besonders in diesem Bereich mehr Werte aufgenommen werden.\\
\\
\\
Bei der Bestimmung von $h$ ist eine Abweichung von $44.3\%$ zur Literatur aufgetreten\cite{planck}.
%Legt man die Ausgleichsgerade nur durch die Werte von gelbem und violettem Licht, so ergibt sich eine Abweichung von $67\%$.
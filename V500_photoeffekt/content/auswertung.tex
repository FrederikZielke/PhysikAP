\section{Auswertung}
\label{sec:Auswertung}

Die in \autoref{sec:Auswertung} gezeigten Grafiken und Rechnungen sind mithilfe der Python-Bibliotheken Matplotlib \cite{matplotlib}, Scipy \cite{scipy} und Numpy \cite{numpy}
erstellt worden.

\subsection{Bestimmung der Grenzspannung}
\label{sec:auswertung_grenzspannung}

Um die Grenzspannung $U_G$ zu ermitteln wird die Spannung $U$ gegen die Quadratwurzel des Photostroms $I$ aufgetragen. 
Durch die linearen Abschnitte wird eine Ausgleichsgerade gezogen.
Der Schnittpunkt der Geraden mit der x-Achse entspricht der Grenzspannung $U_G$.
Mit der Ausgleichsgerade $y \,=\, m \cdot x \,+\, b$, lässt sich die Grenzspannung berechnen mit
\begin{equation}\label{eq:ug}
  U_G = - \frac{b}{a}.
\end{equation}
\\
\\
\begin{table}[H]
  \centering
  \caption{Brems- und Beschleunigungsspannung $U$ und Photostrom $I$ bei violettem Licht.}
  \begin{tabular}{|c|c|}
    \toprule
    $U \,/\, [\si{\volt}]$ & $I \,/\, [\si{\nano\ampere}]$\\
    \midrule
    -1.99 & 0\\
    -1.90 & 0\\
    -1.70 & 0\\
    -1.50 & 0\\
    -1.30 & 0\\
    -1.10 & 0\\
    -1.05 & 0\\
    -1.00 & 0.009\\
    -0.95 & 0.026\\
    -0.90 & 0.046\\
    -0.85 & 0.070\\
    -0.80 & 0.098\\
    -0.75 & 0.130\\
    -0.70 & 0.160\\
    -0.65 & 0.210\\
    -0.60 & 0.250\\
    -0.55 & 0.320\\
    -0.50 & 0.370\\
    -0.45 & 0.420\\
    -0.40 & 0.480\\
    -0.35 & 0.540\\
    -0.30 & 0.580\\
    \bottomrule
  \end{tabular}
  \begin{tabular}{|c|c|}
    \toprule
    $U \,/\, [\si{\volt}]$ & $I \,/\, [\si{\nano\ampere}]$\\
    \midrule
    -0.25 & 0.620\\
    -0.20 & 0.670\\
    -0.15 & 0.720\\
    -0.10 & 0.760\\
    -0.05 & 0.810\\
    -0.01 & 0.860\\
    0.05 & 0.900\\
    0.10 & 0.940\\
    0.15 & 0.980\\
    0.20 & 1.000\\
    0.25 & 1.000\\
    0.30 & 1.100\\
    0.40 & 1.200\\
    0.50 & 1.200\\
    0.70 & 1.400\\
    0.90 & 1.500\\
    1.10 & 1.600\\
    1.30 & 1.800\\
    1.50 & 1.900\\
    1.70 & 2.100\\
    1.90 & 2.300\\
    1.99 & 2.400\\
    \bottomrule
  \end{tabular}
  \label{tab:lila}
\end{table}

\begin{figure}
  \centering
  \includegraphics{build/plotLila.pdf}
  \caption{Plot der Messwerte aus \autoref{tab:lila} und Ausgleichsgerade.}
  \label{fig:plot_lila}
\end{figure}

\begin{table}[H]
  \centering
  \caption{Brems- und Beschleunigungsspannung $U$ und Photostrom $I$ bei grünem Licht.}
  \begin{tabular}{|c|c|}
    \toprule
    $U \,/\, [\si{\volt}]$ & $I \,/\, [\si{\nano\ampere}]$\\
    \midrule
    -1.99 & 0\\
    -1.90 & 0\\
    -1.80 & 0\\
    -1.70 & 0\\
    -1.60 & 0\\
    -1.50 & 0\\
    -1.20 & 0\\
    -1.00 & 0\\
    -0.80 & 0\\
    -0.60 & 0\\
    -0.55 & 0.010\\
    -0.50 & 0.028\\
    -0.45 & 0.055\\
    -0.40 & 0.094\\
    -0.35 & 0.140\\
    -0.30 & 0.200\\
    -0.25 & 0.260\\
    -0.20 & 0.300\\
    -0.15 & 0.350\\
    \bottomrule
  \end{tabular}
  \begin{tabular}{|c|c|}
    \toprule
    $U \,/\, [\si{\volt}]$ & $I \,/\, [\si{\nano\ampere}]$\\
    \midrule
    -0.10 & 0.400\\
    -0.05 & 0.440\\
    -0.01 & 0.470\\
    0.05 & 0.520\\
    0.10 & 0.540\\
    0.15 & 0.580\\
    0.20 & 0.600\\
    0.30 & 0.680\\
    0.40 & 0.740\\
    0.50 & 0.790\\
    0.60 & 0.840\\
    0.70 & 0.900\\
    0.80 & 0.960\\
    0.90 & 1.000\\
    1.20 & 1.150\\
    1.40 & 1.200\\
    1.60 & 1.200\\
    1.80 & 1.400\\
    1.99 & 1.600\\
    \bottomrule
  \end{tabular}
  \label{tab:gruen}
\end{table}

\begin{figure}[H]
  \centering
  \includegraphics{build/plotGruen.pdf}
  \caption{Plot der Messwerte aus \autoref{tab:gruen} und Ausgleichsgerade.}
  \label{fig:plot_gruen}
\end{figure}

\begin{table}[H]
  \centering
  \caption{Brems- und Beschleunigungsspannung $U$ und Photostrom $I$ bei gelbem Licht.}
  \begin{tabular}{|c|c|}
    \toprule
    $U \,/\, [\si{\volt}]$ & $I \,/\, [\si{\nano\ampere}]$\\
    \midrule
    -19.00 & 0\\
    -5.00 & 0\\
    -1.00 & 0\\
    -0.50 & 0\\
    -0.01 & 0\\
    0.01 & 0.22\\
    0.50 & 0.40\\
    1.00 & 0.55\\
    1.50 & 0.68\\
    2.00 & 0.82\\
    2.50 & 0.97\\
    3.00 & 1.00\\
    3.50 & 1.20\\
    4.00 & 1.20\\
    5.00 & 1.40\\
    \bottomrule
  \end{tabular}
  \begin{tabular}{|c|c|}
    \toprule
    $U \,/\, [\si{\volt}]$ & $I \,/\, [\si{\nano\ampere}]$\\
    \midrule
    6.00 & 1.40\\
    7.00 & 1.55\\
    8.00 & 1.60\\
    9.00 & 1.70\\
    10.0 & 1.80\\
    11.0 & 1.80\\
    12.00 & 1.85\\
    13.00 & 1.95\\
    14.00 & 2.00\\
    15.00 & 2.00\\
    16.00 & 2.00\\
    17.00 & 2.10\\
    18.00 & 2.20\\
    19.00 & 2.20\\
    \text{---} & \text{---}\\
    \bottomrule 
  \end{tabular}
  \label{tab:gelb}
\end{table}

\begin{figure}[H]
  \centering
  \includegraphics{build/plotGelb.pdf}
  \caption{Plot der Messwerte aus \autoref{tab:gelb} und Ausgleichsgerade.}
  \label{fig:plot_gelb}
\end{figure}

\begin{table}[H]
  \centering
  \caption{Brems- und Beschleunigungsspannung $U$ und Photostrom $I$ bei rotem Licht.}
  \begin{tabular}{|c|c|}
    \toprule
    $U \,/\, [\si{\volt}]$ & $I \,/\, [\si{\nano\ampere}]$\\
    \midrule
    -1.99 & 0\\
    -1.00 & 0\\
    -0.70 & 0\\
    -0.65 & 0\\
    -0.60 & 0\\
    -0.55 & 0.001\\
    -0.50 & 0.002\\
    -0.45 & 0.002\\
    -0.40 & 0.003\\
    -0.35 & 0.004\\
    -0.30 & 0.004\\
    -0.25 & 0.006\\
    -0.20 & 0.008\\
    -0.10 & 0.011\\
    -0.01 & 0.014\\
    \bottomrule
  \end{tabular}
  \begin{tabular}{|c|c|}
    \toprule
    $U \,/\, [\si{\volt}]$ & $I \,/\, [\si{\nano\ampere}]$\\
    \midrule
    0.10 & 0.018\\
    0.20 & 0.022\\
    0.30 & 0.026\\
    0.40 & 0.029\\
    0.50 & 0.032\\
    0.60 & 0.035\\
    0.70 & 0.038\\
    0.80 & 0.042\\
    1.00 & 0.048\\
    1.20 & 0.052\\
    1.40 & 0.057\\
    1.60 & 0.062\\
    1.80 & 0.066\\
    1.99 & 0.072\\
    \text{---} & \text{---}\\
    \bottomrule
  \end{tabular}
  \label{tab:rot}
\end{table}

\begin{figure}[H]
  \centering
  \includegraphics{build/plotRot.pdf}
  \caption{Plot der Messwerte aus \autoref{tab:rot} und Ausgleichsgerade.}
  \label{fig:plot_rot}
\end{figure}

Mit \autoref{eq:ug} lassen sich die Grenzspannungen berechnen. Die Ergebnisse sind in \autoref{tab:ug} aufgelistet.
\begin{table}
  \centering
  \begin{tabular}{|c|c|c|c|c|}
    \toprule
    Lichtfarbe & $λ$[nm] & $m[\frac{\sqrt{\text{nA}}}{\text{V}}]$ & $b[\sqrt{\text{nA}}]$ & $U_G$[V]\\
    \midrule
    Violett & $405$ & $1.021\pm 0.023$ & $1.117\pm 0.017$ & $-1.094$\\
    Grün & $546$ & $1.45\pm 0.04$ & $0.885\pm 0.017$ & $-0.610$\\
    Gelb & $578$ & $0.334\pm 0.009$ & $0.465\pm 0.007$ & $-1.396$\\
    Rot & $623$ & $0.178\pm 0.012$ & $0.124\pm 0.005$ & $-0.695$\\
    \bottomrule
  \end{tabular}
  \caption{Grenzspannungen $U_G$, Fitparameter $m$ und $b$ für die verschiedenen Lichtfarben.}
  \label{tab:ug}
\end{table}

\newpage

\subsection{Bestimmung des Verhältnisses $\frac{h}{e_0}$}

\begin{figure}
  \centering
  \includegraphics{build/frequenz.pdf}
  \caption{Plot der Grenzspannungen $U_G$ gegen die Frequenz $f$ des Lichts.}
  \label{fig:frequenz}
\end{figure}

Zur Bestimmung des Verhältnisses $\frac{h}{e_0}$ wird die in \autoref{sec:auswertung_grenzspannung} ermittelte Grenzspannung $U_G$
gegen die Frequenz $f$ des Lichts aufgetragen. 
Der Zusammenhang von Frequenz und Grenzspannung wird mit \autoref{eq:Elektron_max} beschrieben.
Die Frequenzen werden entsprechend den Wellenlängen der in \autoref{tab:ug} aufgelisteten
Lichtfarben und Grenzspannungen gewählt. Die Steigung der Ausgleichsgeraden ist gleich dem Verhältnis $\frac{h}{e_0}$.
Die negative Austrittsarbeit $A_k$ des Kathodenmaterial geteilt durch $e_0$ ist gleich dem Ordinatenabschnitt.
\begin{align}
  h &= (-0.8\pm 2.2) \cdot 10^{-15} \si{eVs} \\
  A_k &= (0.5\pm 1.3) \si{\eV}
\end{align}
Wählt man die augenscheinlich physikalisch sinnvollen Werte für die Ausgleichsgerade ergibt sich die schwarze Ausgleichsgerade in \autoref{fig:frequenz}.
Für diese Gerade ergeben sich die Werte:
%h/e: (1.3625572631204357+/-0)e-15, A_k: -2.1031509017147485+/-0
\begin{align}
  h &= (1.353) \cdot 10^{-15} \si{eVs} \\
  A_k &= (-2.1) \si{\eV}
\end{align}
%Hier noch A_k und h/e_0 einfügen
%h/e: (-2.531483029724948+/-0)e-15, A_k: 0.7793278907374918+/-0
%h/e real: (-0.8+/-2.2)e-15, A_k real: -0.5+/-1.3


\subsection{Kurvenverlauf bei 578nm}

\begin{figure}[H]
  \centering
  \includegraphics{build/plotGelb2.pdf}
  \caption{Beschleunigungs- bzw. Bremsspannung gegen Photostrom aufgetragen mit Daten aus \autoref{tab:gelb}.}
  \label{fig:gelb2}
\end{figure}

Ab einer Beschleunigungsspannung von $\SI{10}{\volt}$ flacht die Steigerung des Photostroms stark ab.
Dies ist in \autoref{fig:gelb2} zu erkennen. Die Kurve nähert sich einer Sättigung an.
Der Photostrom ist begrenzt durch die Anzahl der Elektronen, die von Photonen ausgelöst werden.
Die Anzahl der Photonen hängt wiederum von der Intensität der verwendeten Lampe ab.
Gehen alle freiwerdenden Elektronen in die Anode über, so ist der Photostrom gesättigt.
Eine weitere Erhöhung der Beschleunigungsspannung vergrößert den Photostrom nicht mehr.
\\
Das Ohmsche Gesetz $I = \frac{U}{R}$ gilt nur für ohmsche Widerstände und ist deshalb nicht verletzt durch das Verhalten des Photostroms.
\\
Die Spannung bei der die Sättigung auftritt, ist abhänig von der Intensität des Lichts und der Austrittsarbeiten $A_k$ des Kathodenmaterials
und $A_A$ des Anodenmaterials.\\
%Warum wird die Sättigung asymptotisch erreicht?
Da bei der Freisetzung der Elektronen Streuung auftritt, wird die Sättigung nur asymptotisch erreicht.
%Wie muss eine Photozelle konstruert werden um bei endlicher Spannung Sättigung zu erreichen?
Wäre die Kathode vollständig von der Anode umschlossen, würden keine Photoelektronen verloren gehen.
%Warum fällt der Photostrom schon vor U_G merklich ab?
Photoelektronen besitzen eine Energieverteilung, die zwischen $0$ und ihrer maximalen kinetischen Energie liegt.
Ihre Energie ist abhänig davon, welche Energie sie im Festkörper besessen haben.
Da niederenergetische Elektronen nicht zwangsläufig die Strecke zwischen Kathode und Anode überwinden können,
sinkt der Photostrom schon vor $U_G$ merklich ab.\\
%Warum kann ein dem Photostrom entgegengerichteter Strom auftreten?
Besitzt das Anodenmaterial eine höhere Austrittsarbeit als die Kathode,
muss eine Beschleunigungsspannung angelegt werden, damit Elektronen die Anode erreichen können.
Wird die Bremsspannung groß genug gewählt, kann es zu einem negativen Strom kommen.\\
Ist ein negativer Strom schon bei energiearmem Licht zu beobachten, lässt dies auf einne niedrige Austrittsarbeit der Anode schließen.\\
%warum kann hier bei relativ kleinen Spannungen Sättigung aftreten? Keine Ahnung, ist mir auch egal
%Was kann man über die Austrittsarbeit der Anode sagen, wenn mana berücksichtigt, dass der negative Strom bereits bei Einstrahlung energiearmen Lichtes auftritt?
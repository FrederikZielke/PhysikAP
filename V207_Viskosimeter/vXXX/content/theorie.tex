\section{Theorie}
\label{sec:Theorie}

\subsection{Kräfte}
Auf Körper die sich durch eine Flüssigkeit bewegen wirkt eine Reibungskraft $\vec{F}$,
die der Bewegungsrichtung entegengesetzt ist. Die Reibungskraft ist Abhängig von der Berührungsfläche A,
der Geschwindigkeit $v$ und der dynamischen Viskosität $η$.
Die Stokessche Reibung wird durch 
\\
\begin{equation}
    F_{R} = 6πηvr
\end{equation}
\\
angegeben. $\vec{F}_R$ steigt mit zunehmender Geschwindigkeit, bis sich ein Gleichgewicht zwischen $\vec{F}_R$, der Schwerkraft $\vec{F}_g$ 
und dem Auftrieb $\vec{F}_A$ einstellt. Auftriebs- und Reibungskraft wirken der Schwerkraft entgegen.
\\
\subsection{Viskosität}
Die Viskosität einer Flüßigkeit lässt sich mit der Formel
\begin{equation}\label{eqn:2}
    η = K \cdot (ρ_k - ρ_{Fl}) \cdot t
\end{equation}



%Für die Auswertung nach K umstellen:
%\begin{equation} \label{eqn:K}
%    \Leftrightarrow K = \frac{η}{(ρ_K - ρ_{Fl}) \cdot t}
%\end{equation}
%\\
oder mit der Andradschen Gleichung
\begin{equation}
η(T) = A \cdot \symup{e}^{\left(\frac{B}{T}\right)}
\label{eqn:AndradscheGl}
\end{equation}
berechnen.

Die Dichte einer Kugel lässt sich mit 
\begin{equation}\label{eqn:4}
    ρ = \frac{m}{V} = \frac{3 \cdot m}{4\cdot π \cdot r^3}
\end{equation}
ausrechnen.
\\
%\begin{equation}
%η(T) = η_o \cdot exp \left(\frac{E_A}{R\cdot T}\right)
%\end{equation}
\subsection{laminare Strömung}
Die Bewegung von Flüssigkeiten mit unterschiedlichen Strömungsgeschwindigkeiten, ohne dass sich Wirbel ausbilden, wird als laminare Strömung bezeichnet.\\

Die Reynolds-Zahl ermöglicht abzuschätzen ob die Strömung um ein Objekt Laminar ist. Allgemein kann sie mit
\begin{equation}\label{eqn:5}
    Re = \frac{v \cdot l \cdot ρ}{η} \;\text{\cite{reynold}}
\end{equation}
berechnbar. Die Reynolds-Zahl hat einen kritischen Wert ab dem die laminare Strömung instabil wird.
%ich glaube es macht keinen sinn hier die kritische Stroemung in einem Rohr aufzuschreiben

\subsection{Fehlerrechnung}

Der Mittelwert kann mit 
\begin{equation}
    \bar{x}_{arithm} = \frac{1}{n}  \sum_{i=1}^n x_i = \frac{x_1 + x_2 + \cdots + x_n}{n}
    \label{eqn:6}
\end{equation}
berechnet werden.

Die Standardabweichung des Mittelwerts lässt sich mit
\begin{equation}
    \sigma = \sqrt{\sum_{i=1}^n (x_i - \bar{x}_{arithm})^2}
    \label{eqn:7}
\end{equation}
berechnen und die Empirische Standardabweichung mit
\begin{equation}
    s = \sqrt{\frac{1}{n - 1} \sum_{i=1}^n (x_i - \bar{x}_{arithm})^2} .
    \label{eqn:8}
\end{equation}
%\cite{sample}

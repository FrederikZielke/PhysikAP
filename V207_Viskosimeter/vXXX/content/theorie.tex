\section{Theorie}
\label{sec:Theorie}

Auf Körper die sich durch eine Flüssigkeit bewegen wirkt eine Reibungskraft $\vec{F}$,
die der Bewegungsrichtung entegengesetzt ist. Die Reibungskraft ist Abhängig von der Berührungsfläche A,
der Geschwindigkeit $v$ und der dynamischen Viskosität $η$.
Die Stokessche Reibung wird durch 
\\
\begin{equation}
    F_{R} = 6πηvr
\end{equation}
\\
angegeben. $\vec{F}_R$ steigt mit zunehmender Geschwindigkeit, bis sich ein Gleichgewicht zwischen $\vec{F}_R$, der Schwerkraft $\vec{F}_g$ 
und dem Auftrieb $\vec{F}_A$ einstellt. Auftriebs- und Reibungskraft wirken der Schwerkraft entgegen.
\\
% K gleichung mit eta
\begin{equation}
η = K \cdot (ρ_k - ρ_{Fl}) \cdot t
\end{equation}
\\
\begin{equation}
η(T) = A \cdot exp \left(\frac{B}{T}\right)
\label{eqn:AndradscheGl}
\end{equation}
\\
\begin{equation}
η(T) = η_o \cdot exp \left(\frac{E_A}{R\cdot T}\right)
\end{equation}
%\cite{sample}

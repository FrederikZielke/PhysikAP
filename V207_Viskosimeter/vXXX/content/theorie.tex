\section{Theorie}
\label{sec:Theorie}

\subsection{Kräfte}
Auf Körper die sich durch eine Flüssigkeit bewegen wirkt eine Reibungskraft $\vec{F}$,
die der Bewegungsrichtung entegengesetzt ist. Die Reibungskraft ist Abhängig von der Berührungsfläche A,
der Geschwindigkeit $v$ und der dynamischen Viskosität $η$.
Die Stokessche Reibung wird durch 
\\
\begin{equation}
    F_{R} = 6πηvr
\end{equation}
\\
angegeben. $\vec{F}_R$ steigt mit zunehmender Geschwindigkeit, bis sich ein Gleichgewicht zwischen $\vec{F}_R$, der Schwerkraft $\vec{F}_g$ 
und dem Auftrieb $\vec{F}_A$ einstellt. Auftriebs- und Reibungskraft wirken der Schwerkraft entgegen.
\\
\subsection{Viskosität}
% K gleichung mit eta
\begin{equation}
    η = K \cdot (ρ_k - ρ_{Fl}) \cdot t
\end{equation}
mit \begin{equation}
    ρ = \frac{m}{V}
\end{equation}
Für die Auswertung nach K umstellen:
\begin{equation} \label{eq:K}
    \Leftrightarrow K = \frac{η}{(ρ_K - ρ_{Fl}) \cdot t}
\end{equation}
\\
\begin{equation}
η(T) = A \cdot \symup{e}^{\left(\frac{B}{T}\right)}
\label{eqn:AndradscheGl}
\end{equation}
\\
\begin{equation}
η(T) = η_o \cdot exp \left(\frac{E_A}{R\cdot T}\right)
\end{equation}

\subsection{Fehlerrechnung}

\subsubsection{Berechnung des Arithmetischen Mittelwerts}
\begin{equation}
    \bar{x}_{arithm} = \frac{1}{n}  \sum_{i=1}^n x_i = \frac{x_1 + x_2 + \cdots + x_n}{n}
    \label{eqn:6}
\end{equation}

\subsubsection{Standardabweichung des Mittelwerts}
\begin{equation}
    \sigma = \sqrt{\sum_{i=1}^n (x_i - \bar{x}_{arithm})^2}
    \label{eqn:7}
\end{equation}
\subsubsection{Empirische Standardabweichung}
\begin{equation}
    s = \sqrt{\frac{1}{n - 1} \sum_{i=1}^n (x_i - \bar{x}_{arithm})^2}
    \label{eqn:8}
\end{equation}
%\cite{sample}

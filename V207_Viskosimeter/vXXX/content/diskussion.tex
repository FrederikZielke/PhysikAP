\section{Diskussion}
\label{sec:Diskussion}
Bei der Bestimmung des Durchmessers der Kugeln wurde nur eine Messung durchgeführt. Um die Genauigkeit zu erhöhen sollte hier aus mehreren Messungen ein Mittelwert 
gebildet werden. Die Kugeln waren an einigen Stellen beschädigt (kleine Absplitterungen). Dies könnte die unterschiedlichen Ergebnisse für die Dichten der Kugeln eine Erklärung sein.
Die Zeitenmessung mit der Stoppuhr lässt einen gewissen Spielraum für Anwendefehler zu, da zum Beispiel die Reaktionszeit des Anwenders, sowie der Waage das Ergebnis verfälschen.
Um hier genauere Messwerte aufnehmen zu können, könnte die Zeitenmessung mit einer Lichtschranke benutzt werden. 
Die Fallzeiten $t_{runter}$ sind immer etwas höher als die Fallzeiten $t_{hoch}$. Eine Erklärung dafür könnte der unterschiedliche Neigungswinkel bei der Drehung des 
Viskosimeters um 180° sein.
\\
\\
Der errechnete Wert der Viskosität bei $18.7\;\unit{°C}$ beträgt $\eta = 0.001239\pm0.000009\;\unit{Pa s}$. In der Literatur wird ein Wert von $\eta = 0.0010087\;\unit{Pa s}$ bei einer
Temperatur von $20\;\unit{°C}$ angegeben \cite{viskosität_wasser}. Dies bestätigt den Eindruck, dass die Messungen relativ genau ausgeführt wurden. 
\\
\\
Als kritische Reynoldszahl bei einer Rohrströmung gibt die Literatur $Re_{kritisch} \approx 2300$ an \cite{reynold_kritisch}. Die ermittelten Reynoldszahlen $Re_{max} = 112.1 \pm 0.9$ für $v_{max}$, also der Fallgeschwindigkeit
bei $56\unit{°C}$ und $Re_{min} = 28.27 \pm 0.21$ für $v_{min}$, bei $18.7\;\unit{°C}$ liegen deutlich unter dieser Grenze. Es kann davon ausgegangen werden, dass die Strömung laminar ist und keine 
Turbulenzen auftreten.
Der Anstieg der Reynoldszahl ist mit der fallenden Viskosität von Wasser bei steigender Temperatur verbunden. 
\\
\\
Die lineare Ausgleichsgerade liegt 





\newpage
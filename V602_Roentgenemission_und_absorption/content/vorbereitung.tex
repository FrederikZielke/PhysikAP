\section{Vorbereitungsaufgaben}
\label{sec:Vorbereitungsaufgaben}

\subsection{K-Kanten}

Die Kupfer $K_α-$ und $K_β-$Linien liegen bei Cu-$K_α = \SI{8.048}{\kilo\eV}$ und Cu-$K_β = \SI{8.907}{\kilo\eV}$\cite{energie_k}.
Um den Winkel $Θ$ zu berechnen, wird die Photonenenergie \eqref{eq:Energie_Photon} nach $λ$ umgestellt und in die Braggbedingung \eqref{eq:Bragg_Bedingung} eingesetzt.
Daraus ergibt sich für die Braggwinkel $Θ_{K_α} = 22.48°$ und $Θ_{K_β} = 20.21°$.


\subsection{Tabelle}

In der folgenden Tabelle sind die stoffspezifischen Braggwinkel, die Energie der $K$-Linien und die Abschirmkonstanten mehrerer Elemente aufgelistet\cite{energie_k}.
\begin{table}
    \centering
    \caption{Ordnungszahlen, Stoffspezifische Braggwinkel, Energie der $K$-Linien und Abschirmkonstanten tabellarisch aufgelistet.}
    \begin{tabular}{|c|c|c|c|c|}
        \toprule
        {} & {$Z$} & {$E_{K}^{Lit}\left[\unit{keV}\right]$\cite{energie_k}} & {$\Theta_{K}^{Lit}\left[\unit{°}\right]$} & {$\sigma_{K}$}\\
        \midrule
        Zn & 30 & 9.65 & 18.6 & 3.56 \\
        Ge & 32 & 11.11 & 16.08 & 3.66 \\
        Br & 35 & 13.48 & 13.2 & 3.84 \\
        Rb & 37 & 15.21 & 11.67 & 3.93 \\
        Sr & 38 & 16.12 & 11.01 & 3.98 \\
        Zr & 40 & 18.01 & 9.84 & 4.08 \\
        \bottomrule
    \end{tabular}
\end{table}
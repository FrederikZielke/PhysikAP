\section{Zielsetzung}
\label{sec:Zielsetzung}

Im Versuch 602 soll die Bragg Bedingung experimentell gezeigt werden, das Emissionsspektrum einer Kupfer-Röntgenröhre untersucht werden
und die Abschirmungskonstante $σ_k$ verschiedener Materialien bestimmt werden.

\section{Theorie}
\label{sec:Theorie}

\subsection{Erzeugung von Röntgenstrahlung}
\label{sec:Erzeugung_von_Roentgenstrahlung}

In einer evakuierten Röhre werden von einer Glühkathode Elektronen auf eine Anode Beschleunigt. 
Durch das Auftreffen der Elektronen auf die Anode, entsteht Röntgenstrahlung. Die Entstandene Strahlung
setzt sich aus der charakteristischen Röntgenstrahlung und der kontinuierlichen Bremsstrahlung zusammen. 
Die minimale Wellenlänge kommt bei der vollständigen Bremsung des Elektrons zustande.\\
\begin{equation*}\label{eq:lambda_min}
    \lambda_{min} = \frac{h \cdot c}{e_0 \cdot U}
\end{equation*}

Durch die auftreffenden Elektronen kann es dazu kommen, dass Elektronen der Anode auf eine höhere Schale 
gehoben werden. Die entstandene Lücke wird durch einen Elektronenübergang von einer höheren Schale gefüllt.
Die dabei frei werdenende Energie wird in Form von Röntgenstrahlung abgegeben. Da die Schalen feste Energieniveaus besitzen,
ist die Energiedifferenz $h \cdot ν = E_m - E_n$ immer gleich groß. Wegen der festen Quantisierung der Energie,
ist das charakteristische Spektrum in scharfe Linien aufgeteilt.
Die Linien werden beispielsweise $K_α \text{oder} L_β$ bezeichnet. Der Buchstabe bezieht sich auf die 
Schale auf dem das Elektron aufgenommen wird, der Index auf die Schale aus der das Elektron stammt.
Bei Atomen mit mehreren Elektronen schirmen die weiter innen liegenden Elektronen die äußeren ab.
Die Bindungsenergie eines Elektrons auf der n-ten Schale $E_n$ lässt sich berechnen mit
\begin{equation*}\label{eq:Bindungsenergie}
    E_n = -R_\infty z_{\text{eff}}^2 \cdot \frac{1}{n^2}.
\end{equation*}




%Aus der Bragg Bedingung ergibt sich mit 
%\begin{equation}
%    E_{Strahlung} = h \cdot f = \frac{h \cdot c}{\lambda} \Leftrightarrow \lambda = \frac{h \cdot c}{E_{Strahlung}}
%\end{equation}
%der Glanzwinkel 
%\begin{equation}
%    \Theta = \arcsin(\frac{n h c}{})
%\end{equation}
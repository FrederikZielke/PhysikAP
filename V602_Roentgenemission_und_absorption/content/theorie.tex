\section{Zielsetzung}
\label{sec:Zielsetzung}

Im Versuch 602 soll die Bragg Bedingung experimentell gezeigt werden, das Emissionsspektrum einer Kupfer-Röntgenröhre untersucht werden
und die Abschirmkonstante $σ_k$ verschiedener Materialien bestimmt werden.

\section{Theorie}
\label{sec:Theorie}

\subsection{Erzeugung von Röntgenstrahlung}
\label{sec:Erzeugung_von_Roentgenstrahlung}

In einer evakuierten Röhre werden von einer Glühkathode Elektronen auf eine Anode Beschleunigt. 
Durch das Auftreffen der Elektronen auf die Anode, entsteht Röntgenstrahlung. Die Entstandene Strahlung
setzt sich aus der charakteristischen Röntgenstrahlung und der kontinuierlichen Bremsstrahlung zusammen. 
Die minimale Wellenlänge kommt bei der vollständigen Bremsung des Elektrons zustande.\\
\begin{equation*}\label{eq:lambda_min}
    \lambda_{min} = \frac{h \cdot c}{e_0 \cdot U}
\end{equation*}

Durch die auftreffenden Elektronen kann es dazu kommen, dass Elektronen der Anode auf eine höhere Schale 
gehoben werden. Die entstandene Lücke wird durch einen Elektronenübergang von einer höheren Schale gefüllt.
Die dabei frei werdenende Energie wird in Form von Röntgenstrahlung abgegeben. Da die Schalen feste Energieniveaus besitzen,
ist die Energiedifferenz $h \cdot ν = E_m - E_n$ immer gleich groß. Wegen der festen Quantisierung der Energie,
ist das charakteristische Spektrum in scharfe Linien aufgeteilt.
Die Linien werden beispielsweise $K_α \text{oder} L_β$ bezeichnet. Der Buchstabe bezieht sich auf die 
Schale auf dem das Elektron aufgenommen wird, der Index auf die Schale aus der das Elektron stammt.
Bei Atomen mit mehreren Elektronen schirmen die weiter innen liegenden Elektronen die äußeren ab.
Die Bindungsenergie eines Elektrons auf der n-ten Schale $E_n$ lässt sich berechnen mit
\begin{equation*}\label{eq:Bindungsenergie}
    E_n = -R_\infty z_{\text{eff}}^2 \cdot \frac{1}{n^2}.
\end{equation*}
In der Gleichung ist $R_\infty$ die Rydbergenergie, $z_{\text{eff}}$ die effektive Kernladung und $n$ die Schale.
Die effektive Kernladung setzt sich zusammen aus der Kernladung $z$ und der Abschirmkonstante $σ$.
Die Abschirmkonstante ist experimentell bestimmbar und für jedes Elektron verschieden.
Elektronen auf äußeren Schalen haben leicht unterschiedliche Bindungsenergien, weshalb für höhere Schalen
mehrere charakteristische Linien eng beieinander liegen, diese Linien werden als Freinstruktur bezeichnet. Mit dem verwendeten Versuchsaufbau können diese Linien nicht aufgelöst werden.
\\

\subsection{Absorption}
\label{sec:Absorption}

Die vorherrschenden Prozesse bei Absorption im Bereich unter $\SI{1}{\mega\eV}$ sind der Comptoneffekt und
der Photoeffekt. Mit steigender Enerie sinkt der Absorptionskoeffizient.

Die Absorptionskante liegt quasi identisch zur Elektronbindungsenergie. Die jeweiligen Energien werden nach
ihrer Lage den verschiedenen Schalen zugeordnet und heißen Absorptionskanten.
Unter Beachtung der Freinstruktur kann die Bindungsenergie eines Elektrons mit der Sommerfeldschen Feinstrukturformel
ausgerechnet werden
\begin{equation*}\label{eq:Bindungsenergie_Sommerfeld}
    E_{n,j} = -R_\infty \left(z_{\text{eff},1}^2 \cdot \frac{1}{n^2} + α^2z_{\text{eff},2}^4 \cdot \frac{1}{n^3}\left(\frac{1}{j + \frac{1}{2}} - \frac{3}{4n}\right)\right).
\end{equation*}
Hier ist $α$ die Sommerfeldsche Feinstrukturkonstante, n die Hauptquantenzahl und j der Drehimpuls des Elektrons.
Die Abschirmkonstante $σ_{K,\text{abs}}$ lässt sich für die K-Schale bestimmen mit
\begin{equation*}\label{eq:Abschirmkonstante_K}
    σ_K = Z - \sqrt{\frac{E_K}{R_\infty} - \frac{α^2Z^4}{4}}.
\end{equation*}
Durch Vereinfachung, indem die Energiedifferenz zwischen zwei L-Kanten genommen wird, kann die Abschirmkonstante
$σ_L$ berechnet werden mit
\begin{equation*}\label{eq:Abschirmkonstante_L}
    σ_L = Z \cdot \left(\frac{4}{α}\sqrt{\frac{ΔE_L}{R_\infty}} - \frac{5ΔE_L}{R_{\infty}}\right)^{\frac{1}{2}}\left(1 + \frac{19}{32}α^2\frac{ΔE_L}{R_{\infty}}\right)^{\frac{1}{2}}.
\end{equation*}

\subsection{Bragg Reflexion}
\label{sec:Bragg_Reflexion}
Mit der Bragg'schen Reflexion lässt sich die Wellenlänge $λ$ der Röntgenstrahlung analysiert werden.
Hierzu wird der Röntgenstrahl auf ein dreidimensionales Gitter gerichtet. Die Photonen werden am Kristall 
gebeugt und interferieren miteinander. Nur im sogenannten Glanzwinkel $\Theta$ kommt es zu konstruktiver Interferenz.
Die Bragg'sche Bedingung stellt einen Zusammenhang zwischen der Wellenlänge $λ$, dem Glanzwinkel $\Theta$ und der Gitterkonstante $d$ her.
\begin{equation}\label{eq:Bragg_Bedingung}
    2 d \sin(\Theta) = n λ
\end{equation}


% eq:Bragg_Bedingung
% eq:Photonenenergie mit λ eingesetzt

%Aus der Bragg Bedingung ergibt sich mit 
%\begin{equation}
%    E_{Strahlung} = h \cdot f = \frac{h \cdot c}{\lambda} \Leftrightarrow \lambda = \frac{h \cdot c}{E_{Strahlung}}
%\end{equation}
%der Glanzwinkel 
%\begin{equation}
%    \Theta = \arcsin(\frac{n h c}{})
%\end{equation}
\section{Diskussion}
\label{sec:Diskussion}
Im folgenden werden die prozentualen Abweichungen mit 
\begin{equation}\label{1}
    \Delta = |\frac{exp - theo}{theo} \cdot 100|
\end{equation}
Zu Beginn wurde ein fester Sollwinkel von 14° am Röntgengerät eingestellt. Bei der Messung zur Überprüfung der Braggbedingung 
ergibt sich das Maximum der Kurve bei $\Theta_{Bragg} = 14.05°$. D.h. es liegt nach \autoref{1} eine Abweichung von 0.4\%.
Es kann also davon ausgegangen werden, dass die folgenden Messungen nicht durch eine Abweichung vom Sollwinkel verfälscht werden.
\\
\\
Bei der Messung zur Bestimmung des Röntgenemissionsspektrums von Kupfer zeigen sich nur geringe Abweichungen von
\begin{align*}
    \Delta E(K_{\alpha}) &= 0.05\%\\
    \text{und } \Delta E(K_{\beta}) &= 0.09\%.
\end{align*}
Bei der Messung zum Absorptionsspektrum treten folgende Abweichungen zur Literatur \cite{energie_k} auf
\begin{table}\label{tab:2}
    \centering
    \begin{tabular}{c c c c}
        \toprule
        {} & {$\Delta E_{K}\left[\unit{\%}\right]$} & {$\sigma_{K}\left[\unit{\%}\right]$}\\
        \midrule
        Zn & 7.0 & 17.2\\
        Br & 0.0 & 8.9 \\
        Sr & 0.4 & 9.5 \\
        Zr & 1.6 & 4.4 \\       
        \bottomrule
    \end{tabular}
    \caption{Abweichungen von Energie und Abschirmkonstante zur Literatur.}
  \end{table}
Die Abweichungen bei der Absorptionsenergie von Brom, Strontium und Zirkonium sind gering. Bei Zink ist die Abweichung eventuell damit zu erklären, dass entweder die Zinkfolie
des Absorbermediums eine andere Dicke oder beschädigt war. 
Die Abweichungen bei den Abschirmkonstanten sind damit zu erklären, dass bei der Berechnung viele Näherungen gemacht wurden. 
\\
\\
Die Rydbergenergie wird mithilfe einer Ausgleichsrechnung bestimmt, in die die errechneten Energien mit eingehen. Damit ist auch die Abweichung von 
\begin{equation*}
    \Delta R_{\infty} = 9.6\%.
\end{equation*}
\\
\\
Allgemein ist noch hinzuzufügen, dass die Messungen zum Absorptionsspektrum durch unabsichtliche kleine Verstellungen der Messapperaturen beim Wechseln der
Absorbermedien verfälscht sein könnten. 
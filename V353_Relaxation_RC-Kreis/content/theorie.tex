\section{Zielsetzung}
\label{sec:Zielsetzung}

Im Versuch 353 wird das Relaxationsverhalten eines RC-Schwingkreises untersucht. 
Die RC-Konstante des Kreises soll mithilfe des an einem Oszilloskop gemessenen Spannungsabfalls bestimmt werden.
Außerdem soll die Frequenzabhängigkeit der Kondensatorspannungsamplitude in einem RC-Kreis untersucht werden.

\section{Theorie}
\label{sec:Theorie}

\begin{equation*}
    \frac{\symup{d}A}{\symup{d}t} = c\left[A(t) - A(\infty)\right] .
\end{equation*}

\begin{equation*}
    \int_{A(0)}^{A(t)} \frac{\symup{d}A'}{A' - A(\infty)} = \int_{0}^{t} c\symup{d}t' .
\end{equation*}

\begin{equation*}
    \ln\left(\frac{A(t) - A(\infty)}{A(0) - A(\infty)}\right) = c t .
\end{equation*}

\begin{equation*}
    A(t) = A(\infty) + \left[A(0) - A(\infty)\right]\, \symup{e}^{\left(c t\right)}
\end{equation*}
gültig mit $c < 0$.

\begin{equation*}\label{eq:U_C}
    U_C = \frac{Q}{C}.
\end{equation*}

\begin{equation*}\label{eq:I}
    I = \frac{U_C}{R}.
\end{equation*}

\begin{equation*}
    \symup{d}Q = I\symup{d}t.
\end{equation*}

\begin{equation*}
    \frac{\symup{d}Q}{\symup{d}t} = -\frac{1}{RC}Q(t).
\end{equation*}
mit $Q(\infty) = 0.$

\begin{equation*}\label{eq:Q}%irgendwie stimmen hier die Größen von Q und e optisch nicht zueinander
    Q(t) = Q(0)\, \symup{e}^{-\frac{t}{RC}}.
\end{equation*}

%Aufladevorgang
Mit den Randbedinungnen $Q(0) = 0$ und $Q(\infty) = CU_0$
wird der Aufladevorgang durch die Gleichung 
\begin{equation*}\label{eq:Kondensatoraufladung}
    Q(t) = CU_0\left(1 - \symup{e}^{-\frac{t}{RC}}\right)
\end{equation*}
beschrieben.

\begin{equation*}
    \frac{Q(t = RC)}{Q(0)} = \frac{1}{\symup{e}} \approx 0.368.
\end{equation*}

%Relaxationsphänomene bei periodischer Auslenkung
\begin{equation*}
    U(t) = U_0 \cos{\left(\omega t\right)}.
\end{equation*}
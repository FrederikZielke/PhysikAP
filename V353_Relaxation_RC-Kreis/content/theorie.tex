\section{Zielsetzung}
\label{sec:Zielsetzung}

Im Versuch 353 wird das Relaxationsverhalten eines RC-Schwingkreises untersucht. 
Die RC-Konstante des Kreises soll mithilfe des an einem Oszilloskop gemessenen Spannungsabfalls bestimmt werden.
Außerdem soll die Frequenzabhängigkeit der Kondensatorspannungsamplitude in einem RC-Kreis untersucht werden.

\section{Theorie}
\label{sec:Theorie}


Wenn ein System aus seinem Ausganszustand ausgelenkt wird und es nicht-oszillatorisch wieder in diesen zurück kehrt, 
bezeichnet man dies als Relaxation. Die Änderungsrate der betrachteten Größe $A$ ist meist proportional zur Abweichung des Endzustandes $A(\infty)$.
\begin{equation*}
    \frac{\symup{d}A}{\symup{d}t} = c\left[A(t) - A(\infty)\right] 
\end{equation*}
Integriert man die Gleichung vom Zeitpunkt 0 bis zum Zeitpunkt t erhält man
\begin{equation*}\label{eq:int_A}
    \int_{A(0)}^{A(t)} \frac{\symup{d}A'}{A' - A(\infty)} = \int_{0}^{t} c\symup{d}t' .
\end{equation*}
%\begin{equation*}
%    \ln\left(\frac{A(t) - A(\infty)}{A(0) - A(\infty)}\right) = c t .
%\end{equation*}
Stellt man \eqref{eq:int_A} nach $A(t)$ um erhält man
\begin{equation*}
    A(t) = A(\infty) + \left[A(0) - A(\infty)\right]\, \symup{e}^{\left(c t\right)}
\end{equation*}
gültig mit $c < 0$.

Konkret betrachtet wird der Be- und Entladungsprozess eines Kondensators über einen Widerstand.

%\subsection{Kondensator}
Für die Spannung an einem Kondensator gilt
\begin{equation*}\label{eq:U_C}
    U_C = \frac{Q}{C}
\end{equation*}
und dem ohmschen Gesetz ergibt sich der Strom
\begin{equation*}\label{eq:I}
    I = \frac{U_C}{R}.
\end{equation*}

\begin{equation*}
    \symup{d}Q = I\symup{d}t.
\end{equation*}

\begin{equation*}
    \frac{\symup{d}Q}{\symup{d}t} = -\frac{1}{RC}Q(t).
\end{equation*}
mit $Q(\infty) = 0.$

\begin{equation*}\label{eq:Q}%irgendwie stimmen hier die Größen von Q und e optisch nicht zueinander
    Q(t) = Q(0)\, \symup{e}^{-\frac{t}{RC}}.
\end{equation*}

%Aufladevorgang
Mit den Randbedinungnen $Q(0) = 0$ und $Q(\infty) = CU_0$
wird der Aufladevorgang durch die Gleichung 
\begin{equation*}\label{eq:Kondensatoraufladung}
    Q(t) = CU_0\left(1 - \symup{e}^{-\frac{t}{RC}}\right)
\end{equation*}
beschrieben.

\begin{equation*}
    \frac{Q(t = RC)}{Q(0)} = \frac{1}{\symup{e}} \approx 0.368.
\end{equation*}

%Relaxationsphänomene bei periodischer Auslenkung
\begin{equation*}
    U(t) = U_0 \cos{\left(\omega t\right)}.
\end{equation*}

\begin{equation*}
    U_C(t) = A(ω) \cos{\left(ωt +  φ\{ω\}\right)}
\end{equation*}

\begin{equation*}\label{eq:Uges}
    U(t) = U_R(t) + U_C(t)
\end{equation*}

\begin{equation*}
    U_0\cos{ωt} = I(t)R + A(ω)\cos{\left(ωt +  φ\right)}
\end{equation*}

\begin{equation*}\label{eq:Strom_I}
    I(t) = \frac{\symup{d}Q}{\symup{d}t} = C\frac{\symup{d}U_C}{\symup{d}t}
\end{equation*}

\begin{equation*}
    U_0\cos{ωt} = -AωRC\sin{\left(ωt + φ\right)} + A(ω)\cos{\left(ωt + φ\right)}
\end{equation*}

\begin{equation*}
    0 = -ωRC\sin{\left(\frac{π}{2} + φ\right)} + \cos{\left(\frac{π}{2} + φ\right)}
\end{equation*}
daraus folgt wegen $\sin{\left(φ + \frac{π}{2}\right)} = \cos{φ}$ und $\cos{\left(φ + \frac{π}{2}\right)} = -\sin{φ}$
\begin{equation*}
    \frac{\sin{φ}}{\cos{φ}} = \tan{φ (ω)} = -ωRC \quad \text{oder} \quad φ(ω) = \arctan{\left(-ωRC\right)}
\end{equation*}

\begin{equation*}
    ωt + φ = \frac{π}{2}
\end{equation*}

\begin{equation*}
    U_0\cos{\left(\frac{π}{2} - φ\right)} = -AωRC
\end{equation*}

\begin{equation*}
    A(ω) = -\frac{\sin{φ}}{ωRC}U_0
\end{equation*}

\begin{equation*}
    \sin{φ}^2 + \cos{φ}^2 = 1
\end{equation*}

\begin{equation*}
    \sin{φ} = \sqrt{ωRC}{\sqrt{1 + ω^2R^2C^2}}
\end{equation*}

\begin{equation*}
    A(ω) = \frac{U_0}{\sqrt{1 + ω^2R^2C^2}}
\end{equation*}

Mit der Gleichung \eqref{eq:Uges} ergibt sich für
\begin{equation*}
    U(t) = I(t) R + U_C(t).
\end{equation*}
Jetzt Gleichung \eqref{eq:Strom_I} einsetzen um
\begin{equation*}\label{eq:Uges2}
    U(t) = RC\frac{\symup{d}U_C}{\symup{d}t} + U_C(t).
\end{equation*}
Wird $ω >> \frac{1}{RC}$ angenommen so ist $\left\lvert U_C\right\rvert << \left\lvert U_R\right\rvert $
und $\left\lvert U_C\right\rvert << \left\lvert U\right\rvert $.
Dann kann \eqref{eq:Uges2} annähern als
\begin{equation*}
    U(t) = RC\frac{\symup{d}U_C}{\symup{d}t}
\end{equation*}
oder als
\begin{equation*}
    U_C(t) = \frac{1}{RC} \int_{0}^{t} U(\tau)\symup{d}\tau.
\end{equation*}

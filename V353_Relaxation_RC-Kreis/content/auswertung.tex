\section{Auswertung}
\label{sec:Auswertung}

Die in \autoref{sec:Auswertung} gezeigten Grafiken und Ausgleichsrechnungen sind mithilfe der Python-Bibliotheken Matplotlib \cite{matplotlib}, Scipy \cite{scipy} und Numpy \cite{numpy}
erstellt worden.

\subsection{Bestimmung der Zeitkonstante einer RC Glieds}
\begin{table}[H]
    \centering
    \caption{Abgelesene Kondensatorspannung in Abhängigkeit der Zeit.}
    \begin{tabular}{
      S[table-format=4.0]
      S[table-format=2.3]
    }
      \toprule
      {$t\left[\unit{ms}\right]$} & {$U_C\left[\unit{V}\right]$}\\
      \midrule
      0.0 & 4.00 \\
      0.4 & 3.40\\
      0.8 & 3.00\\
      1.2 & 2.50\\
      1.6 & 2.20\\
      2.0 & 1.90\\
      2.4 & 1.70\\
      2.8 & 1.40\\
      3.2 & 1.30\\
      3.6 & 1.10\\
      4.0 & 0.90\\
      4.4 & 0.80\\
      4.8 & 0.70\\
      5.2 & 0.60\\
      5.6 & 0.50\\
      6.0 & 0.40\\
      6.4 & 0.40\\
      6.8 & 0.35\\
      7.2 & 0.30\\
      7.6 & 0.25\\
      8.0 & 0.23\\
      8.4 & 0.20\\
      8.8 & 0.20\\
      9.2 & 0.17\\
      9.6 & 0.15\\
      10.0& 0.10\\
      12.0 & 0.00\\
      \bottomrule
  \end{tabular}
  \end{table}  

Die gemessenen Werte können nun als Funktion von $t$ dargestellt werden. Wird der ln$(\frac{U_C}{U_0})$ gegen $t$ aufgetragen, so entsteht \autoref{rc_graph}.
\begin{figure}[H]
    \includegraphics[width=\linewidth]{build/entladekurve.pdf}
    \caption{Amplitudenverhältnis $\frac{U_C}{U_0}$ logarithmisch gegen die Zeit $t$ aufgetragen.}
    \label{rc_graph}
\end{figure} 
Durch eine lineare Ausgleichsrechnung wird die Funktion
\begin{equation*}
    f(x) = ax + b
\end{equation*}
an die Messwerte gefittet. Mit der Python Bibliothek SciPy \cite{scipy} ergibt sich für die Koeffizienten $a = \SI{-0.351(0.004)}{\frac{1}{s}}$ und $b = -0.04 \pm 0.02$.
Mit dem Zusammenhang 
\begin{equation*}
    \ln{\frac{U_C}{U_0}} = - \frac{t}{RC}
\end{equation*}
lässt sich $RC$ zu $RC = -\frac{1}{a}$ bestimmen, wobei $a$ die Steigung der Ausgleichsgeraden ist. Dadurch ergibt sich für die Zeitkonstante
\begin{equation*}
    RC = -\frac{1}{a} = \SI{2.849(0.032)}{ms}.
\end{equation*}
  
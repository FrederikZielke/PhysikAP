\section{Diskussion}
\label{sec:Diskussion}
%Unterschied der beiden Zeitkonstanten
%Sachen die man Diskutieren kann: Zu wenig Frequenzen gemessen um die Frequenzabhänigkeit richtig zu zeigen
%Warum weichen unsere Werte so krass ab? (Fehler bei der Messung, Fehler bei der Berechnung)
%RC-Konstante Entladevorgang = (2.849 ± 0.032) ms; Phasenabhängig = (4.8 ± 0.1) ms; Phasenverschiebung = (3.3 ± 0.7) ms
%was ist bei Abbildung 6 eig mit diesem einen Ausreisser bei f=10^3?
Die in \autoref{sec:Auswertung} berechneten Zeitkonstanten weichen voneinander ab. Die in \autoref{sec:4.2} bestimmte Zeitkonstante $RC_2$ist 
$68.4\%$ größer als die in \autoref{sec:4.1} bestimmte Zeitkonstante $RC_1$. Die in \autoref{sec:4.3} bestimmte Zeitkonstante $RC_3$ ist $15.8\%$ größer als $RC_1$.
Die Zeitkonstante $RC_2$ liegt auch mit Beachtung des Fehlers weder im Bereich von $RC_1$ noch $RC_3$. 
Die Zeitkonstante $RC_3$ liegt im Fehlerbereich von $RC_1$.
Ein Grund für die Abweichung können systematische Fehler bei der Messung sein.%Ergibt das irgendeinen Sinn? hahaha 
\\
\\
%Die Messwerte aus \autoref{tab:phase} weichen
Die mithilfe von \autoref{tab:phase} berechneten Phasenverschiebungen sind zum Teil nicht in einem klaren Trend zu erkennen.
Dies könnte auf Fehler bei der Aufnahme der Messwerte zurückzuführen sein. Es kam zu großen Ungenauigkeiten beim ablesen 
der Werte auf dem Oszilloskop, da sich der Messbereich aufgrund eines Defekts am Gerät nicht sinnvoll anpassen ließ. 
Um festzustellen ob die Fitfunktion mit $\arctan$ tatsächlich die gemessenen Werten beschreibt, könnte mit mehr Messungen bei höheren Frequenzen überprüft werden. 
Auf den Polarplot aufgetragen ist erkennbar, dass die Messwerte etwas kleiner als theoretisch erwartet sind. Dieser Trend kann jedoch auch nur eindeutig mit mehr Messungen bestätigt werden.
\\
\\
Die integrierten Spannungen entsprechen der erwarteten Form.
Die auffälige Amplitudendifferenz von $U_C$ und $U_0$ lässt sich auf den $\frac{1}{RC}$ Faktor zurückführen.
Qualitativ lässt sich damit die Funktionsweise eines RC-Gliedes als Integrator bestätigen. 
\newpage
\section{Durchführung}
\label{sec:Durchführung}

\subsection{Aufbau}
\label{sec:Aufbau}

\subsection{Messung}
\label{sec:Messung}

Im ersten Aufgabenteil wird die Zeitkonstante eines RC-Gliedes ermittelt.
Dazu wird eine Rechteckspannung an den Kondensator angeschlossen. Mit einem Oszilloskop wird die Spannung $U_C$ am Kondensator gemessen.
Die Frequenz der Rechteckspannung und der Messbereich des Oszilloskops werden so gewählt, dass ein vollständiger Be- oder Entladevorgang
auf dem Oszilloskop abgebildet werden kann. Es werden Messwertpaare der Form $(t, U_C)$ aufgezeichnet.

Im zweiten Aufgabenteil wird die Frequenzabhängigkeit der Kondensatorspannung $U_C$ untersucht.%Ich hab kb mehr und hoer jetzt auf
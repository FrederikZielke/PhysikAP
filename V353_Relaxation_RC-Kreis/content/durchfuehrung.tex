\section{Durchführung}
\label{sec:Durchführung}

%Wollen wir noch Abbildungen einfügen?

\subsection{Aufbau}
\label{sec:Aufbau}

Für alle Versuchsteile wird ein Generator mit variablem Schwinungsmuster verwendet. Ein Tiefpassfilter wird an den Generator angeschlossen.
Der Tiefpassfilter wird mit einem Kondensator und einem Widerstand realisiert. Die Kondensatorspannung wird mit einem Oszilloskop gemessen.

\subsection{Messung}
\label{sec:Messung}

Im ersten Aufgabenteil wird die Zeitkonstante eines RC-Gliedes ermittelt.
Dazu wird eine Rechteckspannung an den Kondensator angeschlossen. Mit dem Oszilloskop wird die Spannung $U_C$ am Kondensator gemessen.%doppelt gemoppelt?
Die Frequenz der Rechteckspannung und der Messbereich des Oszilloskops werden so gewählt, dass ein vollständiger Be- oder Entladevorgang
auf dem Oszilloskop abgebildet werden kann. Es werden Messwertpaare der Form $(t, U_C)$ aufgezeichnet.

Im zweiten Aufgabenteil wird die Frequenzabhängigkeit der Kondensatorspannung $U_C$ untersucht. Dazu wird eine Sinusspannung an den Kondensator angeschlossen.
Mit einem Oszilloskop wird die Spannung $U_C(t)$ am Kondensator und die Generatorspannung $U_{\symup{G}}(t)$ gemessen. %Auch überflüssig?
Die Frequenz der Sinusspannung wird von 41.1 Hz bis 1 kHz gesteigert.
Die Generatorspannung und die Kondensatorspannung werden auf dem Oszilloskop gegen die Zeit abgebildet.
Es werden immer vier Messwerte aufgenommen. Die Frequenz $f$, die Generatorspannung $U_{\symup{G}}$, der zeitliche Abstand $a$ der beiden Nulldurchgänge der Schwinungen und die Schwinungsdauer $b$ einer Schwingung werden gemessen.

%Ich dachte irgendwie es gäbe noch einen dritten Aufgabenteil
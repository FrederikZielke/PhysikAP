\section{Zielsetzung}
\label{sec:Zielsetzung}

Mithilfe eines Ultraschall Doppler-Generators im Pulsbetrieb sollen das Stromungsprofil und die Fließgeschwindigkeit
in Abhängigkeit des Dopplerwinkels und der Strömungsgeschwindigkeit ermittelt werden.

\section{Theorie}
\label{sec:Theorie}

In einem Frequenzbereich von ca. $\SI{20}{\kilo\Hz}$ bis $\SI{1}{\giga\Hz}$ wird von Ultraschall gesprochen.
%Ich werde hier mal die erzeugung vom Ultraschall reinpacken obwohl es in der Anleitung erst viel später kommt
%Ultraschall kann auf unterschiedliche Weise erzeugt werden.
Im Experiment wird Ultraschall durch den reziproken piezo-elektrischen Effekt erzeugt. 
Durch ein elektrisches Wechselfeld schwingt ein piezoelektrischer Kristall und sendet Ultraschallwellen aus.
Die Schwingungsstärke des Kristalls ist am größten, wenn die Frequenz des Wechselfelds seiner Eigenfrequenz entspricht. 
Dann kann er sehr hohe Schallenergien erzeugen. Der Piezokristall kann auch als Schallempfänger dienen, 
indem er durch Schallwellen in Schwingung versetzt wird. Quarze sind die verbreitetsten piezoelektrischen Kristalle, 
weil ihre physikalischen Eigenschaften konstant sind. Ihr piezoelektrischer Effekt ist aber eher schwach.
Die verwendete Sonde besitzt einen Ultraschallempfenger und Sender.
Der Doppler-Effekt beschreibt die Zu- oder Abnahme der Frequenz einer Schallwelle, wenn sich die Schalquelle und 
der Beobachter relativ zueinander bewegen. 
%Ich bin mir nicht sicher welche von beiden Formeln wir tatsächlich brauchen aber ich schreibe erstmal beide auf
%man kann ja dann noch entscheiden welche wir verwenden
Entfernt die Quelle sich vom Boeobachter, so sinkt die Frequenz $ν_{gr}$, nähert Sie sich, so steigt die Frequenz $ν_{kl}$.
\begin{equation}\label{eq:Doppler1}
    ν_{kl/gr} = \frac{ν_0}{1 \mp \frac{v}{c}}.%eigenlich muss das hier mit sicherheit in einen richtigen Satz eingegliedert werden aber who cares
\end{equation}
Bewegt sich nur der Beobachter und die Quelle ruht, so ergibt sich die Formel
\begin{equation}\label{eq:Doppler2}
    ν_{h/n} = ν_0 \left(1 \pm \frac{v}{c}\right)
\end{equation}
für die Frequenzänderung.\\

Der Doppler-Effekt kann genutzt werden, um die Geschwindigkeit von bewegten Objekten zu bestimmen.
Dadurch können beispielsweise Blutströme in Gefäßen oder die Bewegung von Flüssigkeiten in Rohren gemessen werden.
%was fuer ein seltsamer Satz aber naja
Die Frequenzverschiebung wird beschrieben durch
\begin{equation*}
    Δν - ν_0 \frac{v}{c} \left(\cos{α} + \cos{β}\right).
\end{equation*}
Bei dem im Aufbau verwendeten Impuls-Echo-Verfahren ist $α = β$, wodurch sich die Formel zu 
\begin{equation}\label{eq:Frequenzverschiebung}
    Δν = 2\, ν_0 \frac{v}{c} \cos{α}
\end{equation}
vereinfacht.\\
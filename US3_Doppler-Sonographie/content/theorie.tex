\section{Zielsetzung}
\label{sec:Zielsetzung}

Mithilfe eines Ultraschall Doppler-Generators im Pulsbetrieb sollen das Stromungsprofil und die Fliessgeschwindigkeit
in Abhängigkeit des Dopplerwinkels und der Strömungsgeschwindigkeit ermittelt werden.

\section{Theorie}
\label{sec:Theorie}

In einem Frequenzbereich von ca. $\SI{20}{\kilo\Hz}$ bis $\SI{1}{\giga\Hz}$ wird von Ultraschall gesprochen.
%Ich werde hier mal die erzeugung vom Ultraschall reinpacken obwohl es in der Anleitung erst viel später kommt
%Ultraschall kann auf unterschiedliche Weise erzeugt werden.
Im Experiment wird Ultraschall durch den reziproken piezo-elektrischen Effekt erzeugt. 
Durch ein elektrisches Wechselfeld schwingt ein piezoelektrischer Kristall und sendet Ultraschallwellen aus.
Die Schwingungsstärke des Kristalls ist am größten, wenn die Frequenz des Wechselfelds seiner Eigenfrequenz entspricht. 
Dann kann er sehr hohe Schallenergien erzeugen. Der Piezokristall kann auch als Schallempfänger dienen, 
indem er durch Schallwellen in Schwingung versetzt wird. Quarze sind die verbreitetsten piezoelektrischen Kristalle, 
weil ihre physikalischen Eigenschaften konstant sind. Ihr piezoelektrischer Effekt ist aber eher schwach.
Die verwendete Sonde besitzt einen Ultraschallempfänger und Sender.
Der Doppler-Effekt beschreibt die Verschiebung der Frequenz einer Schallwelle, wenn sich die Schallquelle und 
der Beobachter relativ zueinander bewegen. 
%Ich bin mir nicht sicher welche von beiden Formeln wir tatsächlich brauchen aber ich schreibe erstmal beide auf
%man kann ja dann noch entscheiden welche wir verwenden
Entfernt die Quelle sich vom Beobachter, so verschiebt sich die Ausgangsfrequenz $f_0$ zu einer größeren Frequenz $f_{gr}.$
Nähert die Quelle sich, so verschiebt sich die Frequenz $f_0$ zu einer kleineren Frequenz $f_{kl}.$
\begin{equation}\label{eq:Doppler1}
    f_{gr,kl} = \frac{f_0}{1 \mp \frac{v}{c}}%eigenlich muss das hier mit sicherheit in einen richtigen Satz eingegliedert werden aber who cares
\end{equation}
mit der Ausgangsfrequenz $f_0$ der Quelle, der Geschwindigkeit $v$ der Quelle und der Schallgeschwindigkeit $c$.\\
Bewegt sich nur der Beobachter und die Quelle ruht, so ergibt sich die Formel
\begin{equation}\label{eq:Doppler2}
    f_{h/n} = f_0 \left(1 \pm \frac{v}{c}\right)
\end{equation}
für die Frequenzänderung.\\

Der Doppler-Effekt kann genutzt werden, um die Geschwindigkeit von bewegten Objekten zu bestimmen.
Dadurch können beispielsweise Blutströme in Gefäßen oder die Bewegung von Flüssigkeiten in Rohren gemessen werden.
%was fuer ein seltsamer Satz aber naja
Die Frequenzverschiebung wird beschrieben durch
\begin{equation*}
    Δf - f_0 \frac{v}{c} \left(\cos{α} + \cos{β}\right).
\end{equation*}%hier noch etwas zu alpha und beta schreiben
Hierbei beschreibt $α$ den Winkel zwischen dem bewegten Objekt und dem Ultraschallsender und $β$ den Winkel zwischen dem Objekt 
und dem Ultraschallempfänger.\\
Die Frequenzverschiebung berechnet sich hierbei mit 
\begin{equation}\label{eq:f_ver}
    \Delta f = |f_{max} - f_{mean}|.
\end{equation}
Bei dem im Aufbau verwendeten Impuls-Echo-Verfahren ist $α = β$, da sich Empfänger und Sender in der gleichen Sonde befinden. Dadurch vereinfacht sich die Formel zu 
\begin{equation}\label{eq:Frequenzverschiebung}
    Δf = 2\, f_0 \frac{v}{c} \cos{α}.
\end{equation}
\\
%Das muss bestimmt auch in die Theorie aber kp wo
Das Doppler-Prisma wird verwendet um die Ultraschallsonde direkt an die Rohre anzulegen.
Es besitzt drei Einstellflächen, die es ermöglichen in Reproduzierbaren Dopplerwinkeln zu arbeiten.
In jedem der Winkel ist der Abstand zwuschen der Ultraschallsonde under strömenden Flüssigkeit gleich groß.
Mit 
\begin{equation}\label{eq:Dopplerwinkel}
    α = 90° - \arcsin{\left(sin{Θ}\frac{c_L}{c_P}\right)}
\end{equation}
kann der Dopplerwinkel berechnet werden, wobei $Θ$ der Prismawinkel, $c_L$ die Schallgeschwindigkeit in der Dopplerflüssigkeit und $c_P$ die Schallgeschwindigkeit im Prisma ist.
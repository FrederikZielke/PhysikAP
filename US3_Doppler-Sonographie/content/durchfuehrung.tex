\section{Durchführung}
\label{sec:Durchführung}

\subsection{Aufbau}

Für alle Versuchsteile wird ein Computer zur Datenaufnahme und Verwertung verwendet.
An den Rechner sind eine Ultraschallsonde und ein Ultraschall Doppler-Generator im Pulsbetrieb angeschlossen.
Es gibt mehrere Rohre mit verschiedenen Maßen. Die Rohre sind mit Glaskugeln, Wasser und Glyzerin gefüllt.
Durch die Zusammensetzung der Flüssigkeit bildet sich eine laminare Strömung bei mittlerer Strömungsgeschwindigkeit.
Die aufgenommenen Daten werden mit dem Programm FlowView dargestellt und verwertet. 
Die Ultraschallsonde wird mit einem Doppler-Prisma an das Rohr angeschlossen.

\subsection{Messung}

Im ersten Teil des Experiments wird an einem Rohr, für 5 verschiedene Flussgeschwindigkeiten die Beziehung zwischen 
der Strömungsgeschwindigkeit und dem Dopplerwinkel bestimmt. Die Einstellung \texttt{SAMPLE VOLUME} wird auf \texttt{LARGE} gesetzt.
Nach einstellen einer Geschwindigkeit wird für alle drei Dopplerwinkel $Δν$ gemessen. Die Messung wird für vier weitere Geschwindigkeiten wiederholt.

Im zweiten Teil des Experiments wird unter einem Dopplerwinkel von 15°, an einem $3/4$ Zoll Rohr, das Strömungsprofil der 
Flüssigkeit gemessen. Für eine variable Meßtiefe wird \texttt{SAMPLE VOLUME} auf \texttt{SMALL} gestellt.
Die Messung wird in einem Bereich von $\SI{30}{\m\m}$ bis $\SI{11}{\m\m}$ %kann es sein dass das 110mm sein muss?
mit einer Schrittweite von $\SI{0.75}{\m\m}$ durchgeführt. Die Pumpleistung wir auf $70\%$ eingestellt.
Die Strömungsgeschwindigkeit und Streuintensität werden gemessen.\\
Die Messung wird wird für eine halbierte Messleistung von $45\%$ wiederholt.\\
\section{Diskussion}
\label{sec:Diskussion}

Bei allen drei Dopplerwinkeln stellt sich ein linearer Zusammenhang zwischen $\frac{Δf}{\cos{α}}$ und
der Strömungsgeschwindigkeit $v$ ein. Dieser ist nach \autoref{eq:Doppler2} auch zu erwarten.\\
\\
Im zweiten Teil der Auswertung schwanken die Messwerte.
Zum äußeren Rand bildet sich ein Plateau. Die Annahme ist, dass sich das Strömungsprofil wie eine Parabel verhält.
Außerdem bildet sich bei dem Strömungsprofil bei 3870rpm ein Plateau in der Mitte. 
Dies könnte damit zu begründen sein, dass das Messgerät nur grob quantisierte Geschwindigkeiten angezeigt hat.
Außerdem ist im Verlauf des Experiments aufgefallen, dass die Pumpgeschwindigkeit nicht konstant ist.\\
\\
Das Maximum liegt wie zu erwarten sowohl bei der Strömungsgeschwindigkeit, sowie der Intensität in etwa der Mitte des Rohrs.
Der Verlauf der Intensität entspricht wie annähernd einer Parabel.
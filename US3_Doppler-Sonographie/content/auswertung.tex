\section{Auswertung}
\label{sec:Auswertung}
Die in \autoref{sec:Auswertung} gezeigten Grafiken und Ausgleichsrechnungen sind mithilfe der Python-Bibliotheken Matplotlib \cite{matplotlib}, Scipy \cite{scipy} und Numpy \cite{numpy}
erstellt worden.
\subsection{Bestimmung der Strömungsgeschwindigkeit}
Die aufgenommen Ergebnisse der ersten Messreihe werden mit \autoref{eq:f_ver} 
in die Frequenzverschiebung umgerechnet und sind in \autoref{tab:deltaf} aufgelistet.
\begin{table}[H]
    \centering
    \caption{Frequenzverschiebung bei verschiedenen Prisma-Winkeln und steigender Umdrehungszahl der Pumpe.}
    \begin{tabular}{c c c c}
        \toprule
        {rpm} & {$\Delta f_{15} [\unit{Hz}]$} & {$\Delta f_{30} [\unit{Hz}]$} & {$\Delta f_{60} [\unit{Hz}]$}\\
        \midrule
        2020 & 36 & 35 & 51 \\
        3060 & 73 & 90 & 129\\
        4380 & 144& 181& 273\\
        5180 & 280& 375& 414\\
        8400 & 366& 460& 660\\
        \bottomrule
    \end{tabular}
    \label{tab:deltaf}
\end{table}
Mit den zuvor berechneten Dopplerwinkeln aus \autoref{sec:Vorbereitung}, den Frequenzverschiebungen aus \autoref{tab:deltaf}
, $f_{0} = \SI{2}{MHz}$, $c_{Fl} = \SI{1800}{\frac{m}{s}}$ und \autoref{eq:Frequenzverschiebung} ergeben sich die
Strömungsgeschwindigkeiten in \autoref{tab:v}.
\begin{table}[H]
    \centering
    \caption{Strömungsgeschwindigkeiten bei verschiedenen Prisma-Winkeln und steigender Umdrehungszahl der Pumpe.}
    \begin{tabular}{c c c c}
        \toprule
        {rpm} & {$v_{15} [\unit{m/s}]$} & {$v_{30} [\unit{m/s}]$} & {$v_{60} [\unit{m/s}]$}\\
        \midrule
        2020 & 0.10 & 0.05 & 0.04 \\
        3060 & 0.20 & 0.12 & 0.1\\
        4380 & 0.40& 0.24& 0.21\\
        5180 & 0.77& 0.51& 0.32\\
        8400 & 1.01& 0.62& 0.51\\
        \bottomrule
    \end{tabular}
    \label{tab:v}
\end{table}
Wird die Geschwindigkeit $v_k$ gegen $\frac{\Delta f_k}{cos(\alpha)}$ aufgetragen so entstehen die Plots in \autoref{fig:15}, \autoref{fig:30} und \autoref{fig:45}.
\begin{figure}[H]
    \includegraphics[width=\textwidth]{build/15.pdf}
    \caption{Abhängigkeit der Strömungsgeschwindigkeit und dem ersten Dopplerwinkel.}
    \label{fig:15}
\end{figure}
\begin{figure}[H]
    \includegraphics[width=\textwidth]{build/30.pdf}
    \caption{Abhängigkeit der Strömungsgeschwindigkeit und dem zweiten Dopplerwinkel.}
    \label{fig:30}
\end{figure}
\begin{figure}[H]
    \includegraphics[width=\textwidth]{build/45.pdf}
    \caption{Abhängigkeit der Strömungsgeschwindigkeit und dem dritten Dopplerwinkel.}
    \label{fig:45}
\end{figure}

%Die Frequenzverschiebung berechnet sich hierbei mit 
%\begin{equation}\label{eq:f_ver}
%    \Delta f = |f_{max} - f_{mean}|.
%\end{equation}

\subsection{Bestimmung des Strömungsprofils eines Rohres mit 10mm Durchmesser und einem Prismawinkel von 15°}
Die Ergebnisse aus der zweiten Messreihe sind in \autoref{tab:s_v_I} aufgelistet. Dabei sind die Messdaten bei 3870rpm mit einer 1 indiziert und die 
Messdaten bei 6020rpm mit einer 2.
\begin{table}[H]
    \centering
    \caption{Strömungsgeschwindigkeit und Signalintensität bei verschiedenen Messtiefen.}
    \begin{tabular}{c c c c c}
        \toprule
        {$s [\unit{mm}]$} & {$v_{1} [\unit{m/s}]$} & {$I_{1} [\unit{1000V^2/s}]$} & {$v_{2} [\unit{m/s}]$} & {$I_{2} [\unit{1000V^2/s}]$} \\
        \midrule
        12   & 22.3 & 200  & 31.8 & 2212\\
        12.5 & 22.3 & 604  & 35   & 4395\\
        13   & 22.3 & 1167 & 38.2 & 5892\\
        13.5 & 23.9 & 1664 & 41.4 & 6543\\
        14   & 25.5 & 1738 & 50.9 & 5903\\
        14.5 & 28.7 & 1678 & 57.3 & 3976\\
        15   & 31.8 & 1409 & 63.7 & 2823\\
        15.5 & 31.8 & 1012 & 66.9 & 2077\\
        16   & 31.8 & 756  & 66.9 & 1337\\
        16.5 & 31.8 & 656  & 63.7 & 987\\
        17   & 31.8 & 458  & 60.5 & 766\\
        17.5 & 28.7 & 392  & 54.1 & 613\\
        18   & 28.7 & 324  & 55.7 & 522\\
        18.5 & 28.7 & 375  & 54.1 & 729\\
        19   & 28.7 & 399  & 54.1 & 439\\
        19.5 & 27.1 & 383  & 54.1 & 361\\
        \bottomrule
    \end{tabular}
    \label{tab:s_v_I}
\end{table}
In \autoref{fig:s_gegen_v} ist $s$ gegen die Strömungsgeschwindigkeiten $v_1$ und $v_2$ geplottet. Desweiteren ist in \autoref{fig:s_gegen_I} $s$ gegen $I_1$ und $I_2$ aufgetragen.
\begin{figure}[H]
    \includegraphics[width=\textwidth]{build/s_gegen_v.pdf}
    \caption{Abhängigkeit der Strömungsgeschwindigkeit von der Messtiefe.}
    \label{fig:s_gegen_v}
\end{figure}
\begin{figure}[H]
    \includegraphics[width=\textwidth]{build/s_gegen_I.pdf}
    \caption{Abhängigkeit der Signalintensität von der Messtiefe.}
    \label{fig:s_gegen_I}
\end{figure}
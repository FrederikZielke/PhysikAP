\section{Zielsetzung}
\label{sec:Zielsetzung}

Im Versuch V354 soll die Dämpfung eines RLC-Serienschwingkreises ermittelt werden. 
Außerdem soll die Frequenzabhängigkeit der Kondensatorspannung, sowie die Phasenverschiebung zwischen Kondensator- und Erregerspannung bestimmt werden.

\section{Theorie}
\label{sec:Theorie}

\subsection{Gedämpfte Schwingung}
\label{sec:Gedaempfte_Schwingung}

Ein RCL-Schwingkreis besteht aus zwei Energiespeichern, einem Kondensator und einer Spule, und einem Widerstand. 
Das schwingende Verhalten des Stromflusses und der Spannung entsteht durch die beiden Speicher. Durch den Ohmschen Widerstand
ist der Schwingkreis gedämpft. Die Dämpfung wird durch den Widerstand bestimmt. Ein ungedämpfter Schwingkreis 
ist nur theoretisch möglich, da die Verbindungskabel und die Spule selbst einen Widerstand aufweisen.
Durch den Widerstand wird Energie irreversibel in Wärme umgewandelt. Bei einmaliger Anregung geht die Amplitude der Schwingung
mit der Zeit gegen Null.
Nach dem zweiten Kirchhoffschnen Gesetz setzt sich die Gesamtspannung im Kreis zu
\begin{equation*}
    U_R(t) + U_C(t) + U_L(t) = 0
\end{equation*}
zusammen. 
Die einzelnen Spannungen lassen sich als
\begin{equation*}
    U_R(t) = RI(t) \quad \text{und} \quad U_C(t) = \frac{Q(t)}{C}  \quad \text{und} \quad U_L(t) = L\frac{\symup{d}I(t)}{dt}
\end{equation*}
ausdrücken. 
Durch einsezten der Spannungen in die Gesamtspannung ergibt sich 
\begin{equation*}
    L\frac{\symup{d}I(t)}{dt} + RI(t) + \frac{Q(t)}{C} = 0.
\end{equation*}
Abgeleitet nach $t$ ergibt sich mit $I = \frac{\symup{d}Q}{dt}$ die Gleichung
\begin{equation}\label{eq:DGL1}
    \frac{\symup{d}^2I}{dt^2} + \frac{R}{L}\frac{\symup{d}I}{dt} + \frac{1}{LC}I = 0.
\end{equation}
Mit dem Ansatz
\begin{equation}\label{eq:e-ansatz}
    I(t) = A\symup{e}^{i\tilde{ω}t}
\end{equation}
, mit $i = \sqrt{-1}$ und $\tilde{ω}, A$ komplexe Zahlen,
lässt sich die Differentialgleichung schreiben als
\begin{equation*}
    \left(-\tilde{ω}^2 + \frac{iR}{L}\tilde{ω} + \frac{1}{LC}\right)A\symup{e}^{i\tilde{ω}t}.
\end{equation*}
Nun mit \autoref{eq:e-ansatz} die charakteristische Gleichung aufstellen
\begin{equation*}
    -\tilde{ω}^2 + \frac{iR}{L}\tilde{ω} - \frac{1}{LC} = 0.
\end{equation*}
Durch anwenden der PQ-Formel ergibt sich für die Konstante $\tilde{ω}$
\begin{equation*}
    \tilde{ω}_{1,2} = \frac{iR}{2L} \pm \sqrt{\frac{1}{LC} - \frac{R^2}{4L^2}}.
\end{equation*}
Die allgemeine Lösung der Differentialgleichung \ref{eq:DGL1} ist
\begin{equation*}
    I(t) = A_1\symup{e}^{i\tilde{ω}_1t} + A_2\symup{e}^{i\tilde{ω}_2t}
\end{equation*}
($A_1, A_2$ komplexe Zahlen).\\
Zur weiteren Rechnung werden die beiden Definitionen
\begin{equation}\label{eq:reff}
    2πμ \coloneqq \frac{R}{2L} \quad \text{und} \quad 2πν \coloneqq \sqrt{\frac{1}{LC} - \frac{R^2}{4L^2}}
\end{equation} verwendet.\\
Die allgemeine Lösung lautet dann
\begin{equation}\label{eq:einh}
    I(t) = \symup{e}^{-2πμt}\left(A_1\symup{e}^{2πiνt} + A_2\symup{e}^{-2πiνt}\right).
\end{equation}
Je nach Eigenschaften der einzelnen Bauteile kann das oszillatorische Verhalten des RCL-Kreises unterschiedlich sein.\\
Es wird unterschieden ob $ν$ reell oder komplex ist.\\
1. Fall:\\ $\frac{1}{LC} > \frac{R^2}{4L^2}$, d.h. $ν$ reell
Damit $I(t)$ reell wird, muss $A_1 = \overline{A}_2$ sein.
Diese Voraussetzung wird durch den Ansatz
\begin{equation*}
    A_1 = \frac{1}{2}A_0\symup{e}^{iη} \quad \text{und} \quad A_2 = \frac{1}{2}A_0\symup{e}^{-iη}
\end{equation*} mit $A_0$ und $η$ reell.\\
Durch anwendung der Eulerschen Formel für Cosinus lässt sich der Ausdruck umformen zu
\begin{equation*}
    I(t) = A_0\symup{e}^{-2πμt}\cos{\left(2πνt + η\right)}.
\end{equation*}
Die Funktion ist rein oszillatorisch und die Amplitude nimmt mit steigendem $t$ exponentiell ab. 
Die Schwinungsdauer kann mit
\begin{equation*}\label{eq:Schwinungsdauer}
    T = \frac{1}{ν} = \frac{2π}{\sqrt{\frac{1}{LC} - \frac{R^2}{4L^2}}}
\end{equation*}
berechnet werden. Wird $\frac{R^2}{4L^2} << \frac{1}{LC},$ so nähert sich $T$ dem Wert
\begin{equation*}
    T_0 = \frac{2π}{ω_0} = 2π\sqrt{LC}.
\end{equation*}
Damit sich die Amplitude auf den e-ten Teil ihrer ursprünglichen Größe absenkt, muss die Zeit
\begin{equation}\label{eq:TexMex}
    T_{\text{ex}} \coloneqq \frac{1}{2πμ} = \frac{2L}{R}
\end{equation}
vergangen sein. Die Geschwindigkeit der Amplitudenabnahme ist durh $2πμ = \frac{R}{2L}$ charakterisiert.\\
\\
2. Fall:\\ $\frac{1}{LC} < \frac{R^2}{4L^2}, \quad$ d.h. $ν$ komplex
Die \autoref{eq:einh} wird durch die Voraussetzung vollständig reell. Die Gleichung hat keinen oszillatorische Anteil mehr.
Dies wird aperiodische Dämpfung genannt. Je nach Wahl der Konstanten $A_1 \text{und} A_2$ kann $I(t)$ direkt monoton gegen Null 
gehen oder zuerst ncoh einen Extremwert erreichen. Der Strom hat maximal einen Nulldurchlauf und es gilt
\begin{equation*}
    I(t) \sim \symup{e}^{-(\frac{R}{2L} - \sqrt{\frac{R^2}{4L^2}})t}.
\end{equation*}
Besonders interessant ist der Spezialfall 
\begin{equation}\label{eq:r_ap}
    \frac{1}{LC} = \frac{R_{\text{ap}}^2}{4L^2} \quad \text{d.h.} \:\: ν = 0,
\end{equation}
denn dann wird
\begin{equation*}
    I(t) = A\symup{e}^{-\frac{R}{2L}t} = A\symup{e}^{-\frac{t}{\sqrt{LC}}}
\end{equation*}
Dieser fall ist der aperiodische Grenzfall. Es geht $I(t)$ schnellstmöglich gegen Null ohne Überschingung.\\
\\
\subsection{Erzwungene Schwingung}
Wird ein Schwingkreis mit einer Wechselspannung
\begin{equation*}
    U(t) = U_0\symup{e}^{iωt}
\end{equation*}
betrieben, nimmt die Differentialgleichung \eqref{eq:DGL1} die Form
%\begin{equation*}
%    L\frac{\symup{d}I(t)}{dt} + RI(t) + \frac{Q(t)}{C} = U_0\symup{e}^{iωt}
%\end{equation*} 
\begin{equation}\label{eq:DGL2}
    LC\frac{\symup{d}^2U_C}{\symup{d}t^2} + RC\frac{\symup{d}U_C}{\symup{d}t} + U_C = U_0\symup{e}^{iωt}
\end{equation}
an.\\
%\begin{equation*}
%    U_C = \frac{Q(t)}{C}
%\end{equation*}
Abhängig von der Kreisfrequenz $ω$ kann zwischen der Amplitude $A$ der Kondensatorspannung und der Erregerspannung $U(t)$ ein Phasenunterschied $φ$ bestehen.\\
Durch einstzen des Ansatzes
\begin{equation}\label{eq:U_C_omega}
    U_C(ω, t) = U(ω) \symup{e}^{iωt}
\end{equation} mit $U(ω)$ komplex, lässt sich die Differentialgleichung \eqref{eq:DGL2} in die Form
\begin{equation*}
    -LCω^2U + iωRCU + U = U_0
\end{equation*} überführen. Nach $U$ aufgelöst ergibt sich 
\begin{equation}\label{eq:U_abs}
    U = \frac{U_0}{1 - LCω^2 + iωRC} = \frac{U_0\left(1 - LCω^2 + iωRC\right)}{\left(1 - LCω^2\right)^2 + ω^2R^2C^2}.
\end{equation}\\
%\begin{equation*}
%    \left\lvert U\right\rvert = \sqrt{\symfrak{Re}^2(U) + \symfrak{Im}^2(U)} = U_0 \sqrt{\frac{\left(1 - LCω^2\right)^2 + ω^2R^2C^2}{\left(\left(1 - LCω^2\right)^2 + ω^2R^2C^2\right)^2}}
%\end{equation*}
%\begin{equation*}
%    \tan{φ(ω)} = \frac{\symfrak{Im}(U)}{\symfrak{Re}(U)} = \frac{-ωRC}{1 - LCω^2}
%\end{equation*}
Über das Verhältnis $\frac{\symfrak{Im}(U)}{\symfrak{Re}(U)}$ lässt sich die Phase 
\begin{equation}\label{eq:phi}
    φ(ω) = \arctan{\left(\frac{-ωRC}{1 - LCω^2}\right)}
\end{equation} berechnen.\\
Nach der Gleichung \eqref{eq:U_C_omega} ist $U_C = |A|.$ Mit \autoref{eq:U_abs} ergibt sich für die Kondensatorspannung
\begin{equation}\label{eq:leckomio}
    U_C(ω) = \frac{U_0}{\sqrt{\left(1 - LCω^2\right)^2 + ω^2R^2C^2}}.
\end{equation}
Die Abhängigkeit der Kondensatorspannung $U_C$ zur Frequenz $ω$ lässt sich mit eine Resonanzkurve darstellen.
Aus der Gleichung \eqref{eq:leckomio} lässt sich erkennen, dass für $ω \longrightarrow 0$ die Kondensatorspannung $U_C$ gegen die Erregeramplitude $U_0$ strebt.
Für $ω \longrightarrow \infty$ geht $U_C$ gegen $0.$\\
Die Resonanzfrequenz $ω_0$ ist die Frequenz, bei der $U_C$ maximal wird. Die Kondensatorspannung kann dabei größer als $U_0$ sein.
Bei genauer Rechnung ergibt sich für die Resonanzfrequenz
\begin{equation}\label{eq:wres}
    ω_{\text{res}} = \sqrt{\frac{1}{LC} - \frac{R^2}{4L^2}}.
\end{equation}\\
Besonders bedeutend ist dabei der Fall der Schwache Dämpfung $\frac{R^2}{2L^2} << \frac{1}{LC}.$\\
Unter dieser Voraussetzung ist $U_C$ um den Faktor $\frac{1}{ω_0RC}$ größer als $U_0.$ Der Faktor heißt auch als Resonanzüberhöhung oder Güte $q$ des Schwingkreises.\\
\begin{equation*}
    U_{C,\text{max}} = \frac{U_0}{ω_0RC} = \frac{U_0}{R}\sqrt{\frac{L}{C}}.
\end{equation*}
Geht $R \longrightarrow 0$ so kann $U_{C,\text{max}} \longrightarrow \infty.$ Dies wird als Resonanzkatastrophe bezeichnet.\\
Die Schärfe der Resonanz ist die Breite der Resonanzkurve. Sie wird charakterisiert durch die Frequenzen $ω_+$ und $ω_-$.
Die beiden Frequenzen können mit der Gleichung
\begin{equation*}
    \frac{U_0}{\sqrt{2}} \frac{1}{ω_0RC} = \frac{U_0}{C\sqrt{ω_{\pm}^2R^2 + \left(ω_{\pm}^2L - \frac{1}{C}\right)^2}}.
\end{equation*}
berechnet werden.\\
Mit $\frac{R^2}{L^2} << ω_0^2$ ist die Breite der Resonanzkurve gegeben mit
\begin{equation}\label{eq:gute}
    ω_+ - ω_- \approx \frac{R}{L}.
\end{equation}
Es lässt sich folgende Beziehung zwischen der Breite und der Güte der Resonanzkurve aufstellen
\begin{equation*}
    q = \frac{ω_0}{ω_+ - ω_-}.
\end{equation*}\\
Anders lässt sich die Güte noch berechnen mit
\begin{equation}\label{eq:lol}
    q = \frac{1}{\omega_0 R C}.
\end{equation}
Dabei ist $\omega_0$ die Eigenfrequenz und lässt sich beim RLC-Schwingkreis mit $\omega_0 = \sqrt{\frac{1}{LC}}$ berechnen.
\\
Im Falle einer Starken Dämpfung $\frac{R^2}{2L^2} >> \frac{1}{LC}$ ist die Resonanzüberhöhung nicht mehr existent.
Mit zunehmender Frequenz geht $U_C$ von $U_0$ aus monoton gegen $0.$\\
\\
Die Phase $φ$ wird wie in \eqref{eq:phi} berechnet. Bei einer Frequenz von $ω_0^2 = \frac{1}{LC}$ ist $φ = -\frac{\pi}{2}.$\\
Für die Frequenzen $ω_1$ und $ω_2$ für die $φ$ die Werte $\frac{\pi}{4}$ und $\frac{3\pi}{4}$ annimmt, gilt
\begin{equation}\label{eq:w12}
    ω_{1,2} = \pm \frac{R}{2L} + \sqrt{\frac{R^2}{4L^2} + \frac{1}{LC}}
\end{equation}
woraus sich
\begin{equation*}
    ω_1 - ω_2 = \frac{R}{L}
\end{equation*}
ergibt. Durch Vergleich mit \eqref{eq:gute} wird erkennbar, dass bei Schwacher Dämpfung $ω_1 - ω_2$ zu $ω_+ - ω_-$ wird.

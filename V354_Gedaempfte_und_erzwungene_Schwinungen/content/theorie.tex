\section{Zielsetzung}
\label{sec:Zielsetzung}

Im Versuch V354 soll die Dämpfung eines RLC-Serienschwingkreises ermittelt werden. 
Außerdem soll die Frequenzabhängigkeit der Kondensatorspannung, sowie die Phasenverschiebung zwischen Kondensator- und Erregerspannung bestimmt werden.

\section{Theorie}
\label{sec:Theorie}

\subsection{Gedämpfte Schwingung}
\label{sec:Gedaempfte_Schwingung}

Ein RCL-Schwingkreis besteht aus zwei Energiespeichern, einem Kondensator und einer Spule, und einem Widerstand. 
Das schwingende Verhalten des Stromflusses und der Spannung entsteht durch die beiden Speicher. Durch den Ohmschen Widerstand
ist der Schwingkreis gedämpft. Die Dämpfung wird durch den Widerstand bestimmt. Ein ungedämpfter Schwingkreis 
ist nur theoretisch möglich, da die Verbindungskabel und die Spule selbst einen Widerstand aufweisen.
Durch den Widerstand wird Energie irreversibel in Wärme umgewandelt. Bei einmaliger Anregung geht die Amplitude der Schwingung
mit der Zeit gegen Null.
Nach dem zweiten Kirchhoffschnen Gesetz setzt sich die Gesamtspannung im Kreis zu
\begin{equation*}
    U_R(t) + U_C(t) + U_L(t) = 0
\end{equation*}
zusammen. 
Die einzelnen Spannungen lassen sich als
\begin{equation*}
    U_R(t) = RI(t) \quad \text{und} \quad U_C(t) = \frac{Q(t)}{C}  \quad \text{und} \quad U_L(t) = L\frac{\symup{d}I(t)}{dt}
\end{equation*}
ausdrücken. 
Durch einsezten der Spannungen in die Gesamtspannung ergibt sich 
\begin{equation*}
    L\frac{\symup{d}I(t)}{dt} + RI(t) + \frac{Q(t)}{C} = 0.
\end{equation*}
Abgeleitet nach $t$ ergibt sich mit $I = \frac{\symup{d}Q}{dt}$ die Gleichung
\begin{equation*}
    \frac{\symup{d}^2I}{dt^2} + \frac{R}{L}\frac{\symup{d}I}{dt} + \frac{1}{LC}I = 0.
\end{equation*}

\begin{equation*}
    I(t) = A\symup{e}^{i\tilde{ω}t}
\end{equation*}
mit $i = \sqrt{-1}$ und $\tilde{ω}, A$ komplexe Zahlen.
\begin{equation*}
    \left(-\tilde{ω}^2 + \frac{iR}{L}\tilde{ω} + \frac{1}{LC}\right)A\symup{e}^{i\tilde{ω}t}
\end{equation*}
\begin{equation*}
    -\tilde{ω}^2 + \frac{iR}{L}\tilde{ω} - \frac{1}{LC} = 0
\end{equation*}
\begin{equation*}
    \tilde{ω}_{1,2} = \frac{iR}{2L} \pm \sqrt{\frac{1}{LC} - \frac{R^2}{4L^2}}
\end{equation*}
\begin{equation*}
    I(t) = A_1\symup{e}^{i\tilde{ω}_1t} + A_2\symup{e}^{i\tilde{ω}_2t}
\end{equation*}
($A_1, A_2$ komplexe Zahlen)
\begin{equation*}
    2πμ \coloneqq \frac{R}{2L} \quad \text{und} \quad 2πν \coloneqq \sqrt{\frac{1}{LC} - \frac{R^2}{4L^2}}
\end{equation*}
\begin{equation*}
    I(t) = \symup{e}^{-2πμt}\left(A_1\symup{e}^{2πiνt} + A_2\symup{e}^{-2πiνt}\right).
\end{equation*}
1. Fall:\\ $\frac{1}{LC} > \frac{R^2}{4L^2}$, d.h. $ν$ reell
Damit $I(t)$ reell wird, muss $A_1 = \overline{A}_2$ sein.
Diese Voraussetzung wird durch den Ansatz
\begin{equation*}
    A_1 = \frac{1}{2}A_0\symup{e}^{iη} \quad \text{und} \quad A_2 = \frac{1}{2}A_0\symup{e}^{-iη}
\end{equation*} mit $A_0$ und $η$ reell.\\
\begin{equation*}
    I(t) = A_0\symup{e}^{-2πμt}\cos{\left(2πνt + η\right)}.
\end{equation*}
\begin{equation*}
    T = \frac{1}{ν} = \frac{2π}{\sqrt{\frac{1}{LC} - \frac{R^2}{4L^2}}}
\end{equation*}
\begin{equation*}
    T_0 = \frac{2π}{ω_0} = 2π\sqrt{LC}
\end{equation*}
\begin{equation}\label{eq:TexMex}
    T_{\text{ex}} \coloneqq \frac{1}{2πμ} = \frac{2L}{R}
\end{equation}
2. Fall: $\frac{1}{LC} < \frac{R^2}{4L^2}$, d.h. $ν$ komplex
\begin{equation*}
    I(t) \approx \text{oder lieber} \sim ? \symup{e}^{-(\frac{R}{2L} - \sqrt{\frac{R^2}{4L^2}})t}
\end{equation*}
Spezialfall:\\ $\frac{1}{LC} = \frac{R_{\text{ap}}^2}{4L^2},$ d.h. $ν = 0$
\begin{equation*}
    I(t) = A\symup{e}^{-\frac{R}{2L}t} = A\symup{e}^{-\frac{t}{\sqrt{LC}}}
\end{equation*}
\\
\subsection{Erzwungene Schwingung}
\begin{equation*}
    U(t) = U_0\symup{e}^{iωt}.
\end{equation*}
\begin{equation*}
    L\frac{\symup{d}I(t)}{dt} + RI(t) + \frac{Q(t)}{C} = U_0\symup{e}^{iωt}
\end{equation*}
\begin{equation*}
    LC\frac{\symup{d}^2U_C}{\symup{d}t^2} + RC\frac{\symup{d}U_C}{\symup{d}t} + U_C = U_0\symup{e}^{iωt}
\end{equation*}
\begin{equation*}
    U_C = \frac{Q(t)}{C}
\end{equation*}
\begin{equation*}
    U_C(ω, t) = U(ω) \symup{e}^{iωt}
\end{equation*} mit $U(ω)$ komplex.\\
\begin{equation*}
    -LCω^2U + iωRCU + U = U_0
\end{equation*}
\begin{equation*}
    U = \frac{U_0}{1 - LCω^2 + iωRC} = \frac{U_0\left(1 - LCω^2 + iωRC\right)}{\left(1 - LCω^2\right)^2 + ω^2R^2C^2}.
\end{equation*}
\begin{equation*}
    \left\lvert U\right\rvert = \sqrt{\symfrak{Re}^2(U) + \symfrak{Im}^2(U)} = U_0 \sqrt{\frac{\left(1 - LCω^2\right)^2 + ω^2R^2C^2}{\left(\left(1 - LCω^2\right)^2 + ω^2R^2C^2\right)^2}}
\end{equation*}
\begin{equation*}
    \tan{φ(ω)} = \frac{\symfrak{Im}(U)}{\symfrak{Re}(U)} = \frac{-ωRC}{1 - LCω^2}
\end{equation*}
\begin{equation*}
    φ(ω) = \arctan{\left(\frac{-ωRC}{1 - LCω^2}\right)}
\end{equation*}
\begin{equation*}
    U_C(ω) = \frac{U_0}{\sqrt{\left(1 - LCω^2\right)^2 + ω^2R^2C^2}}
\end{equation*}
\begin{equation*}
    ω_{\text{res}} = \sqrt{\frac{1}{LC} - \frac{R^2}{4L^2}}
\end{equation*}
Schwache Dämpfung:\\ $\frac{R^2}{2L^2} << \frac{1}{LC}$
\begin{equation*}
    U_{C,\text{max}} = \frac{U_0}{ω_0RC} = \frac{U_0}{R}\sqrt{\frac{L}{C}}.
\end{equation*}
\begin{equation*}
    \frac{U_0}{\sqrt{2}} \frac{1}{ω_0RC} = \frac{U_0}{C\sqrt{ω_{\pm}^2R^2 + \left(ω_{\pm}^2L - \frac{1}{C}\right)^2}}.
\end{equation*}
\begin{equation*}
    ω_+ - ω_- \approx \frac{R}{L}.
\end{equation*}
\begin{equation*}
    q = \frac{ω_0}{ω_+ - ω_-}.
\end{equation*}
Starke Dämpfung:\\ $\frac{R^2}{2L^2} >> \frac{1}{LC}$
$ω_0^2 = \frac{1}{LC}$
\begin{equation*}
    ω_{1,2} = \pm \frac{R}{2L} + \sqrt{\frac{R^2}{4L^2} + \frac{1}{LC}}
\end{equation*}
$ω_1 - ω_2 = \frac{R}{L}$

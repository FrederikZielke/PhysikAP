\section{Auswertung}
\label{sec:Auswertung}
Die in \autoref{sec:Auswertung} verwendeten Werte für die Kapazität $C$, Induktivität $L$ und Widerstand $R$ sind:
\begin{align*}
  L &= \SI{16.87(0.05)}{\micro\henry} \\
  R_1 &= \SI{67.2(0.1)}{\ohm}\\
  R_2 &= \SI{682(0.5)}{\ohm}\\
  C &= \SI{2.060(0.003)}{\nano\farad}.
\end{align*}
Die in \autoref{sec:Auswertung} gezeigten Grafiken und Ausgleichsrechnungen sind mithilfe der Python-Bibliotheken Matplotlib \cite{matplotlib}, Scipy \cite{scipy} und Numpy \cite{numpy}
erstellt worden.
\subsection{Abklingverhalten des RLC-Kreises}
\label{sec:Abklingverhalten}
\begin{table}[H]
  \begin{minipage}{0.48\linewidth}
    \centering
    \caption{\\Zeit $t$ und zugehörige\\ Kondensatorspannung $U_C$ \\der oberen Einhüllenden.}
    \begin{tabular}{c|c}
      \toprule
      {$U_C\left[\unit{\s}\right]$} & {$t\left[\unit{\s}\right]$}\\
      \midrule
      1.50 & 0.0 \\
      1.30 & 0.4 \\
      1.20 & 0.8 \\ 
      0.90 & 1.2 \\
      0.80 & 1.6 \\
      0.70 & 2.0 \\
      0.60 & 2.4 \\
      0.50 & 2.8 \\
      0.45 & 3.2 \\
      0.40 & 3.6 \\
      \bottomrule
    \end{tabular}
    \vspace{5pt}
    \label{tab:obere}
  \end{minipage}
  \begin{minipage}{0.48\linewidth}
    \centering
    \caption{\\Zeit $t$ und zugehörige\\ Kondensatorspannung $U_C$ \\der unteren Einhüllenden.}
    \begin{tabular}{c|c}
      \toprule
      {$U_C\left[\unit{\s}\right]$} & {$t\left[\unit{\s}\right]$}\\
      \midrule
      -1.40 & 0.2 \\
      -1.20 & 0.6 \\
      -1.00 & 1.0 \\
      -0.85 & 1.4 \\
      -0.70 & 1.8 \\
      -0.60 & 2.2 \\
      -0.55 & 2.6 \\
      -0.50 & 3.0 \\
      -0.40 & 3.4 \\
      -0.35 & 3.8 \\
      \bottomrule
    \end{tabular}
    \vspace{5pt}
    \label{tab:untere}
  \end{minipage}
\end{table}

Da die Spannung $U_C$ proportional zum Strom $I_C$ ist, ist die Form der Einhüllenden nach \autoref{eq:einh} gegeben:
\begin{equation}
  A = A_0 \exp(-2 \pi \mu t).
\end{equation}

















\newpage
Aufgabenteil b:\\
Aperiodischer Grenzfall bei $R = \SI{4.56}{\kilo\ohm}$

\section{Auswertung}
\label{sec:Auswertung}
Die in \autoref{sec:Auswertung} verwendeten Werte für die Kapazität $C$, Induktivität $L$ und Widerstand $R$ sind:
\begin{align*}
  L &= \SI{16.87(0.05)}{\milli\henry} \\
  R_1 &= \SI{67.2(0.1)}{\ohm}\\
  R_2 &= \SI{682(0.5)}{\ohm}\\
  C &= \SI{2.060(0.003)}{\nano\farad}.
\end{align*}
Die in \autoref{sec:Auswertung} gezeigten Grafiken und Ausgleichsrechnungen sind mithilfe der Python-Bibliotheken Matplotlib \cite{matplotlib}, Scipy \cite{scipy} und Numpy \cite{numpy}
erstellt worden.
\subsection{Abklingverhalten des RLC-Kreises}
\label{sec:Abklingverhalten}
\begin{table}[H]
  \begin{minipage}{0.48\linewidth}
    \centering
    \caption{\\Zeit $t$ und zugehörige\\ Kondensatorspannung $U_C$ \\der oberen Einhüllenden.}
    \begin{tabular}{c|c}
      \toprule
      {$U_C\left[\unit{\s}\right]$} & {$t\left[\unit{\s}\right]$}\\
      \midrule
      1.50 & 0.0 \\
      1.30 & 0.4 \\
      1.20 & 0.8 \\ 
      0.90 & 1.2 \\
      0.80 & 1.6 \\
      0.70 & 2.0 \\
      0.60 & 2.4 \\
      0.50 & 2.8 \\
      0.45 & 3.2 \\
      0.40 & 3.6 \\
      \bottomrule
    \end{tabular}
    \vspace{5pt}
    \label{tab:obere}
  \end{minipage}
  \begin{minipage}{0.48\linewidth}
    \centering
    \caption{\\Zeit $t$ und zugehörige\\ Kondensatorspannung $U_C$ \\der unteren Einhüllenden.}
    \begin{tabular}{c|c}
      \toprule
      {$U_C\left[\unit{\s}\right]$} & {$t\left[\unit{\s}\right]$}\\
      \midrule
      -1.40 & 0.2 \\
      -1.20 & 0.6 \\
      -1.00 & 1.0 \\
      -0.85 & 1.4 \\
      -0.70 & 1.8 \\
      -0.60 & 2.2 \\
      -0.55 & 2.6 \\
      -0.50 & 3.0 \\
      -0.40 & 3.4 \\
      -0.35 & 3.8 \\
      \bottomrule
    \end{tabular}
    \vspace{5pt}
    \label{tab:untere}
  \end{minipage}
\end{table}

Da die Spannung $U_C$ proportional zum Strom $I_C$ ist, ist die Form der Einhüllenden nach \autoref{eq:einh} gegeben:
\begin{equation*}
  A = A_0 \exp(-2 \pi \mu t).
\end{equation*}
Durch die Ausgleichsrechnung ergeben sich für die Fitparameter
\begin{align}
  A_0 &= \SI{1.52(0.02)}{V} \\
  \text{und } \mu &= \SI{385.55(14.34)}{\frac{1}{s}}.
  \label{eq:fit1}
\end{align}
Die Messwerte $t$ und $U_C$ sind mit der Ausgleichsfunktion für die oszillierende Spannung in \autoref{fig:einh} dargestellt. Ebenfalls ist dort die Einhüllende
geplottet.
Für die Ausgleichsfunktion der oszillierenden Spannung wurde die Fitfunktion
\begin{equation*}
  F(t) = U_0 \cdot \sin(bt+c) \exp(-ta)
\end{equation*}
verwendet. Dabei ist $\sin(bt+c)$ der Schwingungsanteil und $\exp(-ta)$ die Einhüllende. 
Die Ausgleichsrechnung ergibt die Parameter
\begin{align*}
  a &= \SI{3.87e02}{\frac{1}{s}},\\
  b &= \SI{-1.57e04}{\frac{1}{s}},\\
  \text{und } c &= \SI{1.57}.
\end{align*}
\begin{figure}[H]
  \includegraphics[width=\linewidth]{build/abklingverhalten.pdf}
  \caption{Abklingkurve des RLC-Kreises.}
  \label{fig:einh}
\end{figure}
Mit den Werten aus \autoref{eq:fit1} lässt sich nun $R_{eff}$ und $T_{ex}$ berechnen. Mit \autoref{eq:TexMex} und \autoref{eq:reff} folgt mit 
$\mu = \SI{385.55(14.34)}{\frac{1}{s}}$ und $L = \SI{16.87(0.05)}{\milli\henry}$:
\begin{align*}
  T_{ex} &= 1/(2\pi\mu)                      = \SI{0.000413(0.000015)}{s}\\
  \text{und } R_{eff} &= 2 \cdot \mu \cdot L \; = \SI{13.0(0.5)}{\ohm}.\\
\end{align*}
Ein theoretischer Wert für die Abklingzeit ergibt sich mit 
\begin{equation*}
  T_{theorie} = \frac{2L}{R_1} = \SI{0.000502(0.000002)}{s}.
\end{equation*}
Die Abweichung vom verwendeten Widerstand $R_1$ beträgt also $\SI{52.4}{\ohm}$.

\subsection{Widerstand aperiodischer Grenzfall}\label{sec:ap}
Aus \autoref{eq:r_ap} lässt sich ein theoretischer Wert für den Widerstand $R_{ap}$ berechnen, bei dem der aperiodische Grenzfall eintritt.
Dieser ergibt sich zu
\begin{equation*}
  R_{ap} = \sqrt{\frac{4L}{C}} = \SI{5723(9)}{\ohm}.
\end{equation*}
Der gemessene reale Widerstand, bei dem der aperiodische Grenzfall eingetreten ist, beträgt jedoch
\begin{equation*}
  R_{real} = \SI{4580}{\ohm}.
\end{equation*}
Damit ergibt sich eine Abweichung von
\begin{equation*}
  \Delta_R = |\frac{R_{real}}{R_{ap}} - 1| = 19.97\%.
\end{equation*}

\subsection{Kondensatorspannung in Abhängigkeit der Frequenz}
In \autoref{tab:mess2} befinden sich die Messwerte.
\begin{table}[H]
  \caption{Frequenz $f$ des Signals, abgelesene Kondensatorspannung $U_C$, zeitlicher Abstand der Nulldurchgänge $a$ und Intervalllänge $b$.}
  \label{tab:mess2}
  \begin{minipage}{0.48\linewidth}
    \centering
    \begin{tabular}{
      S[table-format=2.2]
      S[table-format=2.2]
      S[table-format=1.2]
      c
    }
      \toprule
      {$f\left[\unit{kHz}\right]$} & {$U_C\left[\unit{V}\right]$} & {$a\left[\unit{ms}\right]$} & {$b\left[\unit{ms}\right]$}\\
      \midrule
      05.00 & 05.0 & 0.00 & 0.75 \\
      10.00 & 06.0 & 0.02 & 0.40 \\
      12.50 & 06.5 & 0.02 & - \\    
      15.00 & 07.0 & 0.02 & 0.26 \\
      17.50 & 08.0 & 0.02 & - \\
      19.00 & 09.5 & 0.02 & - \\
      20.00 & 10.4 & 0.03 & 0.19 \\
      20.50 & 11.0 & 0.03 & - \\
      21.00 & 11.5 & 0.03 & - \\
      21.50 & 12.0 & 0.04 & - \\
      22.00 & 13.0 & 0.04 & 0.175 \\
      22.25 & 13.4 & 0.04 & - \\
      22.50 & 13.9 & 0.04 & - \\
      22.75 & 14.0 & 0.04 & - \\
      22.90 & 14.5 & 0.04 & - \\
      23.00 & 14.7 & 0.04 & 0.17 \\
      23.10 & 14.9 & 0.04 & - \\
      23.20 & 15.0 & 0.04 & - \\
      23.30 & 15.2 & 0.04 & - \\
      23.40 & 15.3 & 0.04 & - \\
      23.60 & 15.8 & 0.04 & - \\
      23.80 & 16.0 & 0.05 & - \\
      24.00 & 16.2 & 0.05 & - \\
      24.20 & 16.8 & 0.05 & 0.16 \\
      24.40 & 17.0 & 0.05 & - \\
      24.60 & 17.4 & 0.05 & - \\
      24.80 & 17.5 & 0.06 & - \\
      25.00 & 17.9 & 0.06 & 0.15 \\
      25.20 & 18.0 & 0.06 & - \\
      25.40 & 18.1 & 0.06 & - \\
      25.60 & 18.2 & 0.06 & 0.15 \\
      25.80 & 18.3 & 0.06 & - \\
      26.00 & 18.2 & 0.07 & - \\
      26.20 & 18.2 & 0.07 & - \\
      26.40 & 18.1 & 0.08 & - \\
      26.60 & 18.0 & 0.08 & 0.14 \\
      \bottomrule
    \end{tabular}
    \vspace{5pt}
  \end{minipage}
  \begin{minipage}{0.48\linewidth}
    \centering
    \begin{tabular}{
      S[table-format=2.2]
      S[table-format=2.2]
      S[table-format=1.2]
      c
    }
      \toprule
      {$f\left[\unit{kHz}\right]$} & {$U_C\left[\unit{V}\right]$} & {$a\left[\unit{ms}\right]$} & {$b\left[\unit{ms}\right]$}\\
      \midrule
      27.00 & 17.4 & 0.08 & - \\
      27.20 & 17.1 & 0.08 & - \\
      27.40 & 16.7 & 0.08 & - \\
      27.60 & 16.3 & 0.08 & - \\
      27.80 & 16.0 & 0.09 & - \\
      28.00 & 15.5 & 0.09 & - \\
      28.20 & 15.1 & 0.09 & - \\
      28.40 & 14.8 & 0.09 & - \\
      28.60 & 14.1 & 0.10 & - \\
      28.80 & 14.0 & 0.10 & - \\
      29.00 & 13.5 & 0.10 & - \\
      29.20 & 13.0 & 0.10 & - \\
      29.40 & 12.6 & 0.10 & - \\
      29.60 & 12.1 & 0.10 & - \\
      29.80 & 11.9 & 0.10 & - \\
      30.00 & 11.5 & 0.10 & 0.13 \\
      30.30 & 11.0 & 0.10 & - \\
      30.60 & 10.4 & 0.10 & - \\
      31.00 & 09.9 & 0.10 & - \\
      31.50 & 09.2 & 0.10 & - \\
      32.00 & 08.5 & 0.10 & - \\
      32.50 & 08.0 & 0.10 & - \\
      33.00 & 07.5 & 0.10 & - \\
      33.50 & 07.0 & 0.10 & - \\
      34.00 & 06.5 & 0.10 & - \\
      34.50 & 06.1 & 0.10 & - \\
      35.00 & 06.0 & 0.10 & 0.13 \\
      36.00 & 05.0 & 0.10 & - \\
      38.00 & 04.4 & 0.10 & - \\
      40.00 & 03.2 & 0.10 & - \\
      45.00 & 02.1 & 0.08 & - \\
      50.00 & 01.5 & 0.08 & - \\
      55.00 & 01.1 & 0.08 & - \\
      60.00 & 00.9 & 0.06 & - \\
      65.00 & 00.6 & 0.06 & - \\
      70.00 & 00.4 & 0.06 & - \\
      \bottomrule
    \end{tabular}
    \vspace{5pt}
  \end{minipage}
\end{table}
Wird das Amplitudenverhältnis $\frac{U_C}{U_0}$ gegen die Frequenz $f$ aufgetragen, so entsteht der Graph in \autoref{fig:res}. Dabei beträgt $U_0 = \SI{1.5}{V}$.
Um die Breite der Resonanzkurve $f_2 - f_1$ und die Resonanzüberhöhung $q$ zu bestimmen, wird der Bereich um die Resonanzfrequenz $f_{res}$ in \autoref{fig:res} linear dargestellt.
Der Abstand des kritischen Werts $\frac{1}{\sqrt{2}}\frac{U_C}{U_0}$ bis zum Maximum der Spannung $U_{max} = \SI{18.3}{V}$ ist die Resonanzüberhöhung.
Die Breite der Messwerte über diesem kritischen Wert wird auch Breite der Resonanzkurve genannt. 
\begin{figure}[H]
  \includegraphics[width=\linewidth]{build/5c.pdf}
  \caption{Amplitudenverhältnis $\frac{U_C}{U_0}$ in Abhängigkeit der Frequenz $f$.}
  \label{fig:res}
\end{figure}
Aus \autoref{fig:res} lässt sich somit
\begin{align*}
  q_{exp} &= \frac{18.3}{1.5}, \\
  f_1 &= \SI{22.01}{kHz}, \\
  \text{und } f_2 &= \SI{29.22}{kHz}
\end{align*}
Mit \autoref{eq:lol} folgt als theoretischer Wert für die Güte
\begin{equation*}
  q_{theo} = \SI{4.196(0.008)}.
\end{equation*}
bestimmen. Für die Breite der Resonanzkurve folgt somit
\begin{equation*}
  \Delta_f = f_2 - f_1 = \SI{7.20}{kHz}.
\end{equation*}
\newpage
Die theoretischen Werte für $f_{1, theorie}$ und $f_{2, theorie}$ folgen aus \autoref{eq:w12} mit $R_2$ und $f = \frac{\omega}{2\pi}$:
\begin{align*}
  f_1 = \SI{23.97(0.04)}{kHz},\\
  f_2 = \SI{30.41(0.05)}{kHz}.
\end{align*}
Daraus folgt für die theoretische Breite $\Delta_{theorie} = \SI{6.434(0.020)}{kHz}$.
Aus \autoref{tab:mess2} lässt sich die Resonanzfrequenz ablesen. Diese beträgt
\begin{equation*}
  f_{res} = \SI{25.8}{kHz}.
\end{equation*}
Der theoretische Wert berechnet sich nach \autoref{eq:wres} mit $R_2$ zu 
\begin{equation*}
  f_{theorie} = \sqrt{\frac{1}{LC} - \frac{R^2}{2L^2}} \cdot \frac{1}{2\pi} = \SI{26.61(0.04)}{kHz}.
\end{equation*}
Die Abweichung von der Theorie beträgt somit 
\begin{equation*}
  \Delta_{res} = |\frac{f_{res}}{f_{theorie}} - 1| = 3\%.
\end{equation*}

Mit 
\begin{equation*}
  \phi = 2 \cdot \pi \cdot \frac{a}{b}
\end{equation*}
lässt sich die Phasenverschiebung zwischen der Kondensator- und Erregerspannung berechnen.
Wird $\phi$ gegen die Frequenz $f$ aufgetragen, so entsteht der Graph in \autoref{fig:phase1}.
\begin{figure}[H]
  \includegraphics[width=\linewidth]{build/5d.pdf}
  \caption{Phasenverschiebung $\phi$ in Abhängigkeit der Frequenz $f$.}
  \label{fig:phase1}
\end{figure}
Der Bereich um die Resonanzfrequenz $f_{res}$ ist in \autoref{fig:phase2} linear dargestellt. Die Resonanzfrequenz befindet sich an der senkrechten
mittleren Linie, $f_1$ an der senkrechten linken Linie und $f_2$ an der senkrechten rechten Linie.
\begin{figure}[H]
  \includegraphics[width=\linewidth]{build/5d2.pdf}
  \caption{Phasenverschiebung $\phi$ in Abhängigkeit der Frequenz $f$ und lineare Darstellung im Bereich der Resonanzfrequenz.}
  \label{fig:phase2}
\end{figure}







\newpage

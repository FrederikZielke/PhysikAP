\section{Auswertung}
\label{sec:Auswertung}
Die in \autoref{sec:Auswertung} verwendeten Werte für die Kapazität $C$, Induktivität $L$ und Widerstand $R$ sind:
\begin{align*}
  L &= \SI{16.87(0.05)}{\milli\henry} \\
  R_1 &= \SI{67.2(0.1)}{\ohm}\\
  R_2 &= \SI{682(0.5)}{\ohm}\\
  C &= \SI{2.060(0.003)}{\nano\farad}.
\end{align*}
Die in \autoref{sec:Auswertung} gezeigten Grafiken und Ausgleichsrechnungen sind mithilfe der Python-Bibliotheken Matplotlib \cite{matplotlib}, Scipy \cite{scipy} und Numpy \cite{numpy}
erstellt worden.
\subsection{Abklingverhalten des RLC-Kreises}
\label{sec:Abklingverhalten}
\begin{table}[H]
  \begin{minipage}{0.48\linewidth}
    \centering
    \caption{\\Zeit $t$ und zugehörige\\ Kondensatorspannung $U_C$ \\der oberen Einhüllenden.}
    \begin{tabular}{c|c}
      \toprule
      {$U_C\left[\unit{\s}\right]$} & {$t\left[\unit{\s}\right]$}\\
      \midrule
      1.50 & 0.0 \\
      1.30 & 0.4 \\
      1.20 & 0.8 \\ 
      0.90 & 1.2 \\
      0.80 & 1.6 \\
      0.70 & 2.0 \\
      0.60 & 2.4 \\
      0.50 & 2.8 \\
      0.45 & 3.2 \\
      0.40 & 3.6 \\
      \bottomrule
    \end{tabular}
    \vspace{5pt}
    \label{tab:obere}
  \end{minipage}
  \begin{minipage}{0.48\linewidth}
    \centering
    \caption{\\Zeit $t$ und zugehörige\\ Kondensatorspannung $U_C$ \\der unteren Einhüllenden.}
    \begin{tabular}{c|c}
      \toprule
      {$U_C\left[\unit{\s}\right]$} & {$t\left[\unit{\s}\right]$}\\
      \midrule
      -1.40 & 0.2 \\
      -1.20 & 0.6 \\
      -1.00 & 1.0 \\
      -0.85 & 1.4 \\
      -0.70 & 1.8 \\
      -0.60 & 2.2 \\
      -0.55 & 2.6 \\
      -0.50 & 3.0 \\
      -0.40 & 3.4 \\
      -0.35 & 3.8 \\
      \bottomrule
    \end{tabular}
    \vspace{5pt}
    \label{tab:untere}
  \end{minipage}
\end{table}

Da die Spannung $U_C$ proportional zum Strom $I_C$ ist, ist die Form der Einhüllenden nach \autoref{eq:einh} gegeben:
\begin{equation*}
  A = A_0 \exp(-2 \pi \mu t).
\end{equation*}
Durch die Ausgleichsrechnung ergeben sich für die Fitparameter
\begin{align}
  A_0 &= \SI{1.52(0.02)}{V} \\
  \text{und } \mu &= \SI{385.55(14.34)}{\frac{1}{s}}.
  \label{eq:fit1}
\end{align}
Die Messwerte $t$ und $U_C$ sind mit der Ausgleichsfunktion für die oszillierende Spannung in \autoref{fig:einh} dargestellt. Ebenfalls ist dort die Einhüllende
geplottet.
Für die Ausgleichsfunktion der oszillierenden Spannung wurde die Fitfunktion
\begin{equation*}
  F(t) = U_0 \cdot \sin(bt+c) \exp(-ta)
\end{equation*}
verwendet. Dabei ist $\sin(bt+c)$ der Schwingungsanteil und $\exp(-ta)$ die Einhüllende. 
Die Ausgleichsrechnung ergibt die Parameter
\begin{align*}
  a &= \SI{3.87e02}{\frac{1}{s}},\\
  b &= \SI{-1.57e04}{\frac{1}{s}},\\
  \text{und } c &= \SI{1.57}.
\end{align*}
\begin{figure}[H]
  \includegraphics[width=\linewidth]{build/abklingverhalten.pdf}
  \caption{Abklingkurve des RLC-Kreises.}
  \label{fig:einh}
\end{figure}
Mit den Werten aus \autoref{eq:fit1} lässt sich nun $R_{eff}$ und $T_{ex}$ berechnen. Mit \autoref{eq:TexMex} und \autoref{eq:reff} folgt mit 
$\mu = \SI{385.55(14.34)}{\frac{1}{s}}$ und $L = \SI{16.87(0.05)}{\milli\henry}$:
\begin{align*}
  T_{ex} &= 1/(2\pi\mu)                      = \SI{0.000413(0.000015)}{s}\\
  \text{und } R_{eff} &= 2 \cdot \mu \cdot L \; = \SI{13.0(0.5)}{\ohm}.\\
\end{align*}
Die Abweichung vom verwendeten Widerstand $R1$ beträgt also $\SI{52.4}{\ohm}$.

\subsection{Widerstand aperiodischer Grenzfall}
Aus \autoref{eq:r_ap} lässt sich ein theoretischer Wert für den Widerstand $R_{ap}$ berechnen, bei dem der aperiodische Grenzfall eintritt.
Dieser ergibt sich zu
\begin{equation*}
  R_{ap} = \sqrt{\frac{4L}{C}} = \SI{5723(9)}{\ohm}.
\end{equation*}
Der gemessene reale Widerstand, bei dem der aperiodische Grenzfall eingetreten ist, beträgt jedoch
\begin{equation*}
  R_{real} = \SI{4580}{\ohm}.
\end{equation*}
Damit ergibt sich eine Abweichung von
\begin{equation*}
  \Delta_R = |\frac{R_{ap}}{R_{real}} - 1| = 24.97\%.
\end{equation*}












\newpage
Aufgabenteil b:\\
Aperiodischer Grenzfall bei $R = \SI{4.56}{\kilo\ohm}$

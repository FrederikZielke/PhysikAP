\section{Diskussion}
\label{sec:Diskussion}
Beim Abklingverhalten des gedämpften Schwingkreises ergibt sich die Dämpfung (\autoref{eq:fit1}) 
$\mu = \SI{385.55(14.34)}{\frac{1}{s}}$. Diese erscheint etwas klein und dieser Verdacht
bestätigt sich auch bei der Berechnung des effektiven Widerstands, da dieser kleiner mit
$\SI{13.0(0.5)}{\ohm}$ ist als der geschaltete Widerstand $R_1 = \SI{67.2}{\ohm}$.
Der effektive Widerstand sollte größer sein als der geschaltete, da sich zu diesem noch
Innenwiderstände addieren (z.B. Generatorinnenwiderstand). Die Abweichung lässt sich nur mit
systematischen Fehlern begründen. Eventuell war das Fine-Tuning nicht konstant eingestellt
oder das Messintervall wurde zu groß gewählt, wodurch der Abfall der E-Funktion zu flach ist.
\\
\\
Die Abweichung des Widerstands $R_{ap}$ von dem theoretischen Widerstand beim aperiodischen
Grenzfall beträgt 24.97\%. Diese Abweichung lässt sich teils auf den Generatorinnenwiderstand
zurückführen. Ebenfalls war der eingestellte Widerstand weder fein justierbar noch genau ablesbar, 
wodurch Messfehler entstehen.
\\
\\
Die experimentellen Ergebnisse für die Resonanzüberhöhung $q$, die Breite der Resonanzkurve 
$\Delta_f$, die Resonanzfrequenz $f_{res}$, sowie $f_1$ und $f_2$ liegen alle im Toleranzbereich
der theoretisch errechneten Werte \autoref{sec:ap}.
\\
\\
Die Phasenverschiebung bei $f_{res}$ von $\frac{\pi}{2}$ kann nicht aus dem Diagramm entnommen werden.
Generell muss die Phasenverschiebung um den Faktor 2 geteilt werden, damit die erwarteten 
Ergebnisse herauskommen. In diesem Fall könnte eine falsche Einstellung am Oszilloskop
die Erklärung für die Abweichung um einen konstanten Faktor sein.
Der um den Faktor $\frac{1}{2}$ bereinigte Graph ist in \autoref{fig:troll} dargestellt.
\begin{figure}[H]
    \includegraphics[width=\linewidth]{build/diskussion.pdf}
    \caption{Phasenverschiebung $\frac{1}{2}\phi$ in Abhängigkeit der Frequenz $f$ und lineare Darstellung im Bereich der Resonanzfrequenz.}
    \label{fig:troll}
  \end{figure}
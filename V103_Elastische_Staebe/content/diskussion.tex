\section{Diskussion}
\label{sec:Diskussion}

Um die Ergebnisse zu bewerten wird ermittelt aus welchem Metall die Stäbe bestehen. Durch einsetzen des Volumens sowie der in \autoref{sec:Auswertung}
aufgenommenen Gewichte der Stäbe ergibt sich mit $ρ = \frac{m}{V}$ die Dichte. Die Dichte des eckigen Stabes beträgt $\SI{8.37(0.08)e3}{\gram\per\centi\meter\cubed}$, die 
des runden Stabs beträgt $\SI{8.40(0.09)e3}{\gram\per\centi\meter\cubed}.$ Daraus wird geschlossen, dass es sich bei den Stäben um Messing handelt. 
Aus der Literatur\cite{chemie.de} geht für Messing ein Wert von $\SIrange{78}{123}{\giga\pascal}$ hervor.
Zu diesem Theoriewert ergeben sich folgende Abweichungen bei einseitiger Einspannung
\begin{equation*}
    ΔE_{\text{eckig}} = 86.69\% \quad ; \quad ΔE_{\text{rund}} = 89.04\%
\end{equation*} 
und bei beidseitiger Einspannung
\begin{align*}
    ΔE_{\text{eckig, linksseitig}} &= 92.97\% \quad ; &\quad ΔE_{\text{eckig, rechtsseitig}} &= 94.14\% \\
    ΔE_{\text{rund, linksseitig}} &= 93.31\% \quad ; &\quad ΔE_{\text{rund, rechtsseitig}} &= 94.38\%.
\end{align*}
Die Abweichungen wurden mit der Formel
\begin{equation*}
    ΔE = \frac{E_{\text{theorie}}}{E_{\text{experiment}}} - 1
\end{equation*} berechnet.
Die experimentell ermittelten Werte weichen um etwa eine Größenordnung von den theoretischen Werten ab.\\
%Eine mögliche Fehlerquelle ist die Befestigung des Stabes. Es konnte nicht gewährleistet werden, dass sich der Stab bei beidseitiger Einspannung an den Enden nicht 
%mehr bewegt.\\
Alle ermittelten Elastizistätsmodule liegen in einer Größenordnung. Die Abweichungen zur Theorie sind also 
wahrscheinlich auf Systematische Fehler zurückzuführen.\\
\\

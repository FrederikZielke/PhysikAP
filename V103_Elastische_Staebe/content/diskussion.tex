\section{Diskussion}
\label{sec:Diskussion}

\begin{table}
    \centering
    \caption{Elastizitätsmodule $E$ für alle Einspannungskonstellationen.}
    \label{tab:Einspannungskonstellationen}
    \begin{tabular}{|c|c c|}
        \toprule
        & $E_{\text{eckig}}\:[\unit{\giga\pascal}]$ & $E_{\text{rund}}\:[\unit{\giga\pascal}]$\\
        \cmidrule(lr){2-3}
        einseitig & $(0.924 \pm 0.019)\times 10^3$ & $(1.123 \pm 0.027)\times 10^3$\\
        \midrule
        links & $(1.75 \pm 0.12)\times 10^3$ & $(1.84 \pm 0.13)\times 10^3$\\
        rechts & $(2.10 \pm 0.14)\times 10^3$ & $(2.19 \pm 0.15)\times 10^3$\\
        \bottomrule
    \end{tabular}
\end{table}

Um die Ergebnisse zu bewerten wird ermittelt aus welchem Metall die Stäbe bestehen. Durch einsetzen des Volumens sowie der in \autoref{sec:Auswertung}
aufgenommenen Gewichte der Stäbe ergibt sich mit $ρ = \frac{m}{V}$ die Dichte. Die Dichte des eckigen Stabes beträgt $\SI{8.37(0.08)e3}{\gram\per\centi\meter\cubed}$, die 
des runden Stabs beträgt $\SI{8.40(0.09)e3}{\gram\per\centi\meter\cubed}.$ Daraus wird geschlossen, dass es sich bei den Stäben um Messing handelt. 
Aus der Literatur\cite{chemie.de} geht für Messing ein Wert von $\SIrange{78}{123}{\giga\pascal}$ hervor.

Die Abweichungen werden mit der Formel
\begin{equation*}
    ΔE = \left\lvert \frac{E_{\text{experiment}}}{E_{\text{theorie}}} - 1 \right\rvert 
\end{equation*} berechnet.
Zum Theoriewert $\SI{123}{\giga\pascal}$

\begin{equation*}
    ΔE_{\text{eckig}} = 651.2\% \quad ; \quad ΔE_{\text{rund}} = 813\%
\end{equation*} 
und bei beidseitiger Einspannung
\begin{align*}
    ΔE_{\text{eckig, linksseitig}} &= 1322\% \quad ; &\quad ΔE_{\text{eckig, rechtsseitig}} &= 1607\% \\
    ΔE_{\text{rund, linksseitig}} &= 1396\% \quad ; &\quad ΔE_{\text{rund, rechtsseitig}} &= 1680\%.
\end{align*}
Die Abweichung der experimentell ermittelten Werten ist konsistent größer bei beidseitiger Einspannung.\\
%Eine mögliche Fehlerquelle ist die Befestigung des Stabes. Es konnte nicht gewährleistet werden, dass sich der Stab bei beidseitiger Einspannung an den Enden nicht 
%mehr bewegt.\\
\\
Alle ermittelten Elastizistätsmodule liegen in einer Größenordnung. Die Abweichungen zur Theorie sind also 
wahrscheinlich auf Systematische Fehler zurückzuführen.\\
\newpage
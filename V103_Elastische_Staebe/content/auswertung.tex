\section{Auswertung}
\label{sec:Auswertung}
%Elastizitätsmodul bestimmen mit einseitiger und beidseitiger Einspannung von rundem und eckigem Stab
%Flächenträgheitsmoment berechnen für D(x)
%Einseitige Einspannung: 
%Beidseitige Einspannung:
\DeclareSIPower\quartic\tothefourth{4}
Die Maße der Stäbe sind
\begin{align*}
%    d_{eckig} =& \SI{1(0.005)}{\centi\meter} \quad& d_{rund} =& \SI{1(0.005)}{\centi\meter}\\
    d_{eckig} =& (1.0 \pm 0.005)\, \si{\centi\meter} \quad& d_{rund} =& (1.0 \pm 0.005)\, \si{\centi\meter}\\
    l_{eckig} =& \SI{60(0.1)}{\centi\meter} \quad& l_{rund} =& \SI{59.1(0.1)}{\centi\meter}\\
    m_{eckig} =& \SI{502.5(0.1)}{\gram} \quad& m_{rund} =& \SI{390.1(0.1)}{\gram}
\end{align*}

Die in \autoref{sec:Auswertung} gezeigten Grafiken und Ausgleichsrechnungen sind mithilfe der Python-Bibliotheken Matplotlib \cite{matplotlib}, Scipy \cite{scipy} und Numpy \cite{numpy}
erstellt worden.

\subsection{Flächenträgheitsmoment}

Mit Gleichung \eqref{eq:flaeche_traegheit} lässt sich das Flächenträgheitsmoment eines Stabes berechnen.
Es wird für die Berechnung des Elastizistätsmoduls benötigt.\\
\begin{align*}
    I_{eckig} &= \frac{d^4}{12} = \SI{8.33(0.17)e-10}{\quartic\meter} \:\text{und}\\
    &\\
    I_{rund} &= \frac{\pi R^4}{4} = \SI{4.91(0.10)e-10}{\quartic\meter}
\end{align*}

\newpage
\subsection{Einseitige Einspannung}
\label{sec:Einseitige_Einspannung}
\begin{table}
    \begin{minipage}{0.48\linewidth}
        \centering
        \caption{Messwerte für den runden \\Stab bei einseitiger Einspannung.}
        \label{tab:Einseitige_Einspannung_rund}
        \begin{tabular}[h!]{|c|c|}
            \toprule
            {$x\left[\unit{\centi\meter}\right]$} & {$D\left[\unit{\micro\meter}\right]$}\\
            \midrule
            3& 3.0\\
            5& 8.0\\
            7& 16.0\\
            9& 25.0\\
            11& 36.0\\
            13& 49.0\\
            15& 63.0\\
            17& 79.0\\
            19& 97.0\\
            21& 112.0\\
            23& 135.5\\
            25& 152.0\\
            27& 178.5\\
            29& 197.0\\
            31& 222.5\\
            33& 243.0\\
            35& 280.0\\
            37& 301.0\\
            39& 316.5\\
            41& 335.0\\
            43& 364.0\\
            45& 381.0\\
            47& 406.0\\
            49& 433.5\\
            \bottomrule
        \end{tabular}
    \end{minipage}
    \begin{minipage}{0.48\linewidth}
        \centering
        \caption{Messwerte für den eckigen \\Stab bei einseitiger Einspannung.}
        \label{tab:Einseitige_Einspannung_eckig}
        \begin{tabular}[h!]{|c|c|}
            \toprule
            {$x\left[\unit{\centi\meter}\right]$} & {$D\left[\unit{\micro\meter}\right]$}\\
            \midrule
            3& 1.5\\
            5& 4.5\\
            7& 10.0\\
            9& 15.0\\
            11& 22.5\\
            13& 31.0\\
            15& 39.0\\
            17& 49.5\\
            19& 62.0\\
            21& 74.0\\
            23& 86.0\\
            25& 97.0\\
            27& 114.5\\
            29& 127.0\\
            31& 145.0\\
            33& 164.0\\
            35& 182.0\\
            37& 199.0\\
            39& 219.0\\
            41& 243.0\\
            43& 256.0\\
            45& 272.0\\
            47& 295.0\\
            49& 306.0\\
            \bottomrule
        \end{tabular}
    \end{minipage}
\end{table}
\newpage
\begin{figure}
    \centering
    \begin{subfigure}{0.49\textwidth}
        \centering
        \includegraphics[height=5.3cm]{build/rund_einseitig.pdf}
        \caption{Messwerte aus \autoref{tab:Einseitige_Einspannung_rund} mit linearer Regression.}
        \label{fig:rund_einseitig}
    \end{subfigure}
    \begin{subfigure}{0.49\textwidth}
        \centering
        \includegraphics[height=5.3cm]{build/eckig_einseitig.pdf}
        \caption{Messwerte aus \autoref{tab:Einseitige_Einspannung_eckig} mit linearer Regression.}
        \label{fig:eckig_einseitig}
    \end{subfigure}
%    \caption{linreg bei einseitiger Einspannung}%das muss noch schöner geschrieben werden
    \label{fig:einseitig}
\end{figure}

Für die Messwerte in \autoref{tab:Einseitige_Einspannung_rund} ergeben sich mit einer linearen Regression die Koeffizienten $a \approx \SI{4.887(0.063)e-3}{\per\meter\squared}$ und $b \approx \SI{0.010(0.003)e-3}{\meter}$.
Die Ausgleichsgerade und Messwerte sind in \autoref{fig:rund_einseitig} dargestellt.
%Für den runden Stab ergeben sich  die Koeffizienten $a = \SI{4.887544(0.063457)e-3}{\per\meter\squared}$ und $b = \SI{1.0839(0.2872)e-5}{\meter}$.
In den Vorfaktor der Gleichung \eqref{eq:ausgleich} eingesetzt und nach $E$ aufgelöst,
ergibt sich mit $F = m \cdot g$ der Elastizitätsmodul $E = \SI{11.23(0.27)e11}{\pascal}.$\\
\\
Für die Messwerte in \autoref{tab:Einseitige_Einspannung_eckig} ergeben sich mit einer linearen Regression die Koeffizienten $a \approx \SI{3.501(0.019)e-3}{\per\meter\squared}$ und $b \approx \SI{0.001(0.001)e-3}{\meter}$.
Die Ausgleichsgerade und Messwerte sind in \autoref{fig:eckig_einseitig} dargestellt.
%Für den eckigen Stab ergeben sich die Koeffizienten \\$a = \SI{3.500933(0.019402)e-3}{\per\meter\squared}$ und $b = \SI{0.1047(0.0878)e-5}{\meter}$.
Daraus folgt mit Gleichung \eqref{eq:ausgleich} nach $E$ umgestellt der Elastizitätsmodul $E = \SI{9.24(0.19)e11}{\pascal}.$


\newpage
\subsection{Beidseitige Einspannung}
\begin{table}
    \begin{minipage}{0.48\linewidth}
        \centering
        \caption{Messwerte für den runden \\Stab bei beidseitiger Einspannung.}
        \label{tab:Beidseitige_Einspannung_rund}
        \begin{tabular}[h!]{|c|c|}
            \toprule
            {$x\left[\unit{\centi\meter}\right]$} & {$D\left[\unit{\micro\meter}\right]$}\\
            \midrule
            3& 1.0\\
            5& 4.0\\
            7& 4.5\\
            9& 6.5\\
            11& 9.5\\
            13& 13.5\\
            15& 18.0\\
            17& 22.0\\
            19& 24.0\\
            21& 26.0\\
            23& 29.0\\
            25& 30.0\\
            \hline
            27& 34.0\\
            29& 33.0\\
            31& 32.0\\
            33& 32.0\\
            35& 32.0\\
            37& 32.0\\
            39& 29.0\\
            41& 27.0\\
            43& 25.0\\
            45& 22.0\\
            47& 19.0\\
            49& 11.0\\
            51& 7.0\\
            53& 3.0\\
            \bottomrule
        \end{tabular}
    \end{minipage}
    \begin{minipage}{0.48\linewidth}
        \centering
        \caption{Messwerte für den eckigen \\Stab bei beidseitiger Einspannung.}
        \label{tab:Beidseitige_Einspannung_eckig}
        \begin{tabular}[h!]{|c|c|}
            \toprule
            {$x\left[\unit{\centi\meter}\right]$} & {$D\left[\unit{\micro\meter}\right]$}\\
            \midrule
            3& 0.0\\
            5& 0.5\\
            7& 2.0\\
            9& 5.0\\
            11& 6.0\\
            13& 8.5\\
            15& 9.0\\
            17& 12.0\\
            19& 13.5\\
            21& 15.0\\
            23& 17.5\\
            25& 19.0\\
            \hline
            30& 19.0\\
            32& 18.0\\
            34& 18.5\\
            36& 18.0\\
            38& 17.0\\
            40& 15.0\\
            42& 15.0\\
            44& 13.0\\
            46& 10.0\\
            48& 7.5\\
            50& 4.0\\
            52& 2.5\\
            \bottomrule
        \end{tabular}
    \end{minipage}
\end{table}
\newpage
%\begin{figure}[h!]
%    \centering
\begin{figure}%{0.49\textwidth}
    \centering
    \includegraphics[height=6.5cm]{build/beidseitig_e1.pdf}
    \caption{Messwerte zwischen Einspannung und Gewicht aus \autoref{tab:Beidseitige_Einspannung_eckig} mit linearer Regression geplottet.}
    \label{fig:eckig_beidseitig1}
\end{figure}
\begin{figure}%{0.49\textwidth}
    \centering
    \includegraphics[height=6.5cm]{build/beidseitig_e2.pdf}
    \caption{Messwerte zwischen Stabende und Gewicht aus \autoref{tab:Beidseitige_Einspannung_eckig} mit linearer Regression geplottet.}
    \label{fig:eckig_beidseitig2}
\end{figure}
\begin{figure}%{0.49\textwidth}
    \centering
    \includegraphics[height=6.5cm]{build/beidseitig_r1.pdf}
    \caption{Messwerte zwischen Gewicht und Einspannung aus \autoref{tab:Beidseitige_Einspannung_rund} mit linearer Regression geplottet.}
    \label{fig:rund_beidseitig1}
\end{figure}
\newpage
\begin{figure}%{0.49\textwidth}
    \centering
    \includegraphics[height=6.5cm]{build/beidseitig_r2.pdf}
    \caption{Messwerte zwischen Stabende und Einspannung aus \autoref{tab:Beidseitige_Einspannung_rund} mit linearer Regression geplottet.}
    \label{fig:rund_beidseitig2}
\end{figure}
%    \caption{Messwerte für die beidseitige Einspannung der Stäbe aus \autoref{tab:Beidseitige_Einspannung_rund} und \autoref{tab:Beidseitige_Einspannung_eckig} mit linearer Regression.}
%    \label{fig:beidseitig}
%\end{figure}

%Für die Koeffizienten der linearen Regressionen des Graphen aus \autoref{fig:eckig_beidseitig1}, wurden die Werte $a = \SI{1.47243(0.10064)e4}{\per\meter\squared}$ und $b = \SI{5.906(1.1)e6}{\meter}$ ermittelt.
Bei beidseitiger Befestigung der Stäbe wird unterschieden ob Links- oder Rechtsseitig des Gewichtes gemessen wird.
Die Messwerte wurden deshalb in mehrere Graphen aufgeteilt. %Linksseitig wird $D(x)$ mit der Formel %\eqref{hier linksseitige formel einfügen}
%berechnet und Rechtsseitig mit der Formel %\eqref{hier rechtsseitige formel einfügen} berechnet. 
Die Auslenkung $D(x)$ wird mit den Formeln \eqref{eq:auslenkung} berechnet.
Die jeweiligen Koeffizienten aus der linearen Regression sind in \autoref{tab:beidseitig_koeffizienten} aufgelistet.
\newpage
\begin{table}[h!]
    \centering
    \caption{Koeffizienten der linearen Regressionen der Messwerte aus \autoref{tab:Beidseitige_Einspannung_eckig} und \autoref{tab:Beidseitige_Einspannung_rund}.}
    \label{tab:beidseitig_koeffizienten}
%    \sisetup{table-format=2.1}
    \begin{tabular}{|c|c c|}
        \toprule
        & \multicolumn{2}{c}{eckig}\\
        \cmidrule(lr){2-3}
        & $a \:[\unit{\per\meter\squared}]$ & $b \:[\unit{\meter}]$\\% & $a \:[\unit{\per\meter\squared}]$ & $b \:[\unit{\meter}]$ \\
        \midrule
        links &  $(1.472 \pm 0.100)\times 10^{-4}$ & $(-0.059 \pm 0.011)\times 10^{-4}$\\% 
        rechts & $(1.226 \pm 0.076)\times 10^{-4}$ & $(0.017 \pm 0.007)\times 10^{-4}$\\% 
        \midrule
        &\multicolumn{2}{c}{rund}\\ 
        \midrule
        links &  $(2.374 \pm 0.160)\times 10^{-4}$ & $(-0.083 \pm 0.017)\times 10^{-4}$ \\
        rechts & $(1.997 \pm 0.132)\times 10^{-4}$ & $(0.050 \pm 0.014)\times 10^{-4}$ \\
        \bottomrule
    \end{tabular}
\end{table}

Die zugehörigen Elastizistätsmodul $E$ werden durch Umstellen der Formel \eqref{eq:ausgleich} berechnet. Der Vorfaktor der Gleichungen %\eqref{} bzw \eqref{}
wird jeweils nach $E$ umgestellt. Die Kraft $F = m \cdot g$ wird eingesetzt, sowie die entsprechenden Koeffizienten $a$ aus \autoref{tab:beidseitig_koeffizienten}.\\
\\
\begin{table}
    \centering
    \caption{Elastizitätsmodul $E$ für die beidseitige Einspannung der Stäbe.}
    \label{tab:beidseitig_E}
    \begin{tabular}{c|c c|}
        \toprule
        & $E_{\text{eckig}}\:[\unit{\giga\pascal}]$ & $E_{\text{rund}}\:[\unit{\giga\pascal}]$\\
        \cmidrule(lr){2-3}
        links & $(1.75 \pm 0.12)\times 10^3$ & $(1.84 \pm 0.13)\times 10^3$\\
        rechts & $(2.10 \pm 0.14)\times 10^3$ & $(2.19 \pm 0.15)\times 10^3$\\
        \bottomrule
    \end{tabular}
\end{table}
\newpage
%Denk dran die Graphen nochmal zu updaten das D(x) in μm ist und nicht in m
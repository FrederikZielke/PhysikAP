\section{Auswertung}
\label{sec:Auswertung}
%Elastizitätsmodul bestimmen mit einseitiger und beidseitiger Einspannung von rundem und eckigem Stab
%Flächenträgheitsmoment berechnen für D(x)
%Einseitige Einspannung: 
%Beidseitige Einspannung:
\DeclareSIPower\quartic\tothefourth{4}
Die Maße der Stäbe sind
\begin{align*}
    d_{eckig} =& \SI{1(0.005)}{\centi\meter} \quad& d_{rund} =& \SI{1(0.005)}{\centi\meter}\\
    l_{eckig} =& \SI{60(0.1)}{\centi\meter} \quad& l_{rund} =& \SI{59.1(0.1)}{\centi\meter}\\
    m_{eckig} =& \SI{502.5(0.1)}{\gram} \quad& m_{rund} =& \SI{390.1(0.1)}{\gram}
\end{align*}

Die in \autoref{sec:Auswertung} gezeigten Grafiken und Ausgleichsrechnungen sind mithilfe der Python-Bibliotheken Matplotlib \cite{matplotlib}, Scipy \cite{scipy} und Numpy \cite{numpy}
erstellt worden.

\subsection{Flächenträgheitsmoment}

Mit Gleichung \eqref{eq:flaeche_traegheit} lässt sich das Flächenträgheitsmoment eines Stabes berechnen.
\begin{align*}
    I_{eckig} &=  \int_{-\frac{d}{2}}^{\frac{d}{2}}\int_{-\frac{d}{2}}^{\frac{d}{2}} y^2 dydz\\
    \Leftrightarrow I_{eckig} &= \frac{d^4}{12} = \SI{8.33(0.17)e-10}{\quartic\meter} \:\text{und}\\
    I_{rund} &=  \int_{0}^{2π}\int_{0}^{R} y^2 dydz\\
    \Leftrightarrow I_{rund} &= \frac{\pi R^4}{4} = \SI{4.91(0.10)e-10}{\quartic\meter}
\end{align*}

\newpage
\subsection{Einseitige Einspannung}
\begin{table}
    \begin{minipage}{0.48\linewidth}
        \centering
        \caption{Messwerte für den runden \\Stab bei einseitiger Einspannung.\cite{V103}}
        \label{tab:Einseitige_Einspannung_rund}
        \begin{tabular}[h!]{|c|c|}
            \toprule
            {$x\left[\unit{\centi\meter}\right]$} & {$D\left[\unit{\micro\meter}\right]$}\\
            \midrule
            3& 3.0\\
            5& 8.0\\
            7& 16.0\\
            9& 25.0\\
            11& 36.0\\
            13& 49.0\\
            15& 63.0\\
            17& 79.0\\
            19& 97.0\\
            21& 112.0\\
            23& 135.5\\
            25& 152.0\\
            27& 178.5\\
            29& 197.0\\
            31& 222.5\\
            33& 243.0\\
            35& 280.0\\
            37& 301.0\\
            39& 316.5\\
            41& 335.0\\
            43& 364.0\\
            45& 381.0\\
            47& 406.0\\
            49& 433.5\\
            \bottomrule
        \end{tabular}
    \end{minipage}
    \begin{minipage}{0.48\linewidth}
        \centering
        \caption{Messwerte für den eckigen \\Stab bei einseitiger Einspannung.\cite{V103}}
        \label{tab:Einseitige_Einspannung_eckig}
        \begin{tabular}[h!]{|c|c|}
            \toprule
            {$x\left[\unit{\centi\meter}\right]$} & {$D\left[\unit{\micro\meter}\right]$}\\
            \midrule
            3& 1.5\\
            5& 4.5\\
            7& 10.0\\
            9& 15.0\\
            11& 22.5\\
            13& 31.0\\
            15& 39.0\\
            17& 49.5\\
            19& 62.0\\
            21& 74.0\\
            23& 86.0\\
            25& 97.0\\
            27& 114.5\\
            29& 127.0\\
            31& 145.0\\
            33& 164.0\\
            35& 182.0\\
            37& 199.0\\
            39& 219.0\\
            41& 243.0\\
            43& 256.0\\
            45& 272.0\\
            47& 295.0\\
            49& 306.0\\
            \bottomrule
        \end{tabular}
    \end{minipage}\newpage
        \caption{Messwerte aus \autoref{tab:Einseitige_Einspannung_rund} mit linearer Regression.}
    \end{subfigure}
    \begin{subfigure}
        \includegraphics[width=\linewidth]{build/eckig_einseitig.pdf}
        \caption{Messwerte aus \autoref{tab:Einseitige_Einspannung_eckig} mit linearer Regression.}
    \end{subfigure}
\end{figure}

Für den runden Stab ergeben sich  die Koeffizienten $a = \SI{4887.544(63.457)}{\per\meter\squared}$ und $b = \SI{10.839(2.872)}{\per\meter\squared}$.
Daraus folgend ergibt sich für den Elastizitätsmodul $E = .$


\newpage
\subsection{Beidseitige Einspannung}
\subsubsection*{Runder Stab}
\subsubsection*{Eckiger Stab}
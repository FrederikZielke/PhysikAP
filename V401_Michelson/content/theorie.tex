\section{Zielsetzung}
\label{sec:Zielsetzung}

\section{Theorie}
\label{sec:Theorie}

\subsection{Grundlagen}
\label{sec:Grundlagen}

% Licht wird durch angeregte Atome ausgestrahlt


% Licht als E(M)-Welle
Das Verhalten von Licht kann gut durch eine Betrachtung als elektromagnetische Wellen
dargestellt werden. Diese Wellen lassen sich durch die Maxwellgleichungen beschreiben.
\begin{align*}
    \text{Erste } & \text{Maxwellgleichung:} & \text{Zweite } & \text{Maxwellgleichung:} \\
    \vec{\nabla} \cdot \vec{E} &= 0 & \vec{\nabla} \cdot \vec{B} &= 0 \\
    \text{Dritte } & \text{Maxwellgleichung:} & \text{Vierte } & \text{Maxwellgleichung:} \\
    \vec{\nabla} \times \vec{E} &= - \frac{\partial \vec{B}}{\partial t}   &  \vec{\nabla} \times \vec{B} &= \mu_0 \epsilon_0 \frac{\partial \vec{E}}{\partial t}
\end{align*}
Hierbei steht $E$ für die elektrische Feldstärke, $ε_0$ für die elektrische Feldkonstante,
$B$ für die magnetische Flussdichte und $μ_0$ für die magnetische Feldkonstante.
Im weiteren Verlauf wird nur das E-Feld betrachtet.
Die Orts- und Zeitabhängigkeit des Feldes einer ebenen Welle lässt sich ausrücken durch
\begin{equation}\label{eq:ebeneWelle}
    \vec{E}\left(x,t\right) = \vec{E}_0 \cos\left( k x - ω t - δ\right)\, ,
\end{equation}
mit $\vec{E}_0$ als gerichtete Amplitude, $k$ als Wellenzahl, $x$ als Ortskoordinate, $ω$ als Wellenzahl, $t$ als Zeit und $δ$ als Phasenverschiebung.
% lin Superposition
Bei den Maxwellgleichungen handelt es sich um lineare Differentialgleichungen.
Aufgrund dieser Linearität lassen sich einzelne Lichtwellen auch linear Superpositionieren.
Die Feldstärke an einem bestimmten Raumpunkt ist gleich der Summe der einzelnen Wellen.
\begin{equation*}
    \vec{E} = \vec{E}_1 + \vec{E}_2 + \vec{E}_3 + ...
\end{equation*}
% -> Messung des E-Felds durch Lichtintensität I
Das E-Feld selbst lässt sich nicht direkt messen.
Um also das E-Feld zu messen, nimmt man den "Umweg" über die Lichtintensität $I$.
Diese ist proportional zum Betragsquadrat des E-Feldes $I = \text{const} |\vec{E}|^2$.
Die Intensität $I_{\text{ges}}$ an einem Ort $x$ auf den zwei Quellen strahlen,
ergibt sich zu 
\begin{equation*}
    I_{\text{ges}} = \frac{\text{const}}{t_2 - t_1} \int_{t_1}^{t_2} \left( \vec{E}_1 + \vec{E}_2 \right)^2 \left(x,t\right) \text{d}t\, .
\end{equation*}
Durch ausrechnen gelangt man zu dem Term
\begin{equation*}
    I_{\text{ges}} = 2\text{const}\vec{E}_0^2\left(1 + \cos{δ_2-δ_1}\right).
\end{equation*}
Die Intensität setzt sich also aus der Summe der Einzelintensitäten zusammen,
sowie einem \textbf{Interferenzterm}
\begin{equation*}
    2\text{const}\vec{E}_0^2\cos{δ_2-δ_1}\, .
\end{equation*}
Dieser Interferenzterm ist abhängig von der Phasenverschiebung $δ$ und erzeugt eine Abweichung von bis zu $\pm 2\text{const}\vec{E}_0^2.$ 
Der Interferenzterm verschwindet wenn die Differenz der beiden Einzelintensitäten gleich $(2n + 1)π$ mit $n \in \mathbb{N}_0$ ist.

\subsection{Interferenz}
\label{sec:Interferenz}

Licht wird durch angeregte Atome ausgestrahlt. 
Kehren angeregt Elektronen in ihren Grundzustand zurück, wird Licht emmitiert.
Die ausgesandten Wellenzüge haben eine endliche Länge.
Die Wellenlänge und Phasenverschiebung sind dabei statistisch verteilt.
Darum können diese Quellen nicht unverändert zur Erzeugung von Interferenzeffekten verwendet werden.
Nicht interferenzfähiges Lich wird \textbf{inkohärent} genannt.
Kohärentes Licht zeichnet sich durch eine feste Wellenzahl $k,$ Frequenz $ω$ und Phasenverschiebung $δ$ in \autoref{eq:ebeneWelle} aus.
Im Michelson-Interferometer wird ein LASER verwendet.
Dieser erzeugt kohärentes Licht.
%Meinst du wir sollten noch genauer drauf eingehen was ein Laser ist?

Um einen Gangunterschied bei kohährentem Licht zu erzeugen, kann der Strahl gespalten werden.
Beträgt der Wegunterschied an einem Punkt P zwischen den beiden erzeugten Strahlen
\begin{equation*}
    Δ = \left(2n + 1\right)\frac{λ}{2}\, , n \in \mathbb{N}_0\, ,
\end{equation*}
kommt es zu destruktiver Interferenz. Die beiden Strahlen löschen sich aus.
Wichtig ist, dass die Strahlen in einer endlichen Zeit $τ$ ausgesannt werden.
Außerdem darf der Wegunterschied der beiden wieder zusammen geführten Strahlen nicht zu groß sein.
Dadurch ist die Phasenbeziehung zueinander noch intakt.
Diese Eigenschaft heißt Kohärenzlänge und wird berechnet mit
\begin{equation*}
    l = N λ\, .
\end{equation*}
Dabei ist $N$ die Anzahl der Intensitätsmaxima die in P zu sehen sind.
%Welches Licht ist (nicht) interferenzfähig?
%   -> Kohärenz, kohärentes Licht(-> \vec(E) = E cos(kx -  ωt + δ))
%       -> Laser
%       -> endliche länge der Wellenzüge
%       -> Kohärenzlänge, Kohärenzzeit
%   ->wie kann interferenz erzeugt werden?

\subsection{Polarisation}
\label{sec:Polarisation}

%Polarisation

\subsection{Michelson-Interferometer}
\label{sec:Michelson-Interferometer}

%Messung optischer Größen mit Interferenz
%   ->grundlegende Funktionsweise erklären
%   ->...
%   ->Intensitätsfunktion I(d) = ...
%   ->Helligkeitsmaxima Δd = zλ/2 oder bei veränderung des Brechungsindex n b*n = zλ/2
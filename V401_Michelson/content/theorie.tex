\section{Zielsetzung}
\label{sec:Zielsetzung}

\section{Theorie}
\label{sec:Theorie}

\subsection{Grundlagen}
\label{sec:Grundlagen}

% Licht als E(M)-Welle
Das Verhalten von Licht kann gut durch eine Betrachtung als elektromagnetische Wellen
dargestellt werden. Diese Wellen lassen sich durch die Maxwellgleichungen beschreiben.
\begin{align*}
    \text{Erste } & \text{Maxwellgleichung:} & \text{Zweite } & \text{Maxwellgleichung:} \\
    \vec{\nabla} \cdot \vec{E} &= 0 & \vec{\nabla} \cdot \vec{B} &= 0 \\
    \text{Dritte } & \text{Maxwellgleichung:} & \text{Vierte } & \text{Maxwellgleichung:} \\
    \vec{\nabla} \times \vec{E} &= - \frac{\partial \vec{B}}{\partial t}   &  \vec{\nabla} \times \vec{B} &= \mu_0 \epsilon_0 \frac{\partial \vec{E}}{\partial t}
\end{align*}
% lin Superposition
% -> Messung des E-Felds durch Lichtintensität I
% Licht wird durch angeregte Atome ausgestrahlt

\subsection{Interferenz}
\label{sec:Interferenz}

%Welches Licht ist (nicht) interferenzfähig?
%   -> Kohärenz, kohärentes Licht(-> \vec(E) = E cos(kx -  ωt + δ))
%       -> Laser
%       -> endliche länge der Wellenzüge
%       -> Kohärenzlänge, Kohärenzzeit
%   ->wie kann interferenz erzeugt werden?

\subsection{Polarisation}
\label{sec:Polarisation}

%Polarisation

\subsection{Michelson-Interferometer}
\label{sec:Michelson-Interferometer}

%Messung optischer Größen mit Interferenz
%   ->grundlegende Funktionsweise erklären
%   ->...
%   ->Intensitätsfunktion I(d) = ...
%   ->Helligkeitsmaxima Δd = zλ/2 oder bei veränderung des Brechungsindex n b*n = zλ/2
\section{Zielsetzung}
\label{sec:Zielsetzung}

\section{Theorie}
\label{sec:Theorie}

\subsection{Grundlagen}
\label{sec:Grundlagen}

% Licht wird durch angeregte Atome ausgestrahlt
Hier muss noch ergänzt werden

% Licht als E(M)-Welle
Das Verhalten von Licht kann gut durch eine Betrachtung als elektromagnetische Wellen
dargestellt werden. Diese Wellen lassen sich durch die Maxwellgleichungen beschreiben.
\begin{align*}
    \text{Erste } & \text{Maxwellgleichung:} & \text{Zweite } & \text{Maxwellgleichung:} \\
    \vec{\nabla} \cdot \vec{E} &= 0 & \vec{\nabla} \cdot \vec{B} &= 0 \\
    \text{Dritte } & \text{Maxwellgleichung:} & \text{Vierte } & \text{Maxwellgleichung:} \\
    \vec{\nabla} \times \vec{E} &= - \frac{\partial \vec{B}}{\partial t}   &  \vec{\nabla} \times \vec{B} &= \mu_0 \epsilon_0 \frac{\partial \vec{E}}{\partial t}
\end{align*}
Hierbei steht $E$ für die elektrische Feldstärke, $ε_0$ für die elektrische Feldkonstante,
$B$ für die magnetische Flussdichte und $μ_0$ für die magnetische Feldkonstante.
Im weiteren Verlauf wird nur das E-Feld betrachtet.
Die Orts- und Zeitabhängigkeit des Feldes einer ebenen Welle lässt sich ausrücken durch
\begin{equation*}
    \vec{E}\left(x,t\right) = \vec{E}_0 \cos\left( k x - ω t - δ\right)\, ,
\end{equation*}
mit $\vec{E}_0$ als gerichtete Amplitude, $k$ als Wellenzahl, $x$ als Ortskoordinate, $ω$ als Wellenzahl, $t$ als Zeit und $δ$ als Phasenverschiebung.
% lin Superposition
Bei den Maxwellgleichungen handelt es sich um lineare Differentialgleichungen.
Aufgrund dieser Linearität lassen sich einzelne Lichtwellen auch linear Superpositionieren.
Die Feldstärke an einem bestimmten Raumpunkt ist gleich der Summe der einzelnen Wellen.
\begin{equation*}
    \vec{E} = \vec{E}_1 + \vec{E}_2 + \vec{E}_3 + ...
\end{equation*}
% -> Messung des E-Felds durch Lichtintensität I
Das E-Feld selbst lässt sich nicht direkt messen.
Um also das E-Feld zu messen, nimmt man den "Umweg" über die Lichtintensität $I$.
Diese ist proportional zum Betragsquadrat des E-Feldes $I = \text{const} |\vec{E}|^2$.
Die Intensität $I_{\text{ges}}$ an einem Ort $x$ auf den zwei Quellen strahlen,
ergibt sich zu 
\begin{equation*}
    I_{\text{ges}} = \frac{\text{const}}{t_2 - t_1} \int_{t_1}^{t_2} \left( \vec{E}_1 + \vec{E}_2 \right)^2 \left(x,t\right) \text{d}t\, .
\end{equation*}
Durch ausrechnen gelangt man zu dem Term
\begin{equation*}
    I_{\text{ges}} = 2\text{const}\vec{E}_0^2\left(1 + \cos{δ_2-δ_1}\right).
\end{equation*}
Die Intensität setzt sich also aus der Summe der Einzelintensitäten zusammen,
sowie einem \textbf{Interferenzterm}
\begin{equation*}
    2\text{const}\vec{E}_0^2\cos{δ_2-δ_1}\, .
\end{equation*}
Dieser Interferenzterm ist abhängig von der Phasenverschiebung $δ$ und erzeugt eine Abweichung von bis zu $\pm 2\text{const}\vec{E}_0^2.$ 
Der Interferenzterm verschwindet wenn die Differenz der beiden Einzelintensitäten gleich $(2n + 1)π$ mit $n \in \mathbb{N}_0$ ist.

\subsection{Interferenz}
\label{sec:Interferenz}

%Welches Licht ist (nicht) interferenzfähig?
%   -> Kohärenz, kohärentes Licht(-> \vec(E) = E cos(kx -  ωt + δ))
%       -> Laser
%       -> endliche länge der Wellenzüge
%       -> Kohärenzlänge, Kohärenzzeit
%   ->wie kann interferenz erzeugt werden?

\subsection{Polarisation}
\label{sec:Polarisation}

%Polarisation

\subsection{Michelson-Interferometer}
\label{sec:Michelson-Interferometer}

%Messung optischer Größen mit Interferenz
%   ->grundlegende Funktionsweise erklären
%   ->...
%   ->Intensitätsfunktion I(d) = ...
%   ->Helligkeitsmaxima Δd = zλ/2 oder bei veränderung des Brechungsindex n b*n = zλ/2
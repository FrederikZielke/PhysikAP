\section{Diskussion}
\label{sec:Diskussion}
Die Abweichungen der berechneten Größen von der Theorie werden nach 
\begin{equation}\label{1}
    \Delta = |\frac{exp - theo}{theo} \cdot 100|
\end{equation}
berechnet.
\subsection{Die Wellenlänge des Lasers}
Der verwendete Laser emittiert Licht mit einer Wellenlänge von $\lambda_{Herst} = \SI{680}{nm}$. Dadurch ergeben sich die in \autoref{tab:abw1} dargestellten
prozentualen Abweichungen von der Angabe des Lasers.
\begin{table}[H]
    \centering
    \caption{Berechnete Wellenlänge $\lambda$ und Herstellerangabe $\lambda_{Herst}$}
    \begin{tabular}{c c}
        \toprule
        $\lambda \,[\unit{nm}]$ & $\Delta\,[\unit{\%}]$\\
        \midrule
        $\SI{717(14)}{}$ & 5.4\\
        $\SI{808(16)}{}$ & 18.8\\
        \bottomrule
    \end{tabular}
    \label{tab:abw1}
\end{table}
Die Wellenlänge $\lambda_v$ bei vorwärts drehendem Motor weicht mit $\SI{5.4}{\%}$ nicht groß von der Theorie ab. Jedoch ist die Abweichung bei $\lambda_z$ bei rückwärts
drehendem Motor von $\SI{18.8}{\%}$ sogar so groß, dass $\lambda_z = \SI{808(16)}{nm}$ außerhalb des sichtbaren Spektrums liegt.
Gründe dafür sind vielseitig. Zum Einen schien der Motor rückwärts nicht sauber zu drehen. Zudem lief der Zähler einigemale auch weiter, obwohl der Spiegel nicht bewegt wurde.
Desweiteren ist der Versuchsaufbau sehr empfindlich, d.h. schon kleine Erschütterungen können für ein verfälschtes Ergebnis sorgen.

\subsection{Brechungsindex von Luft}
Der Brechungsindex von Luft ist von der Zusammensetzung aus den verschiedenen Gasen abhängig. Um die Messung und das Ergebnis zu überprüfen wird
ein Literaturwert zum Vergleich betrachtet. Dieser ist $n_{Lit} = 1.00027717$ \cite{n_luft}. Die Abweichung des bestimmten Werts beträgt nach \autoref{1}
\begin{table}[H]
    \centering
    \caption{Experimentell bestimmter Brechungsindex $n_{exp}$, Literaturwert $n_{Lit}$ und Abweichung $\Delta$.}
    \begin{tabular}{c c c}
        \toprule
        $n_{exp}$ & $n_{Lit}$ & $\Delta\,[\unit{\%}]$\\
        \midrule
        $\SI{1.000207(0.000021)}{}$ & 1.00027717 & 0.007\\
        \bottomrule
    \end{tabular}
    \label{tab:abw1}
\end{table}
Mit dem verwendeten Aufbau sind Brechungsindexänderungen im Bereich von $10^{-5}$ messbar. Da die Größenordnung der Brechungsindexänderung in einem sehr kleinen Bereich
liegt verliert die prozentuale Abweichung an Aussagekraft. Relevanter ist, ob der Literaturwert im Fehlerbereich der Messung liegt. 
Dies ist nicht gegeben und auf die erwarteten Ungenauigkeiten, wie z.B. dass keine genaue und konstante Druckänderung $\Delta p$ erreicht werden konnte oder bereits kleine
Erschütterungen das Ergebnis verfälschen, zurückzuführen.

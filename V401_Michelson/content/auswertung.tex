\section{Auswertung}
\label{sec:Auswertung}
Die in \autoref{sec:Auswertung} gezeigten Grafiken und Rechnungen sind mithilfe der Python-Bibliotheken Matplotlib \cite{matplotlib}, Scipy \cite{scipy} und Numpy \cite{numpy}
erstellt worden.

\subsection{Bestimmung der Wellenlänge eines Lasers}
\begin{table}[H]
  \centering
  \caption{Gezählte Maxima bei vorwärts und rückwärts drehendem Motor.}
  \begin{tabular}{c c}
      \toprule
      vor & zurück\\
      \midrule
      2403 & 2730\\
      2533 & 2720\\
      2434 & 2689\\
      2435 & 2693\\
      2453 & 2987\\
      \bottomrule
  \end{tabular}
  \label{tab:wellenleange}
\end{table}
Die Messdaten bei Verschiebung des Spiegels sind in \autoref{tab:wellenleange} aufgeführt. Da deutliche Unterschiede bei der Zählung der 
Maxima je nach Drehrichtung des Motors aufgetreten sind, wurde hier zwischen vor und zurück unterschieden. Der Motor bewegt sich um 
$\Delta_{motor} = \SI{5(0.1)}{mm}$. Durch die Hebelübersetzung von 1:5.046 bewegt sich der Spiegel also absolut um $\Delta s = \SI{0.991(0.020)}{mm}$.
\\
\\
Aus den Werten in \autoref{tab:wellenleange} wird jeweils ein Mittelwert $\overline{vor}$ bzw. $\overline{zurueck}$ gebildet. Ein dabei auftretender Fehler wird aufgrund von 
systematischen Fehlern (z.B. wenn der Motor nicht rechtzeitig gestoppt wird oder bei Erschütterungen der Messapperatur) auf $vor_{err} = \pm 100$ und $zurueck_{err} = \pm 50$ festgelegt.
So ergibt sich 
\begin{align*}
  \overline{v} &= \SI{2763.8(100)}{}\\
  \text{und } \overline{z}\, &= \SI{2451.6(50)}{}.
\end{align*}
Der Fehler des Motors wird auf $motor_{err} = \pm 0.1 \unit{mm}$ festgelegt. Einerseits ist dies mit einem verzögerten Ausschalten des Motors sowie einem nicht
ganz sauberen laufenden Motors und dem Standardfehler der Mikrometerschraube zu begründen.
\\
\\
Mithilfe dieser Werte kann dann nach \autoref{eq:lambda} die Wellenlänge des Lasers bestimmt werden. Auch hier wird aufgrund deutlicher Abweichungen je nach Drehrichtung
ein $\lambda_{v}$ und $\lambda_{z}$ angegeben.
Es ergibt sich 
\begin{align*}
  \lambda_{v} &= \SI{717(14)}{nm}\\
  \text{und } \lambda_{z} &= \SI{808(16)}{nm}.
\end{align*}

\subsection{Bestimmung des Brechungsindexes von Luft}
Die Anzahl der Maxima beim reinpumpen und rauslassen der Luft sind in \autoref{tab:Brechungsindex} aufgelistet.
\begin{table}[H]
  \centering
  \caption{Gezählte Maxima beim Rein- und Rauslassen der Luft.}
  \begin{tabular}{c c}
      \toprule
      $Z_{rein}$ & $Z_{raus}$\\
      \midrule
      15 & 48\\
      13 & 43\\
      11 & 42\\
      11 & 44\\
      13 & 47\\
      \bottomrule
  \end{tabular}
  \label{tab:Brechungsindex}
\end{table}
Hierbei ist wichtig zu erwähnen, dass der Zähler nicht nach dem reinlassen der Luft zurückgesetzt wurde, d.h. die Anzahl der gezählten Maxima beim Herauslassen der Luft
ergeben sich mit
\begin{equation*}
  Z_{raus} = Z_{raus} - Z_{rein}.
\end{equation*}
Da beim Rein- bzw. Rauslassen der Luft nicht immer ein $\Delta p = \SI{600}{mmHg}$ gewährt worden konnte, wird ein Fehler von $\pm \SI{20}{mmHg}$ festgelegt, d.h. es gilt 
$\Delta p = \SI{600(20)}{mmHg}$.
Da kein konstantes $\Delta p$ bei jeder Messung erreicht werden konnte, muss auch ein Fehler für die Anzahl der Maxima abgeschätzt werden. Aufgrund der Schwankungen
wird ein Fehler von $\pm 3$ angesetzt.
Aus den Messwerten von $Z_{raus}$ und $Z_{rein}$ wird ein Mittelwert $\overline{Z} = \SI{22.4(2.1)}{}$ gebildet. Der Brechungsindex von Luft ergibt sich dann zu
\begin{equation*}
  n = 1 + \Delta n = \SI{1.000207(0.000021)}{}.
\end{equation*}
Das $\Delta n$ wird dabei nach \autoref{eq:delta_n} wie folgt berechnet:
\begin{equation*}
  \Delta n = \frac{\overline{Z}\lambda \cdot T \cdot p_0}{2b \cdot T_0 \cdot \Delta p}.
\end{equation*}
Dabei ist $\overline{Z}$ die gemittelte Anzahl der gezählten Maxima, $\lambda = \SI{680}{nm}$ die Wellenlänge des Lasers, $T = \SI{293.15(1)}{K}$ die Raumtemperatur,
$p_0 = \SI{101325}{Pa}$ \cite{p_normal} der Normaldruck, $b = \SI{50}{mm}$ die Schichtdicke der Messzelle, $T_0 = \SI{273.15}{K}$ die Normaltemperatur und $\Delta p = \SI{8.00(0.27)e+04}{Pa}$
die durch die Pumpe hervorgerufene Druckänderung.

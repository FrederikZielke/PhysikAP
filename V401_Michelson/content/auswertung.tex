\section{Auswertung}
\label{sec:Auswertung}

\subsection{Bestimmung der Wellenlänge eines Lasers}
\begin{table}[H]
  \centering
  \caption{Gezählte Maxima bei vorwärts und rückwärts drehendem Motor.}
  \begin{tabular}{c c}
      \toprule
      vor & zurück\\
      \midrule
      2403 & 2730\\
      2533 & 2720\\
      2434 & 2689\\
      2435 & 2693\\
      2453 & 2987\\
      \bottomrule
  \end{tabular}
  \label{tab:wellenleange}
\end{table}
Die Messdaten bei Verschiebung des Spiegels sind in \autoref{tab:wellenleange} aufgeführt. Da deutliche Unterschiede bei der Zählung der 
Maxima je nach Drehrichtung des Motors aufgetreten sind, wurde hier zwischen vor und zurück unterschieden. Der Motor bewegt sich um 
$\Delta_{motor} = \SI{5(0.1)}{mm}$. Durch die Hebelübersetzung von 1:5.046 bewegt sich der Spiegel also absolut um $\Delta s = \SI{9.91(0.2)}{mm}$.
\\
\\
Aus den Werten in \autoref{tab:wellenleange} wird jeweils ein Mittelwert $\overline{vor}$ bzw. $\overline{zurueck}$ gebildet. Ein dabei auftretender Fehler wird aufgrund von 
systematischen Fehlern (z.B. wenn der Motor nicht rechtzeitig gestoppt wird oder bei Erschütterungen der Messapperatur) auf $vor_{err} = \pm 100$ und $zurueck_{err} = \pm 50$ festgelegt.
So ergibt sich 
\begin{align*}
  \overline{v} &= \SI{2763.8(100)}{}\\
  \text{und } \overline{z}\, &= \SI{2451.6(50)}{}.
\end{align*}
Der Fehler des Motors wird auf $motor_{err} = \pm 0.1 \unit{mm}$ festgelegt. Einerseits ist dies mit einem verzögerten Ausschalten des Motors sowie einem nicht
ganz sauberen laufenden Motors zu begründen.
\\
\\
Mithilfe dieser Werte kann dann nach \autoref{eq:} die Wellenlänge des Lasers bestimmt werden. Auch hier wird aufgrund deutlicher Abweichungen je nach Drehrichtung
ein $\lambda_{v}$ und $\lambda_{z}$ angegeben.
Es ergibt sich 
\begin{align*}
  \lambda_{v} &= \SI{717(14)}{nm}\\
  \text{und } \lambda_{z} &= \SI{808(16)}{nm}.
\end{align*}

\subsection{}

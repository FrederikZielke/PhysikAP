\section{Zielsetzung}
\label{sec:Zielsetzung}

Im Versuch 308 werden die Magnetfelder unterschiedlicher Spulen untersucht. Es wird der Einfluss von Windungszahl, 
Stromstärke, Spulenkern und Abstand zu den Spulen untersucht.\\

\section{Theorie}
\label{sec:Theorie}

\begin{equation*}
    \vec{B} = μ \cdot \vec{H}
\end{equation*}

\begin{equation*}
    d\vec{B} = \frac{μ_0I}{4π} \frac{d\vec{s} \times \vec{r}}{r^3}
\end{equation*}

\begin{equation*}
    \vec{B}(x) = \frac{μ_0I}{2} \frac{R^2}{(R^2 + x^2)^{3/2}} \cdot \hat{x}
\end{equation*}

\begin{equation*}
    B = μ_{\symup{r}}μ\frac{n}{l}I
\end{equation*}

\begin{equation*}
    B = μ_{\symup{r}}μ_0\frac{n}{2πr_{\symup{T}}}I
\end{equation*}

\begin{equation*}\label{eq:helmholz}
    B(0) = B_1(x) + B_1(-x) = \frac{μ_0IR^2}{(R^2 + x^2)^{3/2}} + \frac{μ_0IR^2}{(R^2 + x^2)^{3/2}}
\end{equation*}

\begin{equation*}
    \frac{dB}{dx} = -3μ_0IR^2\frac{x}{(R^2 + x^2)^{5/2}}
\end{equation*}

\begin{equation*}
    μ_{\text{dif}} = \frac{1}{μ_0}\frac{dB}{dH}
\end{equation*}

\begin{equation*}
    \vec{B} = μ_0(\vec{H} + \vec{M})
\end{equation*}

%\cite{sample}

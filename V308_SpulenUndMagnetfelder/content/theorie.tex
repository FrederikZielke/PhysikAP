\section{Zielsetzung}
\label{sec:Zielsetzung}

Im Versuch 308 werden die Magnetfelder unterschiedlicher Spulen untersucht. Es wird der Einfluss von Windungszahl, 
Stromstärke, Spulenkern und Abstand zu den Spulen untersucht.\\

\section{Theorie}
\label{sec:Theorie}
Das magnetische Feld ist eine wichtige Größe in der Elektromagnetismus-Theorie. 
Es entsteht durch die Bewegung von elektrischen Ladungen und kann von Magneten oder elektrischen Strömen erzeugt werden.
Das magnetische Feld ist eine Vektorgröße, die durch die magnetische Feldstärke $H$ beschrieben wird. 
Die Feldstärke ist ein Maß für die Intensität des magnetischen Feldes und gibt an, wie stark das magnetische Feld an einem bestimmten Punkt ist. 
Die Magnetfeldlinien sind gedachte Linien, die tangential zu den Feldstärkevektoren verlaufen und das magnetische Feld veranschaulichen. 
Sie sind geschlossen und verlaufen von Nord- zu Südpol. Sie zeigen die Richtung des magnetischen Feldes an und geben Auskunft über die Stärke des Feldes an verschiedenen Punkten. 
Die Dichte der Magnetfeldlinien gibt an, wie stark das magnetische Feld an einem bestimmten Punkt ist. Je dichter die Linien, desto stärker ist das magnetische Feld.
\\
Wenn die Wärmebewegung statistisch verteilt ist, ist
\begin{equation}\label{eq:BmuH}
    \vec{B} = \mu_r \mu_0 \vec{H} = μ \cdot \vec{H}
\end{equation}
der Zusammenhang zwischen magnetischer Feldstärke $\vec{H}$ und magnetischer Flussdichte $\vec{B}$. Dabei ist $\mu_0$ die Permeabilität im Vakuum und $\mu_r$ die
stoffspezifische relative Permeabilität.
Mit dem Biosavart-Gesetz 
\begin{equation}\label{Biosavart}
    d\vec{B} = \frac{μ_0I}{4π} \frac{d\vec{s} \times \vec{r}}{r^3}
\end{equation}
lässt sich die magnetische Flussdichte eines stromdurchflossenen Leiters bestimmen. Hierbei ist $\vec{r}$ der Ortsvektor, $I$ die Stromstärke und $d\vec{s}$ ein 
infinitesimales Wegstück. 
Mit Gleichung \eqref{Biosavart} ergibt sich für eine Spule mit $n$-Windungen 
\begin{equation*}
    \vec{B}(x) = n \cdot \frac{μ_0I}{2} \frac{R^2}{(R^2 + x^2)^{3/2}} \cdot \hat{x},
\end{equation*}
wobei $n$ die Windungszahl, $R$ der Spulenradius und $x$ der Abstand zum Spulenzentrum ist.
In einer langen Spule $(l >> D = 2R)$ ist die magnetische Feldstärke in der Mitte der Spule konstant und homogen. Außerhalb ist das Feld inhomogen, da die Feldlinien
dort auseinander gehen. Sind Randeffekte zu vernachlässigen, ergibt sich für eine lange Spule
\begin{equation*}
    B = μ_{\symup{r}}μ\frac{n}{l}I.
\end{equation*}
Das Feld $B$ ist also proportional zu der Windungszahl $n$, der Länge der Spule $l$ und dem Spulenstrom $I$.
\\
Aus einer langen Spule kann eine Ringspule mit $r_{\symup{T}} << l$ geformt werden. Hierbei verschwinden die Randeffekte. Das besondere ist, dass bei einer Ringspule außerhalb der Spule
kein Magnetfeld existiert. Das Magnetfeld im Inneren ist homogen und lässt sich mit $l = 2 \pi r_{\symup{T}}$ mit 
\begin{equation}\label{toroidspule}
    B = μ_{\symup{r}}μ_0\frac{n}{2πr_{\symup{T}}}I
\end{equation}
berechnen.
\\
Werden zwei Kreisspulen ausgerichtet, so dass die Achsen aufeinander liegen, werden sie als Helmholtzspulenpaar bezeichnet.
Fließt durch beide ein gleichsinniger Strom bildet sich ein auf der Symmetrieachse homogenes Magnetfeld. Das entstandene Feld
lässt sich aus dem Biot-Savart-Gesetz \eqref{Biosavart} herleiten. Allgemein lässt sich das Feld eines Helmholtzspulenpaares mit
\begin{equation*}\label{eq:helmholz}
    B(0) = B_1(x) + B_1(-x) = \frac{μ_0IR^2}{(R^2 + x^2)^{3/2}}
\end{equation*}
berechnen.\\
\\
Ferromagnetische Materialien besitzen ein permanentes magnetisches Moment. Die relative Permeabilität dieser materialien ist sehr hoch und wird
nicht durch die Gleichung \eqref{eq:BmuH} beschrieben. Die Magnetisierung wird durch die Hysteresekurve beschrieben. Wegen der irreversiblen Änderung
des Magnetfeldes ist die Kurve abhängig von der Vorgeschichte und nicht eindeutig definiert.
Zur Beschreibung der relativen Permeabilität $μ_{\symup{r}}$ ferromagnetischer Materialien wird die differentielle Permeabilität
$μ_{\text{diff}}$ eingeführt, welche im direkten Bezug zur Neukurve steht. Die Gleichung der differentiellen Permeabilität ist gegeben mit
\begin{equation}\label{eq:diff}
    μ_{\text{dif}} = \frac{1}{μ_0}\frac{dB}{dH}.
\end{equation}\\
\\
Der magnetische Fluß einer Spule erhöht sich wenn sie mit einem ferromagnetischem Kern versehen wird. Die Magnetisierung $M$ des Materials
bestimmt die Erhöhung der Feldstärke $\vec{B}_{\text{FE}} = μ_0\vec{M}.$ Der Magnetische Fluß $\vec{B}$ einer Spule mit Eisenkern kann mit 
\begin{equation}\label{eq:spuleeisenkern}
    \vec{B} = μ_0(\vec{H} + \vec{M})
\end{equation}
berechnet werden.\\
\\
Mit $\vec{H}_{\symup{R}}$ wird die Feldstärke von Randeffekten in der gesamten magnetischen Feldstärke $\vec{H} = \vec{H}_0 + \vec{H}_{\symup{R}}$ einer Spule berücksichtigt.
Wird ein Toroid mit der Spule geformt, gilt $\vec{H} = \vec{H}_0 = \vec{B} / μ_0.$ Die relative Permeabilität ferromagnetischer Materialien liegen in der Größenordnung von $10^2 \text{bis} 10^7$. Unter dieser Voraussetzung lässt sich die 
Gleichung \eqref{eq:spuleeisenkern} vereinfachen zu $\vec{B} \approx  μ_0\vec{M}.$\\

%Diese Glecihungen wurden nicht gebraucht aber ich lass sie mal stehen man weiß ja nie
%\begin{equation*}
%    \vec{B} = μ \cdot \vec{H}
%\end{equation*}
%
%\begin{equation*}
%    d\vec{B} = \frac{μ_0I}{4π} \frac{d\vec{s} \times \vec{r}}{r^3}
%\end{equation*}
%
%\begin{equation*}
%    \vec{B}(x) = \frac{μ_0I}{2} \frac{R^2}{(R^2 + x^2)^{3/2}} \cdot \hat{x}
%\end{equation*}
%
%\begin{equation*}
%    B = μ_{\symup{r}}μ\frac{n}{l}I
%\end{equation*}
%
%\begin{equation*}
%    B = μ_{\symup{r}}μ_0\frac{n}{2πr_{\symup{T}}}I
%\end{equation*}
%
%
%
%\begin{equation*}
%    \frac{dB}{dx} = -3μ_0IR^2\frac{x}{(R^2 + x^2)^{5/2}}
%\end{equation*}
%
%
%
%
%
%%\cite{sample}
%
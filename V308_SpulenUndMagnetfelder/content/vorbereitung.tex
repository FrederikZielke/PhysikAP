\section{Vorbereitungsaufgaben}
\label{sec:Vorbereitungsaufgaben}

Es gibt folgende Arten von Stoffmagnetismus:
\subsection{Diamagnetismus}
    Die Atome oder Moleküle diamagnetischer Stoffe besitzen kein permanentes magnetisches Dipolmoment.
    Wirkt ein magnetisches Feld auf einen diamanetischen Soff entstehen induzierte Dipole. Diese wirken
    dem äußeren Feld entgegen und schwächen es so inerhalb des Stoffes ab \cite[109--114]{Demtröder}.
\\
\subsection{Paramagnetismus}
    Die Atome paramagnetischer Stoffe besitzen in alle Richtungen gerichtete permanente magnetische Dipole.
    Die gemittelte Magnetisierung $M= \frac{1}{V}\sum p_m$ ist null. Wirkt ein Magnetfeld von außen werden 
    die Dipole teilweise ausgerichtet. Das äußere magnetische Feld wird im inneres des Stoffes verstärkt\cite[109--114]{Demtröder}.
\\
\subsection{Ferromagnetismus}
    Ferromagnetische Stoffe bestehen aus Atomen mit magnetischen Dipolen. Die Dipole sind in Weiß'schen Bezirken
    geordnet. Durch ein äußeres Feld können diese unterschiedlcih gerichteten Bezirke parallel gestellt werden. 
    Sind alle Bezirke ausgerichtet spricht man von Sättigung. Wird das äußere Feld entfernt, bleibt eine Restmagnetisierung
    bestehen\cite[109--114]{Demtröder} \cite[208--210]{Nolting}.
\\
\subsection{Berechnung des Magnetfeldes}
Berechnung des Magnetfeldes innerhalb eines Helmholzspulenpaars:
Gegeben ist ein Spulenstrom von $I = 1 \symup{A}$, einem Spulendurchmesser von $d = \SI{125}{\milli\meter}$ und $n = 100$ Windungen.
Das Magnetfeld wird mit Formel \ref{eq:helmholz} berechnet. Um die Anzahl der Windungen zu berücksichtigen wird die Gleichung mit $n$ multipliziert. 
Der Spulenradius ist $\frac{d}{2} = \SI{31.25}{\milli\meter}$ und der optimale Abstand zwischen den Spulen ist $\frac{r}{2} = x = \SI{62.5}{\milli\meter}$.
\begin{equation*}
    B(0) = \frac{4π\cdot10^{-7}\cdot1\symup{A}(0.0625\symup{m})^2}{((0.0625\symup{m})^2)^{\frac{3}{2}}} \cdot 100 = \SI{1.4}{\milli\tesla}.
\end{equation*}
\newpage
\section{Vorbereitungsaufgaben}
\label{sec:Vorbereitungsaufgaben}

Es gibt folgende Arten von Stoffmagnetismus:
\begin{itemize}%Wir sollten uns noch bessere Erklärungen überlegen als das was wir bisher haben
    \item Ferromagnetismus
    \item Diamagnetismus
    \item Paramagnetismus
\end{itemize}

Berechnung des Magnetfeldes innerhalb eines Helmholzspulenpaars:
Gegeben ist ein Spulenstrom von $I = 1 \symup{A}$, einem Spulendurchmesser von $d = \SI{125}{\milli\meter}$ und $n = 100$ Windungen.
Das Magnetfeld wird mit Formel \ref{eq:helmholz} berechnet. Um die Anzahl der Windungen zu berücksichtigen wird die Gleichung mit $n$ multipliziert. 
Der Spulenradius ist $\frac{d}{2} = \SI{31.25}{\milli\meter}$ und der optimale Abstand zwischen den Spulen ist $\frac{r}{2} = x = \SI{62.5}{\milli\meter}$.
\begin{equation*}
    B(0) = \frac{4π\cdot10^{-7}\cdot1\symup{A}(0.0625\symup{m})^2}{((0.0625\symup{m})^2)^{\frac{3}{2}}} \cdot 100 = \SI{1.4}{\milli\tesla}.
\end{equation*}
\newpage
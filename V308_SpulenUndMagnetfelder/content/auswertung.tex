\section{Auswertung}
\label{sec:Auswertung}
Die in Abschnitt \autoref{sec:Auswertung} gezeigten Grafiken sind mithilfe der Python-Bibliotheken Matplotlib \cite{matplotlib} und Numpy \cite{numpy}
erstellt worden.
\subsection{Magnetische Feldstärke einer langen und kurzen Spule in Abhängigkeit der Position auf der x-Achse}

Die aufgenommenen Messwerte für die kurze und lange Spule sind in \autoref{kurzeSpule} und \autoref{langeSpule} dargestellt.
\begin{figure}[H]
    \includegraphics[width=\linewidth]{build/kurzeSpule.pdf}
    \caption{Magnetische Feldstärke in Abhängigkeit der x-Position einer kurzen Spule.}
    \label{kurzeSpule}
\end{figure}
Die magnetische Feldstärke wird zum Mittelpunkt der kurzen Spule hin maximal und nimmt mit zunehmendem Abstand nach außen ab. 
Der Randpunkt der Spule ist in dem Graphen bei $\approx \SI{4}{cm}$ zu lokalisieren. Hier ist eine Änderung des Abfalls zu erkennen.

\begin{figure}[H]
    \includegraphics[width=\linewidth]{build/langeSpule.pdf}
    \caption{Magnetische Feldstärke in Abhängigkeit der x-Position einer langen Spule.}
    \label{langeSpule}
\end{figure}
Bei der langen Spule zeigt sich ein ähnliches Verhalten. Die magnetische Feldstärke nimmt innerhalb der Spule nach außen hin ab. Am Randpunkt, hier bei 
$\approx \SI{7}{cm}$ zu lokalisieren, ändert sich das Abnehmverhalten ebenfalls. Es flacht ab. 
\\
\\
Bei beiden Spulen ist zu erkennen, dass an den Randpunkten ein Wendepunkt vorliegt. In dem Fall ist dort der stärkste Abfall der magnetischen
Feldstärke zu beobachten.
\newpage
\subsection{Helmholtzspulenpaar}
Die Messwerte für die drei Konfigurationen sind in \autoref{helmholtz6}, \autoref{helmholtz12} und \autoref{helmholtz23} dargestellt.

\begin{figure}[H]
    \includegraphics[width=\linewidth]{build/helmholtz6.pdf}
    \caption{Magnetische Feldstärke in Abhängigkeit der x-Position eines Helmholtzspulenpaares mit Abstand $d =\SI{6}{cm}$.}
    \label{helmholtz6}
\end{figure}
Bei der Konfiguration $d = \SI{6}{cm}$ ist ein Abfall außerhalb des Spulenpaares zu erkennen.

\begin{figure}[H]
    \includegraphics[width=\linewidth]{build/helmholtz12.pdf}
    \caption{Magnetische Feldstärke in Abhängigkeit der x-Position eines Helmholtzspulenpaares mit Abstand $d =\SI{12}{cm}$.}
    \label{helmholtz12}
\end{figure}
Die Konfiguration mit $d = \SI{12}{cm}$ zeigt ein lokales Minimum in der Mitte des Spulenpaares bei $x \approx \SI{4}{cm}$. Weil die Spulen weiter auseinander sind
überlagern sich die Feldstärken der einzelnen Spulen nicht mehr stark genug, um ein Maximum in der Mitte des Spulenpaares zu erzeugen. Die magnetische Feldstärke ist also auch insgesamt schwächer im Vergleich zur Konfiguration mit $d = \SI{6}{cm}$ (vgl. \autoref{helmholtz6}).
Daraus resultieren dann auch die Maxima an den Rändern des Spulenpaares bei $x \approx \SI{0.5}{cm}$ und $x \approx \SI{6.5}{cm}$, da die Felder der einzelnen Spulen stärker als die Überlagerung in der Mitte sind. Außerhalb des Spulenpaares, also $d > \SI{12}{cm}$, fällt die 
magnetische Feldstärke ab.

\begin{figure}[H]
    \includegraphics[width=\linewidth]{build/helmholtz23.pdf}
    \caption{Magnetische Feldstärke in Abhängigkeit der x-Position eines Helmholtzspulenpaares mit Abstand $d =\SI{23.85}{cm}$.}
    \label{helmholtz23}
\end{figure}
Die Konfiguration mit $d = \SI{23.85}{cm}$ zeigt ebenfalls ein lokales Minimum in der Mitte des Spulenpaares bei $x \approx \SI{10}{cm}$. An den Rändern sind Maxima der magnetischen Feldstärke zu beobachten.
Dies beruht auf der gleichen Begründung wie bei der Konfiguration mit $d =\SI{12}{cm}$. Außerhalb scheint die Feldstärke wieder abzufallen.

\newpage
\subsection{Hysteresekurve einer Spule mit Eisenkern}
In \autoref{hysterese} ist die Hysteresekurve einer Spule mit Eisenkern aufgetragen. Aus dieser lassen sich die Sättigungsmagnetisierung $B_S$, die Remanenz $B_r$ und
die Koerzitivkraft $H_c$ (da $H_c \propto I_c$) ablesen. Diese ergeben sich zu 
\begin{equation*}
    B_S = \SI{725}{mT}, B_r = \SI{139}{mT} \text{ und } I_c = \SI{-0.7}{A}.
\end{equation*}


\begin{figure}[H]
    \includegraphics[width=\linewidth]{build/hysterese.pdf}
    \caption{$B$-$I$-Diagramm einer Spule mit Eisenkern.}
    \label{hysterese}
\end{figure}













\newpage
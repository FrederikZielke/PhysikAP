\section{Diskussion}
\label{sec:Diskussion}

Sowohl für das Magnetfeld von Spulen, als auch für das Magnetfeld eines Spulenpaares wurden zu wenig Messwerte aufgenommen.
Dadurch können keine aussagekräftigen Schlüsse gezogen werden.\\
Bei der kurzen Spule ist kein Plateau in der magnetischen Flussdichte zu erkennen, da Randeffekte an keinem
Punkt vernachlässigt werden können. \\
Die Flussdichte der langen Spule liegt vier Zentimeter vor dem Spulenrand schon unter dem theoretisch errechneten Wert, wie auch in
\autoref{langeSpule} zu sehen. Die Theorie besagt, dass das Feld innerhalb einer langen Spule
konstant ist. Das deckt sich nicht mit den Messergebnissen. Dieser Umstand lässt sich auf Randeffekte zurückzuführen.\\
\\
Aus \autoref{helmholtz6} kann keine Homogenität des Feldes geschlossen werden, da nur ein Messwert innerhalb des
Spulenpaares aufgenommen wurde. In \autoref{helmholtz12} ist ein Peak bei $x = \SI{7}{\centi\meter}$, was der Theorie entspricht.
Trotzdem ist das lokale Maximum von $B$ nicht eindeutig an diesem Punkt. Es müssten mehr Werte aufgenommen werden um dies zu bestätigen.
In \autoref{helmholtz23} ist in mittig zwischen den Spulen ein Minimum. Dies ist auf den Durchmesser der beiden Spulen 
zurückzuführen, deckt sich also mit den erwartungen.\\
Die genaue Position der Hallsonde im Verhältnis zu den Spulen war nicht genau ermittelbar, da die Längenmarkierungen der Spulen und der Sonde
versetzt waren.\\
Die Messwerte können niedriger als die tatsächliche Flussdichte sein, da leichte Drehungen der Hallsonde nicht vermieden werden konnten.
Dies ist problematisch, da nur senkrecht zur Sonde verlaufende Feldlinien erfasst werden.
Messwerte für die Helmholtz-Konfiguration wurden nicht aufgenommen. Deshalb kann dort kein wirklicher Vergleich zu der Theorie aus \autoref{helmholtztheorie} geschlossen werden.
\\
Die Koerzitiffeldstärke der Spule mit Luftspalt wurde per Fit ermittelt und kann deswegen geringfügig abweichen, ist aber 
vermutlich sehr nahe am realen Wert.\\
Die Remanenz konnte nicht eindeutig ermittelt werden. Sie veränderte sich während des Umpolens, obwohl der Strom $I$ auf 0 gestellt war.\\
Der Kern der Spule konnte nicht restlos entmagnetisiert werden. Die Neukurve und daraus folgen die Hysteresekurve ist etwas schmaler als die typische Theoriekurve.\\
\\
Die Permeabilität bei $H = 0$ und $H = \SI{7070}{\ampere\per\meter}$ (siehe Gleichung \eqref{mudiff}) liegt ungefähr im erwarteten Bereich für ferromagnetische Materialen
von $10^2$ bis $10^7$. 


\newpage
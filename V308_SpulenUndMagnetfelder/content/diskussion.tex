\section{Diskussion}
\label{sec:Diskussion}

Sowohl für das Magnetfeld von Spulen, als auch für das Magnetfeld eines Spulenpaares wurden zu wenig Messwerte aufgenommen.
Dadurch können keine aussagekräftigen Schlüsse gezogen werden.\\
Bei der kurzen Spule ist kein Plateau in der magnetischen Flussdichte zu erkennen, da Randeffekte an keinem
Punkt vernachlässigt werden können. \\
Die Flussdichte der langen Spule liegt vier Centimeter vor dem Spulenrand schon unter dem theoretisch errechneten Wert, wie auch in
\autoref{langeSpule} zu sehen. Dieser Umstand lässt sich auf Randeffekte zurückzuführen.\\
\\
Aus autoref{hel}


%Koerzitiffeldstärke wurde durch fit ermittelt
%wir haben bei dem Spulenpaar viel zu wenig Messwerte 
%bei lange/kurze Spule haben wir auch eher zu wenige Werte
%helmholtz spule messmarkierungen x und abstand d um 3-4cm verschoben
%remanenz ist nicht eindeutig weil es springt beim umpolen
\newpage
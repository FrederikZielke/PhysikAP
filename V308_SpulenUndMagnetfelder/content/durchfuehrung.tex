\section{Versuchsaufbau und Durchführung}
\label{sec:Durchführung}

%unterschiedliche Hallsonden longitudinal und transversal???? welche war welche junge
Der Versuch ist in drei Teile aufgeteilt. In allen Aufgabenteilen werden Hallsonden zum messen der magnetischen Flußdichte verwendet.
Für den ersten Teil werden eine lange und eine kurze Spule benötigt. Eine longitransi Hallsonde wird jeweils gemessene Spule geschoben 
und in kleinen Abständen wird die magn. Flußdichte von einem Magnetfeldlesegerät??? abgelesen.\\
Im zweiten Teil des Versuches werden die magn. Flußdichten eines Spulenpaares gemessen. Eine der beiden Spulen befindet sich auf Schienen 
und kann auf diesen verschoben werden. An der oberen Schiene ist eine longitransi Hallsonde befestigt mit der die Flußdichte an beliebigen 
Stellen zwischen oder außerhalb der Spulen gemessen werden kann. Für alle drei Spulenkonfigurationen werden mehrere Messwerte aufgenommen.\\
Im dritten Teil des Versuches wird eine Hysteresekurve einer Ringspule mit Luftspalt( mit ferromagnetischem Kern?) aufgenommen. Zu erst wird der Kern(oder eher die Spule?) 
mit einer angelegten Wechselspannung entmagnetisiert. Dann wird mit einem Spannungsgerät bei voll aufgedrehtem Spannungsregler in Schritten von 
$\SI{0.5}{\ampere}$ der Stromfluss erhöht. Mit einem Teslameter wird die Flußdichte ermittelt.
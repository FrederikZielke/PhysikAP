\section{Beispiel zu Messunsicherheiten}
\label{sec:Durchführung}
\subsection{Aufgabe}
Die Standardabweichung $\sigma_u$ der Messwerte beträgt $\sigma_u = 10 \,\unit{\frac{m}{s}}$. Der Student kann nun jede beliebige Präzision erreichen, indem er genug Messungen durchführt.
Wieviele Messungen nötig sind lässt sich berechnen:
\begin{equation}
    u = \frac{1}{\sqrt{n}} \cdot \sigma_u \Leftrightarrow n = \Bigl(\frac{\sigma_u}{u}\Bigr)^2
    \label{eqn:messungen}
\end{equation}
Dabei ist $n$ die Anzahl der Messungen, $u$ die Unsicherheit und $\sigma_u$ die Standardabweichung der Messwerte.
\subsection{Berechnung}
\subsubsection{Beispiel 1}
Nun soll die Unsicherheit $\pm 3 \,\unit{\frac{m}{s}}$ betragen. Wir nutzen nun die umgestellte Formel $\eqref{eqn:messungen}$ und setzen die Werte $u = 3 \,\unit{\frac{m}{s}}$ und $\sigma_u = 10 \,\unit{\frac{m}{s}}$ ein:
\begin{equation}
    n = \Bigl(\frac{\sigma_u}{u}\Bigr)^2 = \Biggl(\frac{10 \,\unit{\frac{m}{s}}}{3 \,\unit{\frac{m}{s}}}\Biggr)^2 = 11,\bar{1}
    \label{eqn:rechnung}
\end{equation}
Wie in $\eqref{eqn:rechnung}$ berechnet, müssen also 12 Messungen durchgeführt werden, um eine Unsicherheit von unter $\pm 3 \,\unit{\frac{m}{s}}$ zu erreichen.
\subsubsection{Beispiel 2}
Die Unsicherheit soll jetzt nur noch $0,5 \,\unit{\frac{m}{s}}$ betragen.
Analog zu $\eqref{eqn:rechnung}$ berechnen wir:
\begin{equation}
    n = \Bigl(\frac{\sigma_u}{u}\Bigr)^2 = \Biggl(\frac{10 \,\unit{\frac{m}{s}}}{0,5 \,\unit{\frac{m}{s}}}\Biggr)^2 = 400
    \label{eqn:rechnung2}
\end{equation}
Um eine Unsicherheit von $0,5 \,\unit{\frac{m}{s}}$ zu erreichen, müssen 400 Messungen durchführt werden.
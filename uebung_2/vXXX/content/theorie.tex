\section{Messunsicherheiten}
\label{sec:Theorie}
    \subsection{Mittelwert}
    Der Mittelwert ist die durchschnittliche Größe einer Menge.
    \subsubsection{Berechnung des Arithmetischen Mittelwerts}
    \begin{equation}
        \bar{x}_{arithm} = \frac{1}{n}  \sum_{i=1}^n x_i = \frac{x_1 + x_2 + \cdots + x_n}{n}
        \label{eqn:arithmetischesmittel}
    \end{equation}

    \subsection{Standardabweichung}
    Die Standardabweichung gibt die Breite der Verteilung bzw. die Streuung der Messwerte um den Mittelwert an.
    \subsubsection{Standardabweichung des Mittelwerts}
    \begin{equation}
        \sigma = \sqrt{\sum_{i=1}^n (x_i - \bar{x}_{arithm})^2}
        \label{eqn:StandardabweichungM}
    \end{equation}
    \subsubsection{Empirische Standardabweichung}
    \begin{equation}
        s = \sqrt{\frac{1}{n - 1} \sum_{i=1}^n (x_i - \bar{x}_{arithm})^2}
        \label{eqn:StandardabweichungE}
    \end{equation}
    \subsection{Streuung der Messwerte und Fehler des Mittelwerts}
    Die Streuung der Messwerte gibt die Verteilung der Grundgesamtheit um den Mittelwert an $\eqref{eqn:StandardabweichungM}$, während der Fehler des Mittelwerts angibt, wie sehr sich der Mittelwert einer Stichprobe vom Mittelwert der Grundgesamtheit unterscheidet $\eqref{eqn:StandardabweichungE}$.
    





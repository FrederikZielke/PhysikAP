\section{Zielsetzung}
\label{sec:Zielsetzung}

\section{Theorie}
\label{sec:Theorie}
Der für Menschen hörbare Frequenzbereich liegt zwischen 16-20000Hz. Der Bereich unterhalb des hörbaren Bereichs heißt Infraschall. Ultraschall ist der Bereich von 
$\SI{20}{kHz}$ bis $\SI{1}{GHz}$. Oberhalb von $\SI{1}{GHz}$ spricht man von Hyperschall. Ultraschall findet in der Medizin und Werkstoffprüftechnik Anwendung. So kann jener
genutzt werden, um Auskunft über die zu untersuchende Probe zu gewinnen ohne diese zu zerstören.\\
Mathematisch lässt sich Schall als eine longitudinale Welle beschreiben, die sich aufgrund von Druckänderungen durch ein Medium bewegt. Die Beschreibung hat dann die Form
\begin{equation*}
    p(x, t) = p_0 + v_0 Z cos(\omega t - kx).
\end{equation*}
Dabei entspricht $Z = c_M \rho_M$ der akustischen Impedanz. Die Größen $c_M$ und $\rho_M$ sind die Schallgeschwindigkeit und Dichte im Ausbreitungsmedium. Beide sind also
stoffspezifisch.
In Gasen und Flüssigkeiten ist Schall immer eine Longitudinalwelle. Die Schallgeschwindigkeit hängt da von der Dichte $\rho$ und der Kompressibilität $\kappa$ ab. Es gilt:
\begin{equation*}
    c_{Flüssigkeit} = \sqrt{\frac{1}{\kappa \rho}}.
\end{equation*}
In Festkörpern sind aufgrund von Schubspannungen neben Longitudinalwellen auch Transversalwellen möglich. Hier bestimmt das Elastizitätsmodul $E$ die Schallgeschwindigkeit
im Festkörpern, sodass
\begin{equation*}
    c_{Festkörper} = \sqrt{\frac{E}{\rho}}
\end{equation*}
gilt.
Die Schallgeschwindigkeit der Longitudinalwelle und Transversalwelle kann dabei unterschiedlich groß sein, da Schallgeschwindigkeiten in Festkörpern richtungsabhängig sind.
Durch Absorption nimmt die Intensität der Schallwelle ab. Die Intensität wird dabei mit der Strecke $x$ im Medium skaliert:
\begin{equation*}
    I(x) = I_0 exp(-\alpha x).
\end{equation*}
Dabei ist $I_0$ die anfängliche Intensität und $\alpha$ der Absorptionskoeffizient des Mediums. Bei der Verwendung von Ultraschall nimmt die Amplitude stark durch das 
Durchlaufen von Luft ab. Deshalb wird in der Regel ein Kontaktmittel zwischen Sonde und dem zu untersuchenden Objekt genutzt.\\
\\
Die Ausbreitung von Schall ist ähnlich der einer elektromagnetischen Welle. Bei Grenzflächen zwischen zwei Medien kommt es zur Reflexion und Transmission. Die Welle spaltet 
sich also in einen reflektierten und transmitierten Teil auf. Die Intensität der reflektierten Welle lässt sich mit dem Reflexionskoeffizienten
\begin{equation*}
    R = (\frac{Z_1 - Z_2}{Z_1 + Z_2})^2
\end{equation*}
beschreiben. $Z_1$ und $Z_2$ sind die akustischen Impedanzen des Mediums vor und nach der Grenzfläche. Mit 
\begin{equation*}
    T + R = 1 \Leftrightarrow T = 1 - R
\end{equation*}
lässt sich die transmitierte Intensität berechnen.
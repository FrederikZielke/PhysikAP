\section{Auswertung}
\label{sec:Auswertung}

%Hier noch irgendeinen Text einfuegen lol oder auch nicht hahahahahahahaha

\subsection{Vermessung des Acrylblocks mit Schieblehre}
\label{subsec:Schieblehre}


\begin{table}[!ht]
    \centering
    \begin{tabular}{|c|c|c|c|}
        \toprule
        {Lochnummer} & {Abstand zur Oberseite / [mm]} & {Abstand zur Unterseite / [mm]} & {Durchmesser / [mm]} \\
        \midrule
        1 & 59.00 \pm\, 0.04 & 19.10 \pm\, 0.04 & - \\
        2 & 60.82 \pm\, 0.04 & 17.46 \pm\, 0.04 & - \\
        3 & 13.08 \pm\, 0.04 & 60.64 \pm\, 0.04 & 5.92 \pm 0.06\\
        4 & 21.70 \pm\, 0.04 & 53.06 \pm\, 0.04 & 5.00 \pm 0.06\\
        5 & 30.26 \pm\, 0.04 & 45.46 \pm\, 0.04 & 3.92 \pm 0.06\\
        6 & 38.66 \pm\, 0.04 & 37.90 \pm\, 0.04 & 3.10 \pm 0.06\\
        7 & 46.70 \pm\, 0.04 & 29.86 \pm\, 0.04 & 3.10 \pm 0.06\\
        8 & 54.64 \pm\, 0.04 & 21.86 \pm\, 0.04 & 3.06 \pm 0.06\\
        9 & 62.70 \pm\, 0.04 & 13.70 \pm\, 0.04 & 3.04 \pm 0.06\\
        10 & 70.64 \pm\, 0.04 & 05.78 \pm\, 0.04 & 3.10 \pm 0.06\\
        11 & 15.20 \pm\, 0.04 & 54.24 \pm\, 0.04 & 10.00 \pm 0.06\\
        \bottomrule
    \end{tabular}
    \caption{Ausmessung der Löcher im Acrylblock mit einer Schieblehre.}
    \label{tab:Schieblehre}
\end{table}

Zur Bestimmung der Schallgeschwindigkeit %Und weitern Sachen?
wird der Acrylblock mit einer Schieblehre vermessen. Die Gesamthöhe des Blocks beträgt $h = \SI{79,76}{mm}.$
Die Messwerte sind in Tabelle \ref{tab:Schieblehre} aufgelistet. 
Für alle Löcher wird der Abstand zur Ober- und Unterseite des Blocks gemessen.
Der Durchmesser der Löcher 3 bis 11 wird vermessen.
Die Durchmesser der Löcher 1 und 2 werden nicht weiter betrachtet, da sie zu klein sind um mit der Schieblehre vermessen zu werden.
Die Fehler der Messwerte werden mit $\Delta x = \SI{0,04}{mm}$ angenommen, was zwei Nonius-Einheiten entspricht.
Der Fehler für die Vermessung des Durchmessers wird etwas größer angenommen, da das genaue Ansetzen der Schieblehre nicht immer möglich war.


\subsection{Bestimmung der Schallgeschwindigkeit in Acryl}
\label{sec:Schallgeschwindigkeit}

% Grafik build/AkrylGeschwindigkeit.pdf einfuegen
\begin{figure}[!ht]
    \centering
    \includegraphics[width=\textwidth]{build/AkrylGeschwindigkeit.pdf}
    \caption{Bestimmung der Schallgeschwindigkeit in Acryl.}
    \label{fig:AkryllGeschwindigkeit}
\end{figure}

\begin{table}
    \centering
    \begin{tabular}{|c|c|c|}
        \toprule
        {Lochnummer} & {Laufzeit / [µs]} & {Höhe / [mm]} \\
        \midrule
        3 & 45.4 & 60.64\\
        4 & 39.5 & 53.06\\
        5 & 34.2 & 45.46\\
        6 & 28.6 & 37.90\\
        7 & 22.6 & 29.86\\
        8 & 19.6 & 21.86\\
        9 & 11.1 & 13.70\\
        11 & 40.4 & 54.24\\
        \bottomrule
    \end{tabular}
\end{table}

Zur Bestimmung der Schallgeschwindigkeit in Acryl werden die Laufzeiten zu sieben Löchern gemessen und gegen
die Höhe der Löcher aufgetragen. Die Schallgeschwindigkeit wird dann aus der Steigung der Ausgleichsgeraden bestimmt.
Die Dicke der Anpassungsschicht entspricht der Verschiebung der Ausgleichsgeraden in der y-Richtung.
Die Messwerte sind in Tabelle \ref{tab:Schallgeschwindigkeit} aufgelistet.
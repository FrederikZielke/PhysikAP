\section{Diskussion}
\label{sec:Diskussion}

Alle Abweichungen werden mit 
\begin{equation*}
    \Delta = |\frac{exp - theo}{theo} \cdot 100|
\end{equation*}
berechnet.
% Abweichung von Schallgeschwindigkeit zu Literaturwert
% moegliche gruende
Die im ersten Teil ermittelte Schallgeschwindigkeit $c_{\text{Acryl}} = 2824 ± 84.7 \si{\meter\per\second},$
hat eine Abweichung von $Δc = 3.48\%$ zum Literaturwert \cite{Schallgeschwindigkeit von Acryl}
Diese Abweichung ist auf die nicht auf der Ausgleichsgeraden liegende Laufzeit von Loch 9 zurückführbar.

% Abweichung der ermittelten Durchmesser mit Schieblehremessung als Referenz

\begin{table}
    \centering
    \caption{Prozentuale Abweichung der mit A-\\ und B-Scan ermittelten Durchmessern\\zur Messung der Schieblehre}
    \begin{tabular}{|c|c|c|}
        \toprule
        {Loch} & {A-Scan} & {B-Scan}\\
        \midrule
        3 & 23\pm \,24\% & 296\pm \,24\%\\
        4 & 27\pm \,28\% & 311\pm \,28\%\\
        5 & 30\pm \,40\% & 350\pm \,40\%\\
        6 & 30\pm \,50\% & 370\pm \,50\%\\
        7 & 20\pm \,50\% & 300\pm \,50\%\\
        8 & 50\pm \,50\% & 170\pm \,50\%\\
        9 & 30\pm \,50\% & 180\pm \,50\%\\
        11 & 10\pm \,14\% & 161\pm \,14\%\\
        \bottomrule
    \end{tabular}
\end{table}

Die gemessenen Durchmesser der Löcher weichen besonders beim B-Scan stark ab.
Die Grafische auswertung ist dabei als die Hauptursache anzunehmen.
Die genaue Identifizierung der Lage der Fehlstellen ist nur schwer möglich.
Durch das ablesen mithilfe des Cursors entsteht weitere Ungenauigkeit.
Die Genauigkeit des A-Scans ist höher als die des B-Scans, aber ermöglicht keine so gute intuitive Interpretation der Ergebnisse.
\\
\\
Die Untersuchung des Brustmodells ist nicht besonders aussagekräftig, da eine korrekte Einordnung der Ergebnisse
ohne Vorwissen kaum möglich ist.
Des weiteren ist die genaue Form und Größe mit dem verwendeten Verfahren nicht ermittelbar.
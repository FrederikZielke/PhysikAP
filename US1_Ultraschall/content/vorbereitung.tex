\section{Vorbereitungsaufgaben}
\label{sec:Vorbereitungsaufgaben}

Die Schallgeschwindigkeit ist abhängig vom Material. Die Werte in \autoref{tab:c_schall} sind Literaturwerte, bei $T = \SI{20}{°C}$ und Normaldruck.
\begin{table}[H]
    \centering
    \caption{Schallgeschwindigkeit in verschiedenen Medien.}
    \begin{tabular}{c c}
        \toprule
        {Material} & {$c /\symup{\frac{m}{s}}$}\\
        \midrule
        Luft   & $\SI{331.5}{}{}$\cite{acryl}\\
        Wasser & $\SI{1485}{}$\cite{luftwasser}\\
        Akryl  & $\SI{2730}{}$\cite{luftwasser}\\
        \bottomrule
    \end{tabular}
    \label{tab:c_schall}
\end{table}

Da in dem Versuch ein Akrylblock mit verschiedene Sonden untersucht wird, sind die Wellenlängen und Periodendauern der verschiedenen Frequenzen interessant.
Die Wellenlängen ergeben sich mit
\begin{equation*}
    \lambda = \frac{c}{f}
\end{equation*}
und die Periodendauern mit 
\begin{equation*}
    T = \frac{1}{f}.
\end{equation*}

Folglich ergeben sich die Werte in \autoref{tab:vor}.
\begin{table}[H]
    \centering
    \caption{Frequenz $f$, Wellenlänge $\lambda$ und Periodendauer $T$.}
    \begin{tabular}{c c c}
        \toprule
        {$f/\symup{MHz}$} & {$\lambda /\symup{m}$} & {$T /\symup{s}$}\\
        \midrule
        1 & 0.00273 & $10^{-6}$ \\
        2 & 0.001365 & $2 \cdot 10^{-6}$\\
        4 & 0.0006825 & $4 \cdot 10^{-6}$\\
        \bottomrule
    \end{tabular}
    \label{tab:vor}
\end{table}

\section{Zielsetzung}
\label{sec:Zielsetzung}

Mit dem Versuch 206 Wärmepumpe soll gezeigt werden, dass die Richtung des Wärmeflusses mithilfe zusätzliche aufgewändeter
Energie umgekehrt werden kann.


\section{Theorie}
\label{sec:Theorie}
Nach dem zweiten Hauptsatz der Thermodynamik kann Wärme nicht von sich aus von einem Körper geringerer Temperatur auf einen Körper 
mit höherer Temperatur wechseln. Nur durch Einsatz zusätzlicher Energie lässt sich die Fließrichtung (von Wärme) verändern.
Eine Apparatur die dies ermöglicht wird Wärmepumpe genannt.\\
Weiter folgt aus dem ersten Hauptsatz der Thermodynamik, dass die Wärmemenge $Q_2$ die aus dem gekühlten Reservoir entnommen wird,
sowie die dafür aufgewendete Arbeit $A$ gleich der an das wärmere Reservoir gegebene Wärme $Q_1$ sein muss:

\begin{equation}\label{eqn:hauptsatz1}
    Q_1 = Q_2 + A
\end{equation}

Aus dem Verhältnis der Arbeit und der Erwärmung lässt sich die Güteziffer einer Wärmepumpe berechnen:

\begin{equation}\label{eqn:gueteziffer}
    ν = \frac{Q_1}{A}
\end{equation}
%Hier fehlt noch eine Erklarung fur den Ursprung der folgenden Formel >:(
\begin{equation}\label{eqn:QVerhaeltnis}
    \frac{Q_1}{T_1} - \frac{Q_2}{T_2} = 0
\end{equation}
Damit die Gleichung \eqref{eqn:QVerhaeltnis} gilt muss die Wärmeübertragung reversibel sein. 
Da diese Annahme in der Realität nicht vollständig erfüllt wird kommen wir zu folgender Ungleichung:
\begin{equation}
    \frac{Q_1}{T_1} - \frac{Q_2}{T_2} > 0
\end{equation}
Mit Gleichung \eqref{eqn:gueteziffer} folgt dann für die ideale Güteziffer 
\begin{equation}\label{eqn:guetezfferIdeal}
    ν_{id} = \frac{Q_1}{A} = \frac{T_1}{T_1 - T_2}.
\end{equation}
Durch Umformung lässt sich ein anderer Ausdruck für die reale Güteziffer einer Wärmepumpe aufstellen:
\begin{equation}\label{eqn:guetezifferReal}
    ν_{real} < \frac{T_1}{T_1 - T_2}
\end{equation}
Die Gleichung zeigt, dass die Wärmepumpe besser arbeitet, je geringer die Temperaturdifferenz zwischen den 
Reservoiren ist.\\
Verfahren bei denen mechanische Arbeit in Wärmeenergie umgewandelt wird können maximal so viel Wärme erzeugen,
wie Arbeit aufgewandt wird $Q_{1_{direkt}} \leq A$. 
Die Wärmepumpe kann im Gegensatz dazu je nach Temperaturdifferenz der beiden Reservoire weit mehr Wärme
zur Verfügung stellen:
\begin{equation}
    Q_{1_{rev}} = A\, \frac{T_1}{T_1 - T_2}
\end{equation}

Der Differenzquotient $\frac{ΔQ_1}{Δt}$ ist gegeben durch 
\begin{equation}
    \label{delQ1}
    \frac{ΔQ_1}{Δt} = \left(m_1c_W + m_kc_k\right)\frac{ΔT_1}{Δt}.
\end{equation}
Daraus folgt dann für die reale Güteziffer
\begin{equation}
    \label{vreal}
    ν = \frac{ΔQ_1}{ΔtN}.
\end{equation}
wobei N die vom Wattmeter angezeigte Leistung ist.

Der Differenzquotient $\frac{ΔQ_2}{Δt}$ ist gegeben durch
\begin{equation}\label{eqn:diffQ2}
    \frac{ΔQ_2}{Δt} = \left(m_2c_W + m_kc_k\right)\frac{ΔT_2}{Δt}
\end{equation}
Mit dem Differenzquotienten $\frac{ΔQ_2}{Δt}$ und dem Zusammenhang
\begin{equation}\label{eqn:massendurchsatz}
    \frac{ΔQ_2}{Δt} = L\frac{Δm}{Δt} \Leftrightarrow \frac{Δm}{Δt} = \frac{1}{L}\frac{ΔQ_2}{Δt}
\end{equation}
lässt sich mit bekannter Verdampfungswärme der Massendurchsatz berechnen.

Über das Arbeitsintegral, für die vom Kompressor geleistete Arbeit und die Poissonsche Gleichung (siehe Versuchsanleitung \cite{versuchsbeschreibung})
ergibt sich ein Ausdruck für die mechanische Kompressorleistung. Dieser ist dann gegeben durch
\begin{equation}
    \label{mechLeistung}
    N_{mech} = \frac{ΔA_m}{Δt} = \frac{1}{κ - 1}\left(p_b\sqrt[κ]{\frac{p_a}{p_b}} - p_a\right) \, \frac{ΔV_a}{Δt} 
    = \frac{1}{κ - 1}\left(p_b\sqrt[κ]{\frac{p_a}{p_b}} - p_a\right)\,\frac{1}{ρ}\,\frac{Δm}{Δt}.
\end{equation}
Dabei ist $\rho$ die Dichte des Transportmediums bei Druck $p_a$. $\rho$ lässt sich durch die ideale Gasgleichung herleiten
\begin{align}
    \frac{pV}{T} &= nR = const.\\
    \frac{p_0V_0}{T_0} &= \frac{p_2V_2}{T_2} \Leftrightarrow \frac{p_0m}{ρ_0T_0} = \frac{p_2m}{ρ_2T_2}\\
    \frac{p_0}{ρ_0T_0} &= \frac{p_2}{T_2ρ} 
\end{align}
Für $\rho$ ergibt sich dann
\begin{equation}\label{eqn:roh}
    ρ = \frac{ρ_0T_0p_2}{T_2p_0}
\end{equation}
\section{Auswertung}
\label{sec:Auswertung}

\subsection{Messwerte}
Die aufgenommenen Messwerte sind in \autoref{hallo} dargestellt.
\begin{table}
\centering
\pgfkeys{/pgf/number format/.cd,
fixed,
fixed zerofill,
precision=1,
set thousands separator={}}
\pgfplotstabletypeset[
columns/0/.style={column name={$t\,[\unit{s}]$}},
columns/1/.style={column name={$T_2\,[\unit{°C}]$}},
columns/2/.style={column name={$T_1\,[\unit{°C}]$}},
columns/3/.style={column name={$p_{\text{a}}\,[\unit{Bar}]$}},
columns/4/.style={column name={$p_{\text{b}}\,[\unit{Bar}]$}},
columns/5/.style={column name={$N\,[\unit{W}]$}},
every head row/.style={
before row=\toprule,after row=\midrule},
every last row/.style={
after row=\bottomrule}
]
{content/messwerte.txt}
\caption{Aufgenommene Messwerte $t$, $T_2$, $T_1$, $p_{\text{a}}$, $p_{\text{b}}$ und $N$.}
\label{hallo}
%\caption{Aufgenommene Messwerte $t$, $T_2$, $T_1$, $p_a$, $p_b$ und $N$.}
\end{table}
\newpage

\subsection{Temperaturverfläufe}
In \autoref{abb2} sind die Temperaturen $T_1$ und $T_2$ gegen die Zeit aufgetragen.
\begin{figure}[H]
  \centering
  \includegraphics[width=1\textwidth]{build/temperaturplots.pdf}
  \caption{Messwerte $T_1$ und $T_2$.}
  \label{abb2}
\end{figure}

\subsection{Ausgleichsgerade}
Die Ausgleichsrechnung wurde mit der Funktion $T(t) = A t^2 + B t + C$ approximiert. Mit Curvefit \cite{scipy} und Fehlerrechnung \cite{uncertainties}
wurden die Parameter A, B und C berechnet.\\
Für $T_1$ ergeben sich
\begin{align*}%C muss bei beiden korrigiert
  A_2 &= \SI{-2.6781(2.0285)e-06}{\degreeCelsius\per\square\second}\\
  B_2 &= \SI{2.8204(0.2647)e-02}{\degreeCelsius\per\second}\\
  C_2 &= \SI{19.6551(0.3657)}{\degreeCelsius}.
\end{align*}

Für $T_2$ ergeben sich
\begin{align*}
  A_1 &= \SI{7.8796(2.0285)e-6}{\degreeCelsius\per\square\second}\\
  B_1 &= \SI{-2.9734(0.264)e-2}{\degreeCelsius\per\second}\\
  C_1 &= \SI{23.9453(0.7198)}{\degreeCelsius}.
\end{align*}

\begin{figure}
  \centering
  \includegraphics[width=1\textwidth]{build/ausgleichsplot.pdf}
  \caption{Ausgleichsfit und Messdaten.}
\end{figure}

\newpage

\subsection{Berechnung der Differentialquotienten}
Die Differentialquotienten $\frac{dT_i}{dt}$ ergeben sich durch differenzieren der Fitfunktion $T(t) = At^2 + Bt + C$ und anschließendes Einsetzen der jeweiligen Zeit
in $T(t) = 2At + B$.
\begin{table}[H]
  \centering
  \begin{tabular}{
    S[table-format=4.0]
    S[table-format=2.3]
    S[table-format=2.3]
    S[table-format=3.0]
  }
    \toprule
    {$t\left[\unit{s}\right]$} & {$\frac{dT_1}{dt}\left[\unit{\frac{°C}{s}}\right]$} & {$\frac{dT_2}{dt}\left[\unit{\frac{°C}{s}}\right]$} & {$N\,\left[\unit{W}\right]$}\\
    \midrule
    240 &  {$0.0269 \pm 0.0014$}  & {$-0.0260 \pm 0.0028$} & 195\\
    540 &  {$0.0253 \pm 0.0017$}  & {$-0.0212 \pm 0.0034$} & 207\\
    840 &  {$0.0237 \pm 0.0022$}  & {$-0.0160 \pm 0.0040$} & 213\\
    1140 & {$0.0221 \pm 0.0027$}  & {$-0.0120 \pm 0.0050$} & 205\\
    \bottomrule
\end{tabular}
\caption{Differentialquotienten für vier Zeiten der zwei Messreihen.}
\end{table}

\subsection{Reale und ideale Güteziffer}
Für die Berechnung der idealen Güteziffer wird Gleichung \eqref{eqn:guetezifferReal}
verwendet.
Die reale Güteziffer lässt sich mit Gleichung \eqref{vreal}
berechnen. Mit $m_1 = 3\,\unit{kg}$, $c_w = 4200 \,\unit{\frac{J}{kg \cdot K}}$ und $m_kc_k = 750\,\unit{\frac{J}{K}}$ folgt:
\begin{table}[H]
  \centering
  \begin{tabular}{
    S[table-format=4.0]
    S[table-format=2.3]
    S[table-format=2.3]
    S[table-format=2.3]
  }
    \toprule
    {$t\left[\unit{s}\right]$} & {$v_{\text{ideal}}$} & {$v_{\text{real}}$} & {$\Delta v$}\\
    \midrule
    240 & 57.394  & {$1.84 \pm 0.10$} & 55.554\\
    540 & 12.753  & {$1.63 \pm 0.11$} & 11.123\\
    840 & 8.204   & {$1.49 \pm 0.14$} & 6.714\\
    1140 & 6.765  & {$1.44 \pm 0.18$} & 5.325\\
    \bottomrule
\end{tabular}
\caption{Ideale und reale Güteziffer für vier Zeiten und deren Abweichung}
\end{table}



\subsection{Verdampfungswärme L}
Um die Verdampfungswärme $L$ zu bestimmen wird $ln(p_b)$ gegen $1/T_1$ aufgetragen (siehe V203 \cite{V203}). Aus der Steigung der Ausgleichsgeraden ergibt sich
\begin{equation}
  m = -\frac{L}{R} \;\Leftrightarrow\; L = -m \cdot R\, ,
\end{equation}
wobei m die Steigung der Ausgleichsgeraden, $L$ die Verdampfungswärme und R die allgemeine Gaskonstante ist.

\begin{figure}
  \centering
  \includegraphics[width=1\textwidth]{build/dampfdruck.pdf}
  \caption{Ausgleichsgerade Dampfdruck.}
\end{figure}

Durch die Berechnung mit der linearen Regression mit $f(T) = m \cdot T + b$ ergibt sich $m = \SI{-2194.104(45.174)}{\kelvin}$, $b = 9.420 \pm 0.146$. Mit $R = 8.314 \, \unit{\frac{J}{mol \cdot K}}$ ergibt sich
\begin{equation}
  \Rightarrow L = -m \cdot R = 150.9 \pm 3.1 \, \unit{\frac{J}{g}}.
\end{equation}

\newpage

\subsection{Massendurchsatz}
Für den Massendurchsatz ergeben sich nach \eqref{eqn:massendurchsatz} mit $m_2 = 3\,\unit{kg}$, $c_w = 4200 \, \unit{\frac{J}{kg \cdot K}}$, \\
$m_kc_k = 750 \, \unit{\frac{J}{K}}$ die Werte in \autoref{huhu}.
\begin{table}[H]
  \centering
  \begin{tabular}{
    S[table-format=4.0]
    S[table-format=2.3]
  }
  \toprule
  {$\frac{\Delta m}{\Delta t}(1)$} & {$-2.30 \pm 0.25 \, \unit{\frac{g}{s}}$} \\
  \addlinespace
  {$\frac{\Delta m}{\Delta t}(2)$} & {$-1.88 \pm 0.31 \, \unit{\frac{g}{s}}$} \\
  \addlinespace
  {$\frac{\Delta m}{\Delta t}(3)$} & {$-1.50 \pm 0.40 \, \unit{\frac{g}{s}}$} \\
  \addlinespace
  {$\frac{\Delta m}{\Delta t}(4)$} & {$-1.00 \pm 0.50 \, \unit{\frac{g}{s}}$} \\
  \bottomrule
  \end{tabular}
  \caption{Massendurchsatz für vier Zeiten.}
  \label{huhu}
\end{table}

\subsection{Mechanische Kompressorleistung}
Mit dem errechneten Massendurchsatz lässt sich nun die mechanische Kompressorleistung nach \eqref{mechLeistung}
berechnen. Mit \eqref{eqn:roh}
folgt mit $\rho_0 = 5.51 \,\unit{\frac{g}{L}}$, $\kappa = 1.14$, $T_0 = 273.15 \,\unit{K}$ und $p_0 = 5.1 \,\unit{Bar}$:
\begin{table}[H]
  \centering
  \begin{tabular}{
    S[table-format=4.0]
    S[table-format=2.3]
    S[table-format=1.3]
  }
    \toprule
    {$t\left[\unit{s}\right]$} & {$\rho$} & {$|N_{mech}|\left[\unit{W}\right]$}\\
    \midrule
    240 & 35.3 & {$15.5 \pm 1.7$}\\
    540 & 44.072 & {$12.6 \pm 2.1$}\\
    840 & 51.116 & {$9.6 \pm 2.5$}\\
    1140 & 56.671 &{ $6.7 \pm 3.0$}\\
    \bottomrule
\end{tabular}
\caption{Mechanische Kompressorleistung für vier Zeiten.}
\end{table}



\newpage
%Siehe \autoref{fig:plot}!

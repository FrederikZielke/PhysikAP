\section{Auswertung}
\label{sec:Auswertung}


\subsection{Temperaturverfläufe}
In der Grafik sind die Temperaturen $T_1$ und $T_2$ gegen die Zeit aufgetragen:
\begin{figure}
  \centering
  \includegraphics[width=1\textwidth]{build/temperaturplots.pdf}
  \caption{hier könnte ihre Caption stehen}
\end{figure}

\subsection{Ausgleichsgerade}
Die Ausgleichsrechnung wurde mit der Funktion $T(t) = A t^2 + B t + C$ approximiert. Mit Curvefit und Fehlerrechnung 
wurden die Parameter A, B und C berechnet.\\
Für $T_1$:
\begin{align}
  A_1 &= \SI{7.879 \pm 999}{\degreeCelsius\per\square\minute}\\
  B_1 &= (\num{7.879e-06} \pm 999) \si{\degreeCelsius\per\square\minute}
\end{align}


Siehe \autoref{fig:plot}!

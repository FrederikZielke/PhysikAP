\section{Auswertung}
\label{sec:Auswertung}


\subsection{Temperaturverfläufe}
In der Grafik sind die Temperaturen $T_1$ und $T_2$ gegen die Zeit aufgetragen:
\begin{figure}
  \centering
  \includegraphics[width=1\textwidth]{build/temperaturplots.pdf}
  \caption{hier könnte ihre Caption stehen}
\end{figure}

\subsection{Ausgleichsgerade}
Die Ausgleichsrechnung wurde mit der Funktion $T(t) = A t^2 + B t + C$ approximiert. Mit Curvefit und Fehlerrechnung 
wurden die Parameter A, B und C berechnet.\\
Für $T_1$:
\begin{align*}
  A_1 &= \SI{7.8796(2.0285)e-6}{\degreeCelsius\per\square\minute}\\
  B_1 &= \SI{-2.9734(0.264)e-2}{\degreeCelsius\per\minute}\\
  C_1 &= \SI{23.9453(0.7198)}{\degreeCelsius}
\end{align*}

Für $T_2$:
\begin{align*}
  A_2 &= \SI{-2.6781(2.0285)e-06}{\degreeCelsius\per\square\minute}\\
  B_2 &= \SI{2.8204(0.2647)e-02}{\degreeCelsius\per\minute}\\
  C_2 &= \SI{19.6551(0.7198)}{\degreeCelsius}
\end{align*}

\begin{figure}
  \centering
  \includegraphics[width=1\textwidth]{build/ausgleichsplot.pdf}
  \caption{hier könnte ihre Caption stehen}
\end{figure}


Siehe \autoref{fig:plot}!

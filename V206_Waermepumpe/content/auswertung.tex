\section{Auswertung}
\label{sec:Auswertung}


\subsection{Temperaturverfläufe}
In der Grafik sind die Temperaturen $T_1$ und $T_2$ gegen die Zeit aufgetragen:
\begin{figure}
  \centering
  \includegraphics[width=1\textwidth]{build/temperaturplots.pdf}
  \caption{hier könnte ihre Caption stehen}
\end{figure}

\subsection{Ausgleichsgerade}
Die Ausgleichsrechnung wurde mit der Funktion $T(t) = A t^2 + B t + C$ approximiert. Mit Curvefit und Fehlerrechnung 
wurden die Parameter A, B und C berechnet.\\
Für $T_1$:
\begin{align*}
  A_2 &= \SI{-2.6781(2.0285)e-06}{\degreeCelsius\per\square\minute}\\
  B_2 &= \SI{2.8204(0.2647)e-02}{\degreeCelsius\per\minute}\\
  C_2 &= \SI{19.6551(0.7198)}{\degreeCelsius}
\end{align*}

Für $T_2$:
\begin{align*}
  A_1 &= \SI{7.8796(2.0285)e-6}{\degreeCelsius\per\square\minute}\\
  B_1 &= \SI{-2.9734(0.264)e-2}{\degreeCelsius\per\minute}\\
  C_1 &= \SI{23.9453(0.7198)}{\degreeCelsius}
\end{align*}

\begin{figure}
  \centering
  \includegraphics[width=1\textwidth]{build/ausgleichsplot.pdf}
  \caption{Ausgleichsgeraden und Messdaten}
\end{figure}

\subsection{Dampfdruckwärme L}

\begin{figure}
  \centering
  \includegraphics[width=1\textwidth]{build/dampfdruck.pdf}
  \caption{Ausgleichsgerade Dampfdruck}
\end{figure}

\subsection*{Formelsammlung}

\begin{align*}
  \frac{ΔQ_1}{Δt} &= \left(m_1c_W + m_kc_k\right)\frac{ΔT_1}{Δt}\\
  ν &= \frac{ΔQ_1}{ΔtN}
\end{align*}

\begin{align*}
 \frac{ΔQ_2}{Δt} &= \left(m_2c_W + m_kc_k\right)\frac{ΔT_2}{Δt}\\
 \frac{ΔQ_2}{Δt} &= L\frac{Δm}{Δt}
\end{align*}

\begin{align*}
  A_m &= -\int_{V_a}^{V_b}p \,dV\\
  p_aV_a^κ &= p_bV_b^κ = pV^\kappa\\
  A_m &= -p_aV_a^κ \int_{V_a}^{V_b} V^{-κ}dV = \frac{1}{κ - 1}p_aV_a^κ\left(V_b^{-κ+1} - V_a^{-κ}\right)
  = 
\end{align*}

%Siehe \autoref{fig:plot}!

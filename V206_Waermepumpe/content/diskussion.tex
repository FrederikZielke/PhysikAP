\section{Diskussion}
\label{sec:Diskussion}

Die reale Güteziffer lag $Proxente angeben$ unter den theoretischen Werten. Dies ist auf die Annahmen für $ν_{ideal}$ zurückzuführen,
welche in der Realität nicht gelten. Die Wärmeübertragung verläuft nicht vollkommen reversiel.
\\
\\
Der Massendurchsatz hat einen negativen Wert. Unsere Annahme ist, dass damit die Flussrichtung des Massendurchsatzes angegeben wird 
und es deshalb auch nicht weiter schlimm ist.
\\
Aus dem negativen Massendurchsatz folgt auch eine negative mechanische Kompressorleistung. Physikalisch ist eine negative Leistung jedoch nicht möglich. 
Deswegen haben wir den Betrag der Leistung genommen.\\
\\
Die schlechte Isolierung des Versuchsaufbaus hat dazu geführt, dass die realen Werte geringer waren als die der Theorie.
Hinzu kommt, dass der Ableseprozess nicht genau ist.
Die gemessenen Größen sind alle zeitabhängig. Es ist nicht möglich gewesen immer exakt die gleiche Zeit zwischen den Messungen zu lassen.
Die entstandene Abweichung konnte nicht ausgemerzt werden.\\
%keine Adiabatische Kompression...
\newpage
\section{Diskussion}
\label{sec:Diskussion}

Der Versuchsaufbau ist schlecht isoliert. Die Kompression verläuft deshalb nicht adiabatisch. Dieser Umstand macht sich in 
den realen Güteziffern bemerkbar.
Sie lag 78.7\% bis 96.8\% unter den theoretischen Werten. Dies ist auf die Annahmen für $ν_{\text{ideal}}$ zurückzuführen,
welche adiabatische Kompression voraussetzen. Die Wärmeübertragung verläuft nicht vollkommen reversibel.
\\
Der Massendurchsatz hat einen negativen Wert. Das Vorzeichen gibt die Flussrichtung an, deshalb ist das negative Vorzeichen nicht weiter schlimm.
\\
Aus dem negativen Massendurchsatz folgt auch eine negative mechanische Kompressorleistung. Physikalisch ist eine negative Leistung jedoch nicht möglich. 
Deswegen wird der Betrag der Leistung genommen.
\\
Hinzu kommt, dass der Ableseprozess nicht genau ist.
Die gemessenen Größen sind alle zeitabhängig. Es ist nicht möglich gewesen immer exakt die gleiche Zeit zwischen den Messungen zu lassen.
Die entstandene Abweichung konnte nicht ausgemerzt werden.\\

\newpage
\section{Diskussion}
\label{sec:Diskussion}
%Brauchen wir das Überhaupt?
Im Folgenden werden die prozentualen Abweichungen mit 
\begin{equation}\label{eq:1}
    \Delta = |\frac{exp - theo}{theo}|\cdot 100\%
\end{equation}
berechnet.

%Tabelle
%\begin{table}[H]
%    \centering
%    \caption{Experimentell bestimmte Halbwärtszeit $T_{exp}$, Theoriewert $T_{theo}$ und Abweichung in \%.}
%    \begin{tabular}{c c c}
%        \toprule
%        {$T_{exp}\,/\symup{s}$} & {$T_{theo}\,/\symup{s}$} & {$\Delta\,/\symup{\%}$}\\
%        \midrule
%        
%        \bottomrule
%    \end{tabular}
%    \label{tab:Diskussion}
%\end{table}

% gewisse Unsicherheit bei Druckangabe, da skala nur in 20mb Schritten und Messabstände in 50mb Schritten

% Die für die mittlere Reichweite bestimmten Energien stehen im Widerspruch zueinander

% warum entspricht der Zerfall keiner Poissonverteilung??
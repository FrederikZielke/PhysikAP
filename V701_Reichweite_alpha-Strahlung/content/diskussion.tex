\section{Diskussion}
\label{sec:Diskussion}
%Brauchen wir das Überhaupt?
%Im Folgenden werden die prozentualen Abweichungen mit 
%\begin{equation}\label{eq:1}
%    \Delta = |\frac{exp - theo}{theo}|\cdot 100\%
%\end{equation}
%berechnet.


% gewisse Unsicherheit bei Druckangabe, da skala nur in 20mb Schritten und Messabstände in 50mb Schritten
Die im ersten und zweiten Auswertungsteil verwendete Größe der effektiven Reichweite $x$ ist
abhängig vom gemessenen Druck $p$. Da die Skala des verwendeten Manometers nur in $\SI{20}{\milli\bar}$-Schritten
beschriftet ist, die Messabstände aber in $\SI{50}{\milli\bar}$-Schritten erfolgten, ist die Angabe des Drucks
mit einem Ablesefehler behaftet. \\
\\
% ?R_m möglicherweise nicht ganz richtig, da linearität der Impulsabnahme nur näherungsweise gegeben ist?
Die mittlere effektive Reichweite $R_m$ wurde durch eine lineare Ausgleichsrechnung bestimmt. 
Da es sich aber eigentich nur um einen näherungsweisen linearen Zusammenhang handelt, ist die Bestimmung von $R_m$ mit einer gewissen Unsicherheit behaftet.\\
% Unterschied der effektiven Reichweiten:
Die effektiven Reichweiten unterscheiden sich um $\SI{3.5}{mm}$.
Dieser Unterschied ist auf die unterschiedlichen Ausgleichsgeraden zurückzuführen.
Mit mehr Messwerten wäre eine genauere Bestimmung möglich gewesen.\\
% Die für die mittlere Reichweite bestimmten Energien stehen im Widerspruch zueinander
% War das mit der Ausgleichsgeraden der sinnvollste Weg?
Die für die mittlere Reichweite bestimmten Energien hätten auch mit $E_\alpha = \left(\frac{R_m}{3.1}\right)^{\frac{2}{3}}$
berechnet werden können. 
Die verwendete Methode liefert Ergebnisse ohne Überprüfung des Gültigkeitsbereichs.\\
% Die Chanel max angaben sind mit steigendem Druck irgendwann nutzlos weil wir einen threshold eingestellt haben
Das für die Berechnung der Energieverlustes verwendete Channelmaximum verliert zum Ende der Messung an Aussagekraft,
da die Diskriminatorschwelle des Verstärkers ein Maximum unter dem eingestellten Threshold abschneidet.\\
% Abweichung des Energieverlustes?
Aus den abweichenden mittleren Reichweiten $R_m$ folgt auch eine Abweichung des Energieverlustes $\frac{\symup{d}E}{\symup{d}x}$.
\\
% warum entspricht der Zerfall keiner Poissonverteilung??
Die Wahrscheinlichkeitsverteilung der Zerfallsrate im letzten aufgabenteil entspricht nicht der Poissonverteilung.
Anderen Versuchen und Literatur ist zu entnehmen, dass eigentlich eine Poissonverteilung zu erwarten ist.\\
Dies könnte an einer zu geringen Anzahl an Messwerten liegen.\\
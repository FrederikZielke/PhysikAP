\section{Auswertung}
\label{sec:Auswertung}
Die in \autoref{sec:Auswertung} gezeigten Grafiken und Rechnungen sind mithilfe der Python-Bibliotheken Matplotlib \cite{matplotlib}, Scipy \cite{scipy} und Numpy \cite{numpy}
erstellt worden. Die Fehlerrechnung wird mithilfe von Uncertainties \cite{uncertainties} durchgeführt.

\subsection{Ermittlung der mittleren effektiven Reichweite}
\label{sec:reichweite}

Die erste Messung wird mit einem Abstand von $x_0 = \SI{5.5}{cm}$ zwischen Alphastrahler und Detektor durchgeführt.
Die Messwerte sind in \autoref{tab:reichweite1} aufgeführten und in \autoref{fig:reichweite1} graphisch dargestellt.
Die effektive Reichweite von $\alpha$-Strahlung wird mit \autoref{eq:reichweite2} aus dem Drcuk $p$ und dem Abstand $x_0$ bestimmt. 
\begin{table}
  \centering
  \caption{Messwerte zur Bestimmung der mittleren Reichweite mit einem Abstand von $\SI{5.5}{cm}$.
  Angegeben sind der Druck $p$, die gemessenen Impulse $N$ und die zugehörigen Kanalnummern mit den meisten Impulsen.}
  \begin{tabular}{|c c c|}
    \toprule
    {$p\,/\symup{mbar}$} & {$N$} & {Chanel nr. max} \\
    \midrule
    0 & 15729 & 699\\
    50 & 15396 & 700\\
    100 & 15104 & 655\\
    150 & 14515 & 623\\
    200 & 14321 & 540\\
    250 & 13452 & 495\\
    300 & 12539 & 440\\
    350 & 9862 & 382\\
    400 & 3643 & 220\\
    450 & 224 & 335\\
    500 & 0 & 0\\
    \bottomrule
  \end{tabular}
  \label{tab:reichweite1}
\end{table}\\

Die mittlere effektive Reichweite $R_m$ wird durch eine lineare Ausgleichsrechnung ermittelt.
In die Ausgleichsgerade gehen nur die Messwerte nach dem anfänglichen Plateau ein.
So entspricht $R_m$ dem Schnittpunkt der Ausgleichsgerade mit einer zur $x$-Achse parallelen Geraden, 
die auf halber Höhe des Plateaus liegt. An diesem Punkt treffen noch die Hälfte der $\alpha$-Teilchen auf den Detektor.\\

\begin{figure}[H]
  \centering
  \includegraphics{build/plot1.pdf}
  \caption{Treffer pro Sekunde gegen die effektive Länge aufgetragen mit linearer Regression und Plateauhalbierenden.}
  \label{fig:reichweite1}
\end{figure}

Die Steigung und der y-Achsenabschnitt der Ausgleichsgerade sind
\begin{align*}
  m &= \SI{-1.33(0.14)e04}{\per\meter\per\second} \\
  b &= \SI{325(29)}{\per\second}.\\
\end{align*}
Die mittlere effektive Reichweite ist der $x$-Achsenabschnitt des Schnittpunktes der Ausgleichsgeraden mit der Plateauhalbierenden
\begin{equation*}
  R_{m,1} = \SI{0.0205(0.0031)}{\meter}.
\end{equation*}

Die Messung wird mit einem Abstand von $x_0 = \SI{4.5}{cm}$ wiederholt.
Die Messwerte sind in \autoref{tab:reichweite2} aufgeführt und in \autoref{fig:reichweite2} graphisch dargestellt.
Die Auswertung erfolgt analog zur ersten Messung.
\begin{table}
  \centering
  \caption{Messwerte zur Bestimmung der mittleren effektiven Reichweite mit einem Abstand von $\SI{4.5}{cm}$.}
  \begin{tabular}{|c c c|}
    \toprule
    {$p\,/\symup{mbar}$} & {$N$} & {Chanel nr. max} \\
    \midrule
    0 & 22090 & 720\\
    50 & 21788 & 687\\
    100 & 21427 & 630\\
    150 & 20792 & 604\\
    200 & 20063 & 556\\
    250 & 19549 & 520\\
    300 & 18888 & 478\\
    350 & 17508 & 463\\
    400 & 14566 & 398\\
    450 & 9412 & 344\\
    500 & 1309 & 324\\
    550 & 73 & 326\\
    600 & 0 & 0\\
    \bottomrule
  \end{tabular}
  \label{tab:reichweite2}
\end{table}\\

\begin{figure}[H]
  \centering
  \includegraphics{build/plot2.pdf}
  \caption{Treffer pro Sekunde gegen die effektive Länge aufgetragen mit linearer Regression und Plateauhalbierenden.}
  \label{fig:reichweite2}
\end{figure}

Die Steigung und der y-Achsenabschnitt der Ausgleichsgerade sind
\begin{align*}
  m &= \SI{-1.48(0.18)e04}{\per\meter\per\second} \\
  b &= \SI{4.3(0.4)e02}{\per\second}.\\
\end{align*}
Die mittlere effektive Reichweite ergibt sich zu
\begin{equation*}
  R_{m,2} = \SI{0.024(0.004)}{\meter}.
\end{equation*}

Mit der Beziehung \eqref{eq:HierEinfügenLennart} lässt sich die Energie der mittleren Reichweite bestimmen.
Unter der Annahme, dass die Energie der $\alpha$-Teilchen $E_\alpha \leq \SI{2.5}{MeV}$ beträgt, ergibt sich für $R_{m,1}$
\begin{equation*}
  E_{\alpha,1}\left(R_{m,1}\right) = \SI{3.5(0.4)}{MeV}.
\end{equation*}
Für $R_{m,2}$ ergibt sich
\begin{equation*}
  E_{\alpha,2}\left(R_{m,2}\right) = \SI{3.9(0.4)}{MeV}.
\end{equation*}
Die berechneten Energien erfüllen die Bedingung $E_\alpha \leq \SI{2.5}{MeV}$ nicht.

\subsection{Energieverlust}
\label{sec:energieverlust}

Zur Bestimmung des Energieverlustes von $\alpha$-Teilchen in Luft wird die Energie der Teilchen in Abhängigkeit von der effektiven Länge aufgetragen.
Aus der Anleitung ist bekannt, dass bei einem Druck von ca. $\SI{0}{mbar}$ der Energiepeak bei $\SI{4}{MeV}$ liegt.
Per Dreisatz wird die Energie der Teilchen bei den anderen Drücken bestimmt. 
Die Energiepeaks entsprechen den Channeln in \autoref{tab:reichweite1} und \autoref{tab:reichweite2}.
Um nun den Energieverlust zu bestimmen wird eine Ausgleichsgerade durch die Messwerte gelegt.
Die Steigung der Ausgleichsgerade entspricht dem Energieverlust.
Die Messwerte und die Ausgleichsgerade sind in \autoref{fig:energieverlust1} und \autoref{fig:energieverlust2} dargestellt.

\begin{figure}% doppelfigure mit beiden plots nebeneinander
  \centering
  \begin{subfigure}{0.48\textwidth}
    \includegraphics[width=\textwidth]{build/plot1_Energie.pdf}
    \caption{Messreihe mit Abstand $x_0 = \SI{5.5}{cm}$}
    \label{fig:energieverlust1}
  \end{subfigure}
  \hfill
  \begin{subfigure}{0.48\textwidth}
    \includegraphics[width=\textwidth]{build/plot2_Energie.pdf}
    \caption{Messreihe mit Abstand $x_0 = \SI{4.5}{cm}$}
    \label{fig:energieverlust2}
  \end{subfigure}
  \caption{Energie der $\alpha$-Teilchen in Abhängigkeit von der effektiven Länge mit linearer Regression.}
  \label{fig:energieverlust}
\end{figure}

Die Steigung und der Ordinatenabschnitt der Ausgleichsgeraden der ersten Messung ist
\begin{align}\label{alg:energieverlust1}
  m_1 &= \SI{-118(13)}{MeV/m} = \frac{\symup{d}E_1}{\symup{d}x}\\
  b_1 &= \SI{4.39(0.2)}{MeV}.
\end{align}
Die Steigung und der Ordinatenabschnitt der Ausgleichsgeraden der zweiten Messung ist
\begin{align}\label{alg:energieverlust2}
  m_2 &= \SI{-78.4(2.7)}{MeV/m} = \frac{\symup{d}E_2}{\symup{d}x}\\
  b_2 &= \SI{3.97(0.05)}{MeV}.
\end{align}

% Im Zweifelsfall entfernen
Mit der Regressionsgerade lässt sich die Energie in \autoref{sec:reichweite} bestimmten mittleren effektiven Reichweite bestimmen.
Mit den Parametern aus \autoref{alg:energieverlust1} ergibt sich für $R_{m,1}$ eine Energie von
\begin{equation*}
  E_1\left(R_{m,1}\right) = \SI{2.0(0.5)}{MeV}.
\end{equation*}
Mit \autoref{alg:energieverlust2} ergibt sich für $R_{m,2}$ eine Energie von
\begin{equation*}
  E_2\left(R_{m,2}\right) = \SI{2.06(0.33)}{MeV}.
\end{equation*}

\subsection{Statistik des radioaktiven Zerfalls}
\label{sec:statistik}

\begin{table}
  \centering
  \caption{hier caption einfuegen}
  \begin{tabular}{|c|c|c|}
    \toprule
    \multicolumn{3}{|c|}{Impulse pro 10 Sekunden}\\
    \midrule
    226 & 273 & 267 \\
    \bottomrule
  \end{tabular}
  \label{tab:statistik}
\end{table}

% Histogram einfügen, Mittelwert und Varianz angeben, auf Vergleich mit Gauß und Poisson eingehen
% auf die Bins und Delta N eingehen?? 
Bei einem Abstand von $x_0 = \SI{4.5}{cm}$, einem Druck von $p \approx \SI{0}{mbar}$ und einer Messzeit von $t = \SI{10}{s}$ werden die Zerfälle pro Sekunde gemessen.
Die Messwerte sind in \autoref{tab:statistik} aufgeführt.
\begin{figure}
  \centering
  \includegraphics[width=\textwidth]{build/hist.pdf}
  \caption{Histogramm der Zerfälle pro Sekunde.}
  \label{fig:hist}
\end{figure}


%Tabellen nebeneinander
%\begin{table}[H]
%  \centering
%  \caption{}
%  \resizebox{0.4\textwidth}{!}{
%  \begin{tabular}{S[table-format=3] S[table-format=2(2)]}
%      \toprule
%      {$t\,/\symup{s}$} & {$N$} \\
%      \midrule
%
%      \bottomrule
%  \end{tabular}
%  \begin{tabular}{S[table-format=3] S[table-format=2(2)]}
%      \toprule
%      {$t\,/\symup{s}$} & {$N$} \\
%      \midrule
%      
%      \bottomrule
%  \end{tabular}
%  }
%  \label{tab:}
%\end{table}
%Plots und Bilder
%\begin{figure}[H]
%  \includegraphics[width=\linewidth]{plots/.pdf}
%  \caption{}
%  \label{fig:}
%\end{figure}
\section{Zielsetzung}
\label{sec:Zielsetzung}
Der Versuch beschäftigt sich mit der Reichweite von $\alpha$-Strahlung in Luft. 
\section{Theorie}
\label{sec:Theorie}
$\alpha$-Strahlung besteht aus Heliumkernen, die aus zwei Protonen und zwei Neutronen bestehen. Sie entstehen bei der Kernumwandlung von schweren Elementen. 
Die $\alpha$-Strahlung ist positiv geladen und besitzt eine hohe kinetische Energie. Die Reichweite von $\alpha$-Strahlung in Luft ist abhängig von der Energie der Strahlung. 
Die Energie der Strahlung nimmt mit der Reichweite ab. Die $\alpha$-Strahlung wechselwirkt mit Materie durch die elastische Streuung an den Hüllenelektronen, 
durch die inelastische Streuung an den Hüllenelektronen und durch die direkte Wechselwirkung mit dem Kern. 
Die elastische Streuung an den Hüllenelektronen ist nur bei kleinen Energien von Bedeutung. Die inelastische Streuung an den Hüllenelektronen ist bei mittleren Energien von Bedeutung. 
Die direkte Wechselwirkung mit dem Kern ist bei hohen Energien von Bedeutung. \\
Beim Durchlaufen von Materie gibt die $\alpha$-Strahlung ihre Energie durch Ionisation und Anregung ab. Die Energieverluste durch Ionisation und Anregung sind proportional zur Energie.
Die Energieverluste durch Ionisation und Anregung sind bei kleinen Energien dominant. Bei hohen Energien sind die Energieverluste durch Ionisation und Anregung vernachlässigbar und die
Bethe-Bloch-Gleichung beschreibt den Energieverlust der $\alpha$-Strahlung. Die Bethe-Bloch-Gleichung ist gegeben durch
\begin{equation}
    -\frac{\symup{d}E_\alpha}{\symup{d}x} = \frac{z^2e^4}{4\pi\epsilon_0m_e}\frac{nZ}{v^2}\ln{\frac{2m_ev^2}{I}}.
    \label{eqn:bethebloch}
\end{equation}
Dabei ist $z$ die Ladung der $\alpha$-Teilchen, $e$ die Elementarladung, $m_e$ die Elektronenmasse, $n$ die Teilchendichte, $Z$ die Ordnungszahl des Targetgases, $v$ die Geschwindigkeit der $\alpha$-Teilchen und $I$ die Ionisierungsenergie des Targetgases.
\subsection{Herleitung der Bethe-Bloch-Gleichung}
$\alpha$-Strahlung besteht aus Heliumkernen, also schweren geladenen Teilchen. Die $\alpha$-Teilchen wechselwirken mit den Hüllenelektronen des Targetgases. 
Der beim Vorbeifliegen der $\alpha$-Teilchen an den Hüllenelektronen übertragene Impuls ist
\begin{align*}
    \Delta p &= \int_{-\infty}^{\infty}F\symup{d}t\\
             &= \frac{1}{v}\int_{-\infty}^{\infty}F_{\perp}\symup{d}x\\
             &= \frac{e}{v}\int_{-\infty}^{\infty}E_{\perp}\symup{d}x.\\
\end{align*}
Dabei ist $F$ die Coulomb-Kraft zwischen dem $\alpha$-Teilchen und dem Hüllenelektron, $v$ die Geschwindigkeit des $\alpha$-Teilchens, $F_{\perp}$ die Kraftkomponente senkrecht zur Bewegungsrichtung des $\alpha$-Teilchens, $E_{\perp}$ die elektrische Feldstärke senkrecht zur Bewegungsrichtung des $\alpha$-Teilchens und $e$ die Elementarladung.
Dieser Zusammenhang beruht auf der Annahme, dass die kinetische Energie des Teilchens groß gegenüber der Bindungsenergie des Elektrons ist und die Teilchenbahn als
gerade angenommen werden kann. \\
Mit dem Gauss'schen Integralsatz folgt
\begin{equation*}
    2\pi b \cdot \int_{-\infty}^{\infty}E_{\perp}\symup{d}x = \frac{Q}{\epsilon_0} = \frac{ze}{\epsilon_0}.
\end{equation*}
Damit gilt für den Impulsübertrag
\begin{equation*}
    \Delta p = \frac{2\pi ze^{2}}{\epsilon_0vb}.
\end{equation*}
Die Wechselwirkung der $\alpha$-Teilchen mit allen Elektronen entlang der Teilchenbahn führt zu einer Änderung der Energie pro Wegstrecke $\symup{d}x$ von
\begin{equation*}
    \symup{d}E = -(\int_{b_{min}}^{b_{max}}\frac{\Delta p^2}{2m_e}n_e \cdot 2\pi b \symup{d}b)\symup{d}x.
\end{equation*}
Es wird also über alle Stoßparameter $b$ integriert. Die Grenzen hängen vom Verhältnis der Teilchenenergie zur Bindungsenergie ab. Dabei ist $n_e = \frac{z\rho}{Au}$ die Elektronendichte. $A$ ist die Massenzahl des Targetgases, $u$ die Atomare Masseneinheit und $\rho$ die Dichte des Targetgases. \\
Somit ergibt sich der Energieverlust zu 
\begin{equation*}
    \frac{\symup{d}E}{\symup{d}x} = -\frac{4\pi z^2e^4n_e}{m_e\epsilon_0^2v^2}\ln{\frac{b_{max}}{b_{min}}}.
\end{equation*}
Die Formel hier ist vereinfacht und gilt nur für den Fall $\frac{v}{c} << 1$, also kleine kinetische Energien.
Jedoch ist die Formel auch nur für hinreichend große Energien gültig, wenn diese hoch genug sind, dass die Teilchen keine Hüllenelektronen anregen,
denn dann treten Ladungsaustauschsprozesse statt.

\subsection{Reichweite eines Teilchens}
Die Reichweite eines Teilchens ist die Strecke, die das Teilchen in einem Medium zurücklegt, bis es seine gesamte Energie abgegeben hat bzw. gebremst wurde.
Die mittlere Reichweite ergibt sich aus der Energie der $\alpha$-Teilchen $E_{\alpha}$ und dem mittleren Energieverlust. Die mittlere Reichweite ist gegeben durch
\begin{equation}
    R_m = -\int_0^{E_{\alpha}}\frac{\symup{d}E}{\symup{d}E/\symup{d}x}.
    \label{eqn:reichweite}
\end{equation}
Da bei niedrigeren Energien die Energieverluste durch Ionisation und Anregung nicht mehr durch die Bethe-Bloch-Gleichung beschrieben werden können, wird die Reichweite
durch die empirische bestimmte Bragg-Kurve beschrieben. Bei konstanter Temperatur und konstantem Volumen ist die Reichweite proportional zum Druck $p$ des Targetgases.
Deshalb kann die Reichweite durch eine Absorptionsmessung bestimmt werden, bei der der Druck $p$ variiert wird. Die Reichweite ist dann
\begin{equation*}\label{eq:reichweite2}
    R_m = R_m(p=0)\frac{p}{p_0}.
\end{equation*}
Dabei ist $p_0$ der Normaldruck und $R_m(p=0)$ die Reichweite bei $p=0$.

\subsection{Funktionsweise eines Halbleiter-Sperrschichtzählers}
\label{sec:sperrschicht}
Der Halbleiter-Sperrschichtzähler besteht aus einem p- und einem n-dotierten Halbleiter, die aneinander grenzen. Die p- und n-Schicht werden durch eine Sperrschicht getrennt.
Die Sperrschicht wird durch eine äußere Spannung vergrößert und ist im wesentlichen eine ladungsfrei Zone.
Dringt $\alpha$-Strahlung in die Sperrschicht ein, so entstehen durch inelastische Stöße der Heliumkerne mit den "freien" Hüllenelektronen Elektron-Loch-Paare. Die Elektronen und Löcher werden durch das elektrische Feld der Sperrschicht getrennt und wandern zu den Elektroden.
Dadurch entsteht ein Stromimpuls, der gemessen werden kann. Die Energie der $\alpha$-Teilchen kann durch die Höhe des Stromimpulses bestimmt werden.
Der Halbleiter-Sperrschichtzähler hat einige Vorteile gegenüber einer Ionisationskammer. Durch die höhere Dichte des Halbleitermaterials, werden die $\alpha$-Teilchen stärker abgebremst und die Reichweite ist kleiner.
Dadurch kann die Energie der $\alpha$-Teilchen genauer bestimmt werden. Außerdem ist die Ansprechzeit des Halbleiter-Sperrschichtzählers kleiner als die einer Ionisationskammer, weil die Elektronen und Löcher durch das elektrische Feld schneller getrennt werden. Zusätzlich
ist die notwendige Energie zur Erzeugung eines Elektron-Loch-Paares geringer als die Energie, die zur Ionisation eines Gases benötigt wird. Es können also auch Teilchen mit geringerer Energie gemessen werden. \\
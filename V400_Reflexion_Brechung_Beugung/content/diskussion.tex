\section{Diskussion}
\label{sec:Diskussion}
Im Folgenden werden die Abweichungen wie in \autoref{sec:Brechungsgesetz} berechnet.

% Allgemeine Probleme: Es war nicht gut möglich, die Winkel genau abzulesen, da die Skala nicht fein 
% genug und der Strahl zu breit war. Der Aufbau war nicht besonders Stabil und konnte leicht während 
% der Durchführung verrutschen.
Bei allen Messungen war es nicht möglich, die Winkel genau abzulesen, da die Skala nicht fein genug war.
Des weiteren war der Laserstrahl zu breit, um eine akkurate Messung durchzuführen.\\
Der gesamte Aufbau war nicht besonders stabil und konnte leicht während der Durchführung verrutschen.\\

% Aufgabe 1 Reflexionsgesetz: immer 0.5° daneben -> systematischer Fehler, Anbringung?
Bei Aufgabe 1 \ref{sec:Reflexionsgesetz} kommt es zu Abweichung der Messdaten vom theoretisch erwarteten Wert zwischen $1.67\%$ und $0.71\%$.
Die Abweichung ist bei Allen Messwerten $0.5°$, was auf einen systematischen Fehler schließen lässt.
Dieser könnte durch eine ungenaue Anbringung des Spiegels entstanden sein.\\
\\
Der bei Aufgabe 2 bis 5 verwendete Schirm zur Messung des Brechungswinkels war nicht genau in die richtige Form zu bringen.\\
\\
Die Abweichung der Laserwellenlänge wird durch einzelne ausreißer verfälscht. 
Durch Selektion von Messwerten könnte die Abweichung verringert werden.\\
\newpage
% Aufgabe 2,3,4,5 -> der Schirm war nicht gut in die richtige Form zu bekommen

%Tabelle
%\begin{table}[H]
%    \centering
%    \caption{Experimentell bestimmte Halbwärtszeit $T_{exp}$, Theoriewert $T_{theo}$ und Abweichung in \%.}
%    \begin{tabular}{c c c}
%        \toprule
%        {$T_{exp}\,/\symup{s}$} & {$T_{theo}\,/\symup{s}$} & {$\Delta\,/\symup{\%}$}\\
%        \midrule
%        
%        \bottomrule
%    \end{tabular}
%    \label{tab:Diskussion}
%\end{table}
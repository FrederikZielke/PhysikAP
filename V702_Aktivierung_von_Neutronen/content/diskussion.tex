\section{Diskussion}
\label{sec:Diskussion}
Im Folgenden werden die prozentualen Abweichungen mit 
\begin{equation}\label{eq:1}
    \Delta = |\frac{exp - theo}{theo} \cdot 100|\%
\end{equation}
berechnet.
\subsection{Nulleffekt}
Der Nulleffekt wurde in \autoref{sec:Nulleffekt} zu $N_0 = \SI{0.2}{\frac{Impulse}{s}}$ bestimmt. 
Mit der Ausgleichsrechnung beim Zerfall von Vanadium und Silber kann eine Abschätzung für den Nulleffekt getroffen werden. Die Daten sind in \autoref{tab:dis} 
aufgelistet.

\begin{table}[H]
    \centering
    \caption{Nulleffekt und Abschätzung durch Ausgleichsrechnung.}
    \begin{tabular}{c c}
        \toprule
        {Messung} & {$N_0$}\\
        \midrule
        Vanadium & $\SI{0.2924(0.094)}{\frac{Impulse}{s}}$\\
        Silber 1 & $\SI{0.8610(0.120)}{\frac{Impulse}{s}}$\\
        Silber 2 & $\SI{0.7020(0.006)}{\frac{Impulse}{s}}$\\
        \bottomrule
    \end{tabular}
    \label{tab:dis}
\end{table}

Alle durch die Ausgleichsrechnung bestimmten Werte liegen über dem gemessenen Nulleffekt.
Ein Grund dafür kann sein, dass die Messungen nicht lange genug durchgeführt worden sind. Gerade bei dem Zerfall von Silber ist es gegen Ende 
der Messdaten schwierig den Zerfall und die Hintergrundstrahlung von einander zu unterscheiden, da beide Größen dort 
in der gleichen Größenordnung liegen.

\subsection{Zerfall von Vanadium}
Die bestimmte Halbwärtszeit, der Theoriewert \cite{vanadium} und die Abweichung sind in \autoref{tab:van} aufgelistet.
\begin{table}[H]
    \centering
    \caption{Experimentell bestimmte Halbwärtszeit $T_{exp}$, Theoriewert $T_{theo}$ und Abweichung in \%.}
    \begin{tabular}{c c c}
        \toprule
        {$T_{exp}\,/\symup{s}$} & {$T_{theo}\,/\symup{s}$} & {$\Delta\,/\symup{\%}$}\\
        \midrule
        $225\pm30$ & 224.58 & 0.19\\
        \bottomrule
    \end{tabular}
    \label{tab:van}
\end{table}
Der Verlauf der Messwerte folgt dem erwarteten exponentiellen Abfall. Je geringer jedoch die Strahlung wird, desto mehr schwanken die Messwerte, da Hintergrundstrahlung und
Strahlung des Zerfalls in einem sehr ähnlichen Bereich liegen. Die Halbwärtszeit liegt im erwarteten Bereich.

\subsection{Zerfall von Silber}
Die bestimmten Halbwärtszeiten, der Theoriewert für den kurzen \cite{silber_kurz} und für den langen Zerfall \cite{silber_lang} und die Abweichung sind in \autoref{tab:sil} aufgelistet.
\begin{table}[H]
    \centering
    \caption{Experimentell bestimmte Halbwärtszeit $T_{exp}$, Theoriewert $T_{theo}$ und Abweichung in \%.}
    \begin{tabular}{c c c c}
        \toprule
        {Messreihe} & {$T_{exp}\,/\symup{s}$} & {$T_{theo}\,/\symup{s}$} & {$\Delta\,/\symup{\%}$}\\
        \midrule
        1, kurz & $\SI{37(4)}{}$ & 24.56 & 50.65\\
        1, lang & $\SI{110(16)}{}$ & 142.92 & 23.03\\
        2, kurz & $\SI{33.6(2.7)}{}$ & 24.56 & 36.81\\
        2, lang & $\SI{124(16)}{}$ & 142.92 & 13.24\\
        \bottomrule
    \end{tabular}
    \label{tab:sil}
\end{table}
Die experimentell bestimmten Halbwärtszeiten liegen zwar in der richtigen Größenordnung, weichen jedoch alle deutlich von der Theorie ab. Da die Messwerte generell 
bei geringer Intensität großen Schwankungen unterliegen, sind die aus der Ausgleichsrechnung resultierenden Halbwärtszeiten mit einem großen Fehler behaftet. Trotz des 
großen Fehlers liegt der Theoriewert bei keinem experimentell besimmten Wert im Fehlerbereich. \\
Da der Zerfall von instabilen Isotopen ein stochastischer Prozess ist und die Hintergrundstrahlung nicht gut genug vom Signal getrennt werden kann, sind Schwankungen
zu erwarten gewesen.\\
Generell ist der Versuch dennoch geeignet, um eine grobe Abschätzung für den Zerfall zu erhalten.
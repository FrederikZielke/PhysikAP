\section{Diskussion}
\label{sec:Diskussion}
Im Folgenden werden die prozentualen Abweichungen mit 
\begin{equation}\label{eq:1}
    \Delta = |\frac{exp - theo}{theo} \cdot 100|\%
\end{equation}
berechnet.
\subsection{Nulleffekt}
Der Nulleffekt wurde in \autoref{sec:Nulleffekt} zu $N_0 = \SI{0.2}{\frac{Impulse}{s}}$ bestimmt. 
Mit der Ausgleichsrechnung beim Zerfall von Vanadium und Silber kann eine Abschätzung für den Nulleffekt getroffen werden. Die Daten sind in \autoref{tab:dis} 
aufgelistet.

\begin{table}[H]
    \centering
    \caption{Nulleffekt und Abschätzung durch Ausgleichsrechnung.}
    \begin{tabular}{c c}
        \toprule
        {Messung} & {$N_0$}\\
        \midrule
        Vanadium & $\SI{0.2924(0.094)}{\frac{Impulse}{s}}$\\
        Silber 1 & $\SI{0.2870(0.040)}{\frac{Impulse}{s}}$\\
        Silber 2 & $\SI{0.2340(0.002)}{\frac{Impulse}{s}}$\\
        \bottomrule
    \end{tabular}
    \label{tab:dis}
\end{table}

Alle durch die Ausgleichsrechnung bestimmten Werte liegen knapp über dem gemessenen Nulleffekt.
Ein Grund dafür kann sein, dass die Messungen nicht lange genug durchgeführt worden sind. Generell ist es gegen Ende 
der Messdaten schwierig den Zerfall und die Hintergrundstrahlung von einander zu unterscheiden, da beide Größen dort 
in der gleichen Größenordnung liegen.

\subsection{Zerfall von Vanadium}

\section{Theorie}
\label{sec:Theorie}
Das Ziel des Versuchs ist es, bei einer senkrecht eingespannten Feder die Federkonstante $D$ zu bestimmen. Die Theorie	hierzu besagt, dass die Kraft $F$, die auf die Feder wirkt, proportional zu der Auslenkung $\Delta x$ ist:
\begin{equation}
    \begin{aligned}
    F = D \cdot \Delta x
\end{aligned}
\end{equation}
Um diesen theoretischen Zusammenhang zu prüfen, wird eine Feder senkrecht zwischen einen Seilzug und ein Kraftmessgerät eingespannt. Der Seilzug ist an einer Messskala befestigt, die die Auslenkung der Feder in $\unit{\cm}$ angibt. Oben an der Feder ist das Kraftmessgerät befestigt, welches die auf die Feder wirkende Kraft (in $\unit{\N}$) auf zwei Nachkommastellen genau digital angibt.
Es werden 10 Messwertepaare zur Bestimmung der Federkonstante aufgenommen. Die beiden Methoden zur Bestimmung sind:
\begin{enumerate}
    \item Mittelwertbildung
    \item Lineare Ausgleichsrechnung
\end{enumerate}
\subsection{Methode der Mittelwertbildung}
\begin{equation}
    \begin{aligned}
    \overline{D} {} = \,\frac{1}{10} \sum_{i=1}^{10} D_i
    \label{eqn:methode1}
\end{aligned}
\end{equation}
\subsection{Methode der linearen Ausgleichsrechnung}
\begin{equation}
\begin{aligned}
      \underline{A}^T \cdot \underline{A} \cdot \vec{x} = \underline{A}^T \cdot \vec{b}
\end{aligned}
\end{equation}


\newpage
\section{Auswertung}
\label{sec:Auswertung}
\subsection{Messwerte}
Aus $F = D \cdot \Delta x \Leftrightarrow D = \frac{F}{\Delta x}$ folgt:
\begin{table}[H]
  \centering
  \begin{tabular}
    {
        S[table-format = 2.0]
        S[table-format = 1.2]
        S[table-format = 1.5]
    }
    \toprule
    {$\Delta x \left[\unit{\cm}\right]$} &
    {$F \left[\unit{\N}\right]$} &
    {$D \left[\unit{\N/\cm}\right]$} \\
    \midrule    
5  &   0.15  & 0.03000\\
10   &   0.29  & 0.02900\\
15   &   0.44   & 0.02933\\
20  &   0.59  & 0.02950\\
25   &   0.74   & 0.02960\\
30   &   0.89  & 0.02967\\
35  &   1.04 & 0.02971\\
40   &   1.19   & 0.02975\\
45   &   1.34   & 0.02978\\
50  &   1.49  & 0.02980\\
    \bottomrule
\end{tabular}
  \caption{\label{tab:Messdaten}Messdaten und Federkonstante D}
\end{table}

\subsection{Bestimmung der Federkonstante aus Mittelwertbildung}
%\begin{equation}
\begin{multline}
\begin{aligned}
  \overline{D} {} = & \,\frac{1}{10} \sum_{i=1}^{10} D_i \\
        =& \,\frac{1}{10} (0.03000 + 0.02900 + 0.02933 + 0.02950 + 0.02960\\
        & \:\;+ 0.02967 + 0.02971 + 0.02975 + 0.02978 + 0.02980)\\
        =& \,0.02961
\end{aligned}
\end{multline}
%\end{equation}
\newpage
\subsection{Bestimmung der Federkonstante aus linearer Ausgleichsrechnung}
Wir benutzen zur Berechnung die Methode der kleinsten Quadrate:
\begin{equation}
\begin{aligned}
  \underline{A}^T \cdot \underline{A} \cdot \vec{x} = \underline{A}^T \cdot \vec{b}
\end{aligned}
\end{equation}
mit 
  \begin{minipage}{0.48\linewidth}

    \[
    \underline{A} =
    \left(\begin{array}{cc}
      1 & 5\\
      1 & 10\\
      1 & 15\\
      1 & 20\\
      1 & 25\\
      1 & 30\\
      1 & 35\\
      1 & 40\\
      1 & 45\\
      1 & 50\\
    \end{array} \right)\\
    \]
    \vspace{10pt}
    \end{minipage}
    \begin{minipage}{0.01\linewidth}
    ,
    \end{minipage}
    \begin{minipage}{0.48\linewidth}
    
    \[
    \vec{b} =
    \left(\begin{array}{c}
      0.15\\
      0.29\\
      0.44\\
      0.59\\
      0.74\\
      0.89\\
      1.04\\
      1.19\\
      1.34\\
      1.49\\
    \end{array} \right) \\
    \]
    \vspace{10pt}
    \end{minipage}
    \newline %funktioniert irgendwie nicht wie ich mir das vorstelle :(
    \newline
    Wir erhalten $\underline{A}^T \cdot \underline{A} = \begin{pmatrix}
      10 & 275\\ 
      275 & 9625
    \end{pmatrix}$ 
    und $\underline{A}^T \cdot \vec{b} =  \begin{pmatrix}
      8.16\\ 
      286.15
    \end{pmatrix}$.
    \newline
    \noindent Daraus folgt nun das Gleichungssystem 
    $\begin{pmatrix}
      10 & 275\\ 
      275 & 9625
    \end{pmatrix}
    \cdot \vec{x}
    = \begin{pmatrix}
      8.16\\ 
      286.15
    \end{pmatrix}$
    mit $\vec{x} = \begin{pmatrix}
      b\\ 
      m
    \end{pmatrix}$
    wobei b der y-Achsenabschnitt und m die Steigung der linearen Ausgleichsgerade ist.\\
    \noindent Das Gleichungssystem kann nun beispielsweise mit dem Gauß-Algorithmus gelöst werden. Wir erhalten $b = -\frac{3}{500}, m = \frac{411}{13750} \approx 0.02989$
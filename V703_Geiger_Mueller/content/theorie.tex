\section{Zielsetzung}
\label{sec:Zielsetzung}

\section{Theorie}
\label{sec:Theorie}

Das Geiger-Müller-Zählrohr ist ein Detektor für ionisierende Strahlung. 
Ionisierende Strahlung ist in α-, β- und γ-Strahlung unterteilt. 
α-Strahlung besteht aus Heliumkernen, β-Strahlung aus Elektronen und γ-Strahlung aus Photonen.
Das Geiger-Müller-Zählrohr ist am besten für die Messung von 
α- und γ-Strahlung geeignet. Die Nachweisbarkeit von γ-Strahlung ist klein.
\\
Das Zählrohr besteht aus aus einem Anodendraht und einer Kathodenhülle.
Zwischen Kathode und Anode wird eine Spannung angelegt.
Durch eine dünne Mylarfolie ist das Rohr verschlossen.
Im Inneren befindet sich ein Gasgemisch aus Argon und einem Alkohol.
Durch eintretende Strahlung wird Argon ionisiert. 
Durch das E-Feld im Inneren werden die Gasionen zum Kathodenzylinder und die Elektronen zum Anodendraht hin beschleunigt.
Durch interaktion mit weiteren Gasatomen kommt es zu einer Kettenreaktion.
\section{Durchführung}
\label{sec:Durchführung}

\subsection{Aufbau}

Der Versuch besteht aus einer radioaktiven Quelle, einem Geiger-Müller-Zählrohr und der Messelektronik.
Das Zählrohr, ein Amperemeter und die Strahlungsquelle befinden sich in einer Aluminiumbox.
Der Ausgang des Geiger-Müller-Zählrohrs wird an ein Oszilloskop und an einen Zähler angeschlossen.
Zwischen Zähler und Zählrohr befindet sich ein Verstärker. 
Betrieben wird das GM-Zählrohr mit einer Spannungsquelle, welche in $\SIrange{1}{100}{\volt}$ einstellbar ist.
Der Messzeitraum kann am Zähler frei eingestellt werden.
Um sicherzustellen, dass der Fehler bei der Messung unter $1\%$ liegt, wird bei einer Integrationszeit von $60\si{\second}$ eine Probemessung vorgenommen.

\subsection{Messung}

Im ersten Versuchsteil wird die Spannungsquelle auf $\SI{510}{\volt}$ eingestellt. Der Zähler wird auf eine Integrationszeit von $\SI{100}{\second}$ eingestellt.
Die Anfangsspannung ist so gewählt, dass gerade Strahlung detektiert wird. Nun wird die Spannung in $\SI{20}{\volt}$ schritten erhöht.
Für jede Spannung wird der Strom, der durch das Zählrohr fließt und die Anzahl der Impulse gemessen.
Die Spannung wird so lange erhöht, bis die Detektierten Impulse stark ansteigen.\\
\\
Im zweiten Versuchsteil wird die Spannung auf den Arbeitspunkt gestellt. Die Integrationszeit wird auf $\SI{120}{\second}$ eingestellt. 
Der Messzeitraum wird auf $\SI{120}{\second}$ erhöht. Die Strahlungsquelle wird näher am Detektor positioniert.
Die Anzahl der Impulse wird aufgezeichnet. Danach wird eine zweite Strahlenquelle in der Aluminiumbox platziert.
Die Anzahl der Impulse wird wieder aufgezeichnet. 
Die erste Quelle wird entfernt und es wird ein drittes mal die Anzahl der Impulse aufgezeichnet.\\
\\
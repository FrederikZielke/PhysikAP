\section{Diskussion}
\label{sec:Diskussion}

Um die Charakteristik des Geiger-Müller-Zählrohres zu überprüfen, wird beurteilt, wie stark 
der Plateauanstieg auf der Arbeitsebene des Zählrohres ist. Je geringer der Anstieg, desto besser das Zählrohr.
Dieser Anstieg ergibt sich zu $s_{\%} = 7\%$. Da die Messwerte und die Fehlerbereiche in einer erwarteten Größenordnung liegen, kann davon ausgegangen werden,
dass der relative Anstieg von $7\%$ ausreichend ist.
\\
\\
Die bestimmte Totzeit liegt bei beiden Messmethoden in der gleichen Größenordnung. 
$\tau_1$ und $\tau_2$ liegen zwar nicht in den Fehlerbereichen voneinander, aber da die Werte trotzdem
nah beieinander liegen, können beide als Abschätzung gesehen werden.
\\
\\
Der Zusammenhang zwischen der Betriebsspannung und den freigesetzten Teilchen scheint annähernd linear zu sein.
\\
\\
Aufgefallen ist, dass der Strom $I$ erstens nicht genau abzulesen war und zweitens auch nicht
wirklich konstant geblieben ist. Dies könnte eventuelle Abweichungen bei \autoref{fig:totzeit} erklären.
\section*{Aufgabe 1}
    \subsection*{Aufgabenstellung}
        Berechnen Sie das Volumen eines Hohlyzlinders, mit R$_{innen} = (10 \pm 1)$ cm,\\    
        R$_{aussen}=(15 \pm 1)$ cm und h$=(20 \pm 1)$ cm.

    \subsection*{Rechnung}
        \begin{align}
        V &= \pi r_a^2 h - \pi r_i^2 h &ΔV &= \sqrt{\sum_{k=1}^n (\frac{∂V}{∂x_k}\cdotΔx_k)^2}\\
        r_{aussen} &= 15 \unit{cm} & Δr_{aussen} &= \SI{1}{cm}\\
        r_{innen} &= \SI{10}{cm} & Δr_{innen} &= \SI{1}{cm}\\
        h &= \SI{20}{cm} & Δh &= \SI{1}{cm}\\
        \end{align}
        \begin{equation}
            V = \pi \cdot \SI{20}{cm} \cdot ((\SI{15}{cm})^2 - (\SI{10}{cm})^2)^2 = 2500\pi
        \end{equation}
        \begin{equation}
            \frac{∂V}{∂r_a} = 2\pi r_a h \qquad \frac{∂V}{∂r_i} = -\pi r_i h \qquad \frac{∂V}{∂h} = \pi (r_a^2 - r_i^2)\\
        \end{equation}
        \begin{equation}
            ΔV = \sqrt{(\frac{∂V}{∂r_a}\cdotΔr_a)^2 + (\frac{∂V}{∂r_i}\cdotΔr_i)^2 + (\frac{∂V}{∂h}\cdotΔh)^2}\\
        \end{equation}
            $= \sqrt{(2\pi \cdot \SI{15}{cm} \cdot \SI{20}{cm} \cdot \SI{1}{cm})^2 + (-2\pi \cdot \SI{10}{cm} \cdot \SI{20}{cm} \cdot \SI{1}{cm})^2
            + (2\pi \cdot ((\SI{15}{cm})^2 - (\SI{10}{cm})^2)\cdot 1)^2}$ 
        \begin{equation}
            = \SI{2300}{cm^3}
        \end{equation}
        \begin{equation}
            V = \SI{7,9(2,3)e3}{cm^3}
        \end{equation}
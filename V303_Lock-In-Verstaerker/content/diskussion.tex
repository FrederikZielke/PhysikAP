\section{Diskussion}
\label{sec:Diskussion}

% Werte bei Noise waren geringer obwohl es ein Verstärker ist.
Die gemessenen Spannungen in Abhängigkeit der Phase zeigen, dass der Lock-In-Verstärker es ermöglicht effektiv 
störende Frequenzen herauszufiltern. Die Werte für $U_{outnoise}$ sind im Vergleich mit $U_{out}$ etwas kleiner.
%Vermutlich liegt dies daran, dass durch die generierte Noise Spannung abfällt.
Die Kurve entspricht einer Sinus- und nicht wie erwartet einer Cosinusfunktion. Die gemessenen Werte sind um 
$\frac{\pi}{2}$ zur Theorie verschoben. Dieser Umstand ist möglicherweise auf den Phase-Shifter zurückzuführen.
Die Abweichung des Parameters c in der Fitkurve lässt sich mit dieser Verschiebung erklären. Eine andere mögliche Erklärung 
ist, dass für $0°$ die integration des Tifpass auf $0$ festgelegt wurde.
\\
\\
Bei der Messung der Intensität der LED wurde ein Abfall der Intensität mit $r^{-3}$ beobachtet. Dieser Abfall ist 
größer als durch die Theorie erwartet. Diese würde $I \propto r^{-2}$ vorhersagen. 
Dieser Abfall ist möglicherweise auf den Lichteinfall 
durch die Fenster zurückzuführen, da wir den Lichtsensor mit zunehmender Entfernung auch von den Fenstern entfernt haben. Die angenommene Näherung
der Lichtquelle als Punktquelle wäre dann nicht mehr gültig\\
%
%bei led ist der Abfall mit r^-3?!
\newpage